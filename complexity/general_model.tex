\section{Общая модель вычислений и сложности}\label{sec:general_model}

В этом разделе будет определена структура вычислений и стоимости достаточно общая, чтобы объяснить все последующие примеры. Для множества $S$ обозначим через $fin(S)$ множества всех конечных подмножеств $S$.

\begin{definition}
  \textbf{Вычислительная структура} --- это множества $M$ вместе с отображением $\gamma:M \times fin(M) \to [0, \infty]$ таким, что:
  \begin{enumerate}[1.]
    \item $im(\gamma)$ хорошо упорядочено, то есть любое подмножество $im(\gamma)$ имеет минимум.
    \item $\gamma(\omega,U)=0$, если $\omega \in U$
    \item $U \subseteq V \Longrightarrow \gamma(\omega,V) \leq \gamma(\omega,U) $ для всех $\omega \in M; \; U,V \subseteq fin(M)$
  \end{enumerate}
\end{definition}

$M$ --- множество объектов, с которыми мы производим вычисления. $\gamma(\omega,U)$ --- стоимость вычисления $\omega$ при данном $U$ <<за один шаг>>. \\
$\gamma(\omega,U)=0$, если $\omega \in U$ потому, что не нужно платить за уже вычисленный элемент. $\gamma(\omega,U)=1$, если существуют $u,v \in U$ такие, что $\omega = u\; op\; v$, где $op$ обозначает какую-то арифметическую операцию, то есть $\omega$ вычисляется из $u$ и $v$ за одну арифметическую операцию. Во всех остальных случаях $\gamma(\omega,U)=\infty$, то есть нельзя вычислить $\omega$ <<за один шаг>> из $U$.

\begin{definition}
  \begin{enumerate}[1.]
	\item  Последовательность $\beta=(\omega_1, \ldots, \omega_m)$ элементов из $M$ называется \textbf{вычислением} или \textbf{алгоритмом} со входом $X \subseteq fin(M)$, если
    	\[
    		\forall j \leq m: \omega_j \in X \lor \gamma(\omega_j, V_j) < \infty \mbox{, где } V_j = \{ \omega_1, \ldots, \omega_{j-1} \}
    	\]
    	\item $\beta$ \textbf{вычисляет} множество $Y \in fin(M)$, если $Y \subseteq \{ \omega_1, \ldots, \omega_m \}$
    	\item \textbf{Стоимость} $\beta$ --- это $\Gamma(\beta, X) \overset{Def}{=} \sum\limits_{j=1}^{m} \gamma(\omega,V_j)$  
  \end{enumerate}
\end{definition}
В вычислении каждый $\omega_i$ может быть получен из вычисленных до этого элементов, то есть элементов $V_j$ или из $X$ (<<входа>>).

\begin{definition}
  \textbf{Сложность} $Y$ при данном $X$ определяется как
  \[
  	C(Y,X) \overset{Def}{=} min \{ \Gamma(\beta,X) \mid \beta \mbox{ вычисляет } Y \mbox{ по } X \}
  \]
\end{definition}
Сложность множества $Y$ --- это не более чем стоимость самого дешевого вычисления, с помощью которого можно получить $Y$. 

Если множество $X$ пустое или понятно из контекста, то вместо $C(Y,X)$ будем писать $C(Y)$. Если вычисляется один элемент, то есть $Y = \left\{ y \right\}$, то вместо $C(\left\{ y \right\},X)$ будет $C(y,X)$.

\textbf{Мерой Островского} называют вычислительную структуру, где $M = K(x_1, \ldots , x_n)$ --- поле рациональных функций от неизвестных $x_1, \ldots , x_n$ над полем $K$, и где есть четыре (или три) бинарных операции, а именно умножение, деление, сложение и вычитание. Деление --- частичная операция, определённая только если второй аргумент ненулевой. Для любого $\lambda \in K$ существует ещё унарная операция, а именно умножение на скаляр $\lambda$. Стоимости этих операций задаются так:
\begin{center}
  \begin{tabular}{c|c}
	  операция  & стоимость   \\
	  \hline
	  $\cdot,/$ & 1 \\
	  $+,-$ & 0 \\
	  $\lambda \cdot$ & 0
  \end{tabular}
\end{center}

Сложность, даваемая мерой Островского, будет обозначаться как $C^{*/}$, а если не используется деление, то $C^{*}$.


