\section{Экспонента матричного умножения}

Обозначим через $M_K(n)$ общее число арифметических операций $\left\{ \times , +, - \right\}$ необходимых чтобы перемножить две $n \times n$-матрицы над полем $K$.
\begin{definition}\label{def:omega}
	\textbf{Экспонентой матричного умножения над $K$} называют наименьшее действительное число $\omega(K) > 0$, определяемое как
	\[
		\omega(K) := \inf \left\{ \tau \in \mathbb{R}^+ \mid M_K(n) = O(n^\tau), n \to \infty \right\}
	\]
\end{definition}
Обозначение $\omega(K)$ призвано показать возможную зависимость от базового поля $K$. Было доказано, что $\omega(K)$ не измениться, если мы заменим $K$ на любое алгебраическое расширение $\overline{K}$ \cite[383]{bur}. Также было доказано, что $\omega(K)$ определяется только характеристикой $Char(K)$ поля $K$, поэтому $\omega(K) = \omega(\mathbb{Q})$, если $Char(K) = 0$, и $\omega(K) = \omega(\mathbb{Z}_p)$ в противном случае, где $\mathbb{Z}_p$ --- это конечное поле целых чисел по модулю простого $p$ с характеристикой $p$ \cite{Pan1984}. Так как $Char(\mathbb{C}) = Char(\mathbb{R}) = Char(\mathbb{Q}) = 0$, это означает, что $\omega(\mathbb{C}) = \omega(\mathbb{R}) = \omega(\mathbb{Q})$. В этом подразделе мы продолжим писать $\omega(K)$, чтобы показать зависимость от базового поля, но в следующих разделах оставим формализм и будем писать просто $\omega$, потому как дальше речь пойдёт исключительно о произведениях комплексных матриц.

Но вернёмся к $M_K(n)$. Чтобы получить произведение $n \times n$-матриц $A B = C$ при помощи стандартного алгоритма, потребуется вычислить $n^2$ элементов вида $C_{ik} = \sum_{1 \leq j \leq n} A_{ij} B_{jk}$ для всех $1 \leq i,k \leq n$. Это потребует $n^3$ умножений и $n^3 - n^2$ сложений, что даёт верхнюю оценку $M_K(n) = 2n^3 - n^2 < 2 n^3 = O(n^3)$, то есть $M_K(n) < C' n^3$ для константы $C' = 2$, и означает верхнюю оценку $\omega(K) \leq 3$. Для нижней оценки заметим, что так как произведение двух $n \times n$-матриц состоит из $n^2$ элементов, то общее число операций будет по меньшей мере равнo $C n^2$ для $C \geq 1$, что обозначается как $M_K(n) = \Omega(n^2)$ и эквивалентно нижней оценке $2 \leq \omega(K)$ \cite[375]{bur}. Неформально мы доказали следующий элементарный результат:

\begin{prop}\label{prop:mol:2.11}
     Для любого поля $K$
	\begin{enumerate}
	     \item $2 \leq \omega(K) \leq 3$
	     \item $\omega(K) = h \iff \Omega(n^{h + \varepsilon}) = M_K(n) = O(n^{h + \varepsilon})$, где $h$ минимально для любого $\varepsilon > 0$
	\end{enumerate}
\end{prop}

Следующее утверждение связывает экспоненту и концепцию билинейного ранга:
\begin{prop}\label{prop:bur:15.1}(утверждение 15.1 из \cite[376-377]{bur})
Для любого поля $K$
	\[
		\omega(K) = \inf \{ \tau \in \mathbb{R} \mid R(\langle n,n,n \rangle) = O(n^\tau), n \to \infty \},
	\]
	где $\langle n,n,n \rangle$ обозначает тензор матричного умножения двух $n \times n$-матриц над полем $K$.  
\end{prop}
Это означает, что для данного поля $K$ и любого $\varepsilon > 0$ существует константа $C_{K, \varepsilon}$, независящая от $n$ такая, что $R(\left\langle n,n,n \right\rangle) \leq C_{K,\varepsilon} n^{\omega(K)+\varepsilon}$ для всех $n$. Есть гипотеза, что $\omega(\mathbb{C}) = 2$. С этого момента $\omega$ будет обозначать $\omega(\mathbb{C})$.

