\section{Билинейные задачи}

\begin{definition}\cite{wiki:qf}
  \textbf{Квадратичной формой} называют однородный многочлен степени 2 от нескольких переменных. Например,
  \[
  	4x^2+2xy-3y^2
  \]
  будет квадратичной формой от переменных $x$ и $y$.
\end{definition}

Пусть $K$ --- поле, обычно оно называется полем скаляров, и пусть $M=K(x_1, \ldots, x_N)$. Мера Островского потребуется для следующего. Мы будем задаваться вопросами вида
\[
	C^{*/}(F)=?,
\]
где $F=\{ f_1, \ldots, f_k \}$ --- множество квадратичных форм
\[
	f_\kappa = \sum\limits_{\mu,\nu=1}^{N} t_{\kappa \mu \nu} x_\mu x_\nu.
\]

Большую часть времени мы будем рассматривать особый случай билинейных форм, то есть наши переменные разделены на два непересекающихся множества и только произведения одной переменной из первого множества и одной из второго встречаются в $f_k$.

<<Трехмерный массив>> $t:=(t_{\kappa \mu \nu})_{\kappa = 1, \ldots, k;  \mu ,\nu = 1, \ldots, N } \in K^{k \times N \times N}$ называется \textbf{тензором, соответствующим $F$}. Так как $x_\mu x_\nu = x_\nu x_\mu$, существует несколько тензоров, представляющих одну и ту же задачу $F$. Тензор $s$ является \textbf{симметрически эквивалентным} тензору $t$, если
\[
	s_{\kappa \mu \nu} + s_{\kappa \nu \mu} = t_{\kappa \mu \nu} + t_{\kappa \nu \mu} \mbox{ для всех } \kappa, \mu, \nu.
\]

Два тензора описывают одно и тоже множество квадратичных форм, если они симметрически эквивалентны.


