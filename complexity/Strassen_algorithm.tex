\section{Алгоритм Штрассена}\label{Strassen_algorithm}

\begin{definition}\cite{wiki:bf}
	Если $V$ --- это векторное пространство над полем скаляров $F$, то \textbf{билинейной формой} называют функцию $f: V \times V \to F$, которая удовлетворяет следующим условиям:
	\begin{enumerate}[(1)]
	     \item $f(u + v, w) = f(u, w) + f(v, w), f(u, v + w) = f(u, v) + f(u, w)$ и $f(kv, w) = kf(v, w) = f(v, kw)$ для всех $u, v, w \in V$ и всех $v, w \in V$.
	     \item $f$ называется \textbf{симметричной билинейной формой}, если она также удовлетворяет:
	     $f(v, w) = f(w, v)$ для всех $v, w \in V$
	\end{enumerate}
\end{definition}

Пусть даны две $n \times n$-матрицы $X=(x_{ik})$ и $Y=(y_{kj})$, чьи элементы являются неизвестными из поля $K$. Мы хотим вычислить их произведение $XY=(z_{ij})$. Элементы $z_{ij}$ задаются следующими хорошо известными билинейными формами:
\[
	z_{ij}= \sum\limits_{k=1}^{n} x_{ik} y_{kj}, \; 1 \leq i,j \leq n.
\]

Каждый $z_{ij}$ --- это сумма $n$ произведений, поэтому чтобы вычислить $z_{ij}$ нужно $n$ умножений и $n-1$ сложение. В итоге получим алгоритм, требующий в общей сложности $n^3$ умножений и $n^2(n-1)$ сложение. Этот алгоритм выглядит так естественно и интуитивно, что очень сложно представить, что существует лучший способ перемножения матриц. Однако в 1969 году Штрассен \cite{Strassen:1969} придумал как перемножить $2 \times 2$-матрицы всего за 7 умножений, но 18 сложений.

Пусть $z_{ij}\; 1 \leq i,j \leq 2$ задаются как
\[
	\left( 
	\begin{array}{cc}
	  z_{11} & z_{12}\\
	  z_{21} & z_{22}
	\end{array}
	 \right)=
	 \left( 
	\begin{array}{cc}
	  x_{11} & x_{12}\\
	  x_{21} & x_{22}
	\end{array}
	 \right)
	 \left( 
	\begin{array}{cc}
	  y_{11} & y_{12}\\
	  y_{21} & y_{22}
	\end{array}
	 \right).
\]
Вычислим семь произведений
\[
	\begin{array}{l}
	  p_1=(x_{11}+x_{22})(y_{11}+y_{22})\\
  	  p_2=(x_{11}+x_{22})y_{11}\\
  	  p_3=x_{11}(y_{12}-y_{22})\\
  	  p_4=x_{22}(-y_{11}+y_{12})\\
  	  p_5=(x_{11}+x_{12})y_{22}\\
  	  p_6=(-x_{11}+x_{21})(y_{11}+y_{12})\\
  	  p_7=(x_{12}-x_{22})(y_{21}+y_{22})\\
	\end{array}
\]
и тогда можно выразить каждый из $z_{ij}$ как линейную комбинацию этих семи произведений, а именно
\[
	\left( 
	\begin{array}{cc}
	  z_{11} & z_{12}\\
	  z_{21} & z_{22}
	\end{array}
	 \right)=
	 \left( 
	\begin{array}{cc}
	  p_1+p_4-p_5+p_7 & p_3+p_5\\
	  p_2+p_4 & p_1+p_3-p_2+p_6
	\end{array}
	 \right).
\]

Возникает вопрос, стоит ли сохранять одно умножение, чтобы получить 14 дополнительных сложений? Важным моментом является то, что алгоритм Штрассена работает не только над полями, но и над некоммутативными кольцами. В частности, элементы $2 \times 2$-матриц могут сами быть матрицами, и можно применять этот алгоритм рекурсивно. А для матриц умножение --- по крайней мере если использовать наивный алгоритм --- гораздо дороже сложения, а именно \hyperref[o_notation]{$O(n^3)$} против $n^2$.










