\section{Реализация матричного умножения с помощью групп}\label{sec:realizing}

Если $S$ --- подмножество группы, то пусть $Q(S)$ обозначает множество правых частных $S$, то есть $Q(S)=\{ s_1 s_2^{-1} \mid s_1,s_2 \in S\}$.

\begin{definition}\label{def:tpp}
  \textbf{Группа $G$ реализует} $\langle n_1, n_2, n_3 \rangle$, если существуют подмножества $S_1, S_2, S_3 \subseteq G$ такие, что $|S_i| = n_i$, и для всех $q_i \in Q(S_i)$, если 
  \[
  	q_1 q_2 q_3 = 1,
  \]
  тогда $q_1 = q_2 = q_3 = 1$. Это условие на $S_1, S_2, S_3$ называется \textbf{свойством тройного произведения}  (англ. triple product property).
\end{definition}

В большинстве примеров матричное умножение будет реализовываться через подгруппы $H_1, H_2, H_3$ группы $G$, а не через случайные подмножества. В этом случае свойство тройного произведения особенно просто, потому что $Q(H_i)=H_i$: оно гласит, что если $h_1 h_2 h_3 = 1$ и $h_i \in H_i$, тогда $h_1 = h_2 = h_3 = 1$. Эквивалентная формулировка заменяет $h_1 h_2 h_3 = 1$ на $h_1 h_2 = h_3$.

Возможно самым простым примером будет произведение $Cyc_n \times Cyc_m \times Cyc_p$ циклических групп, которое очевидно реализует $\langle n, m, p \rangle$ через $Cyc_n \times \{1\} \times \{1\}$, $\{1\} \times Cyc_m \times \{1\}$ и $\{1\} \times \{1\} \times Cyc_p$.

\begin{lemma}
  Если $G$ реализует $\langle n_1, n_2, n_3 \rangle$, то это будет верно также для любой перестановки $n_1, n_2$ и $n_3$.
\end{lemma}
\begin{proof}
Положим, что $G$ реализует $\langle n_1, n_2, n_3 \rangle$ через $S_1, S_2, S_3$ и положим, что $s_i, s_i' \in S_i$. Нужно доказать, что порядок, в котором появляются 1, 2 и 3 в уравнении
\[
	s_1' s_1^{-1} s_2' s_2^{-1} s_3' s_3^{-1}=1,
\]
не имеет значения. Сопряжение элементом $s_1' s_1^{-1}$ покажет, что это уравнение эквивалентно
\[
	s_2' s_2^{-1} s_3' s_3^{-1} s_1' s_1^{-1} =1,
\]
поэтому мы можем выполнять циклический сдвиг индексов. Чтобы получить транспозицию, возведём в $-1$ степень изначальное уравнение, что даст
\[
	s_3 s_3'^{-1} s_2 s_2'^{-1} s_1 s_1'^{-1}=1,
\]
то есть транспозицию 1 и 3 ($s_i$ и $s_i'$ поменялись ролями, но это не имеет значения). Эти две перестановки порождают все перестановки $\{1, 2, 3 \}$, это следует из элементарного факта теории групп, что $S_3 = \langle (1\;2\;3), (1\;3) \rangle$. 
\end{proof}

\begin{lemma}\label{lem:03:2.2}
  Если $N$ --- нормальная подгруппа $G$, которая реализует $\langle n_1, n_2, n_3 \rangle$ и $G/N$ реализует $\langle m_1, m_2, m_3 \rangle$, тогда $G$ реализует $\langle n_1m_1, n_2m_2, n_3m_3\rangle$.
\end{lemma}
\begin{proof}
Положим, что $N$ реализует $\langle n_1, n_2, n_3 \rangle$ через $S_1, S_2, S_3$ и пусть $G/N$ реализует $\left\langle m_1,m_2,m_3 \right\rangle$ через множества $U_1, U_2, U_3$, состоящие из смежных классов подгруппы $N$, то есть
\[
	U_i = \left\{ u_i = v_i N \mid v_i \in G \right\}
\]
Определим \textbf{подъёмы} $T_i \subseteq G$ множеств $U_i$ как
\[
	T_i = \left\{ t_i \in G \mid t_i = v_i n_i \text{ для какого-то } n_i \in N \text{ и } u_i = v_i N \in U_i \text{ для какого-то } v_i \in G  \right\}.
\]
Другими словами в $T_i$ попадает лишь по одному представителю из каждого смежного класса, входящего в $U_i$. Множества $T_i$, которые необязательно будут уникальными для $U_i$, обладают свойствами 
\begin{enumerate}
     \item $t_i N = u_i$ для всех $t_i \in T_i$ и $u_i \in U_i$
     \item $t \in T_i$ означает $T_i \cap t N = \left\{ t \right\}$.
\end{enumerate}
Утверждается, что $G$ реализует $\langle n_1m_1, n_2m_2, n_3m_3\rangle$ через поточечные произведения $S_1T_1, S_2T_2, S_3T_3$. Необходимо проверить, что для $s_i, s_i' \in S_i$ и $t_i, t_i' \in T_i$
\begin{equation}\label{eq:lem2.2}
	(s_1't_1')(s_1t_1)^{-1}(s_2't_2')(s_2t_2)^{-1}(s_3't_3')(s_3t_3)^{-1}=1_G
\end{equation}
тогда и только тогда, когда $s_i = s_i'$ и $t_i = t_i'$. 

Так как $s_i \in N$ и поэтому $s_iN = N$, то после приведения этого уравнения по модулю $N$ получим
\begin{align*}
     t_1'N(t_1N)^{-1}t_2'N(t_2N)^{-1}t_3'N(t_3N)^{-1} & = N\\
     u_1' u_1^{-1} u_2' u_2^{-1} u_3' u_3^{-1} & = 1_{G/N}.
\end{align*}
Множества $U_i$ удовлетворяют свойству тройного произведения в группе $G/N$, поэтому $u_i' = u_i$, откуда получим $t_i'N=t_iN$. Во множестве $T_i$ есть только по одному представителю от каждого смежного класса, поэтому $t_i'N=t_iN$ тогда и только тогда, когда $t_i'=t_i$. Следовательно уравнение \eqref{eq:lem2.2} в $G$ будет иметь вид
\[
	s_1' s_1^{-1} s_2' s_2^{-1} s_3' s_3^{-1}=1_G=1_H,
\]
откуда следует, что $s_i' = s_i$, так как множества $S_i$ удовлетворяют свойству тройного произведения.
\end{proof}

Один полезный особый случай леммы \ref{lem:03:2.2}: пусть $G_1$ реализует $\langle n_1, m_1, p_1 \rangle$ и $G_2$ реализует $\langle n_2, m_2, p_2 \rangle$, тогда $G_1 \times G_2$ реализует $\langle n_1n_2, m_1m_2, p_1p_2 \rangle$.

Следующая теорема описывает вложение матричного умножения в умножение в групповой алгебре.

\begin{theorem}\label{th:03:2.3}
  Пусть $F$ --- любое поле. Если $G$ реализует $\langle n, m, p \rangle$, тогда число операций необходимых для перемножения $n \times m$ и $m \times p$-матриц над полем $F$ не превосходит числа операций необходимых для перемножения элементов $F[G]$. Более того $\left\langle n,m,p \right\rangle_F \leq F[G]$.
\end{theorem}

Определение отношения <<$\leq$>> из последнего предложения смотрите на странице \pageref{def:bi:6.2}.

\begin{proof}
Пусть $G$ реализует $\langle n, m, p \rangle$ через подмножества $S, T, U$, положим $A$ --- $n \times m$-матрица и $B$ --- $m \times p$-матрица. Будем индексировать строки и столбцы матрицы $A$ элементами множеств $S$ и $T$ соответственно, строки и столбцы $B$ --- множествами $T$ и $U$, а строки и столбцы $AB$ --- множествами $S$ и $U$.

Рассмотрим произведение
\[
	\left( \sum_{s \in S, t \in T} A_{st} s^{-1}t \right)\left( \sum_{t' \in T, u \in U} B_{t'u} t'^{-1}u \right)
\]
в групповой алгебре. Мы имеем
\begin{align*}
     (s^{-1}t)(t'^{-1}u) & = s'^{-1}u'\\
     s's^{-1}tt'^{-1}uu'^{-1} & = 1
\end{align*}
тогда и только тогда, когда $s=s', t=t', u=u'$ поэтому коэффициент при $s^{-1}u$ в произведении равен
\[
	\sum_{t \in T} A_{st}B_{tu} = (AB)_{su}.
\]

Следовательно можно легко получить матричное произведение из произведения в групповой алгебре, если посмотреть на коэффициенты при $s^{-1}u$, где $s \in S, u \in U$. Это доказывает утверждение теоремы.
\end{proof}


\begin{theorem}\label{th:05:1.8}
  Пусть $G$ реализует $\langle n, m, p \rangle$, и $\{ d_i \}$ --- это степени характеров $G$. Тогда
  \[
  	(nmp)^{\omega/3} \leq \sum\limits_i d_i^\omega
  \]
\end{theorem}
\begin{proof}
	По теореме \ref{th:03:2.3}
	\begin{equation}
	  \langle n,m,p \rangle \leq \mathbb{C}[G] \simeq \bigoplus_i \langle d_i, d_i, d_i \rangle \label{inq:1}
	\end{equation}

	Далее потребуется два факта о ранге матричного умножения:\\
	первый (теорема \ref{th:4.7}) о том, что для всех $n',m',p'$
	\[
		(n'm'p')^{\omega/3} \leq R(\langle n',m',p' \rangle),
	\]
	второй (утверждение \ref{prop:bur:15.1}) о том, что для любого $\varepsilon > 0$ существует такое $C > 0$, что для всех $k$,
	\[
		R(\langle k,k,k \rangle) \leq C k ^{\omega+\varepsilon}.
	\]

	Если мы используем равенство
	\[
		\langle n_1, m_1, p_1 \rangle \otimes \langle n_2, m_2, p_2 \rangle \simeq \langle n_1n_2, m_1m_2, p_1p_2 \rangle,
	\]
	то получим, что $\ell$-ая степень тензора \eqref{inq:1} равна
	\[
		\langle n^\ell , m^\ell , p^\ell  \rangle \leq \bigoplus_{i_1, \dotsc, i_\ell } \langle d_{i_1} \dotsm d_{i_\ell }, d_{i_1} \dotsm d_{i_\ell }, d_{i_1} \dotsm d_{i_\ell } \rangle.
	\]
	Если взять ранг от левой части этого неравенства и воспользоваться теоремой \ref{th:4.7}, получим
	\[
		R(\langle n^\ell , m^\ell , p^\ell  \rangle) \geq (n^\ell m^\ell p^\ell )^{\omega/3} = (nmp)^{\omega \ell / 3}.
	\]
	Если взять ранг от правой и использовать утверждение \ref{prop:bur:15.1}, то 
	\begin{align*}
	     R(\bigoplus_{i_1, \dotsc, i_\ell } \langle  d_{i_1} \dotsm d_{i_\ell }, d_{i_1} \dotsm d_{i_\ell }, d_{i_1} \dotsm d_{i_\ell } \rangle) & \leq 
			\sum\limits_{i_1, \dotsc, i_\ell } R(\langle d_{i_1} \dotsm d_{i_\ell }, d_{i_1} \dotsm d_{i_\ell }, d_{i_1} \dotsm d_{i_\ell } \rangle) \\
			& \leq 	\sum\limits_{i_1, \dotsc, i_\ell } C (d_{i_1} \dotsm d_{i_\ell })^{\omega + \varepsilon} \\
			& \leq C \left( \sum\limits_i d_i^{\omega + \varepsilon} \right)^\ell.
	\end{align*}
	\begin{question}
	  Правильно ли я тут извлёк ранги?
	\end{question}
	После объединения этих двух результатов получим
	\[
		(nmp)^{\omega \ell /3} \leq C \left( \sum\limits_i d_i^{\omega + \varepsilon} \right)^\ell .
	\]
	Если теперь извлечь $\ell$-ый корень и устремить $\ell$ в бесконечность, получим
	\[
		(nmp)^{\omega/3} \leq \sum\limits_i d_i^{\omega + \varepsilon},
	\]
	так как $\lim_{\ell \to \infty} \sqrt[\ell]{C} = 1$. Наконец, потому как это неравенство выполняется для всех $\varepsilon > 0$, оно должно выполняться также для $\varepsilon = 0$ по непрерывности.
\end{proof}

\begin{corollary}\label{cor:05:1.9}
  Пусть $G$ реализует $\langle n,m,p \rangle$ и имеет наибольшую степень характера равную $d$. Тогда $(nmp)^{\omega / 3} \leq d ^{\omega-2} |G|$.
\end{corollary}
\begin{proof}
По теореме \ref{th:05:1.8}
\[
	(nmp)^{\omega/3} \leq \sum\limits_i d_i^\omega = \sum\limits_i d_i^2 d_i^{\omega - 2} \leq \sum\limits_i d_i^2 d^{\omega - 2} = d^{\omega - 2} \sum\limits_i d_i^2 = d^{\omega - 2}|G|. \qedhere
\]
\end{proof}

Нужно заметить, что теорема \ref{th:05:1.8} даст нетривиальную оценку на $\omega$, исключив возможность $\omega=3$, тогда и только тогда, когда 
\[
	nmp > \sum\limits_i d_i^3.
\]






