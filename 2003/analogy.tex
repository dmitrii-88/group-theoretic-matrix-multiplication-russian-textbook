\section{Аналогия с быстрым умножением многочленов}

Существует близкое сходство между тем, что было предложено Коном и Умансом в их статье 2003 года, и хорошо известным алгоритмом перемножения двух многочленов за $O(n \log n)$ операций с использованием быстрого преобразования Фурье (БПФ). Текущий раздел будет посвящён этой аналогии, которая поможет получить общее представление о технике, изложенной в статье.

Пусть мы хотим перемножить многочлены $A(x)=\sum_{i=0}^{n-1} a_i x^i$ и $B(x)=\sum_{i=0}^{n-1} b_i x^i$. Наивный способ потребует вычисления $n^2$ произведений вида $a_i b_j$ и получения из них $2n-1$ коэффициента $A(x)B(x)$. Конечно, существуют гораздо лучшие алгоритмы. Ниже будет описан один из таких. Вдобавок его будет легко перевести в схему, разработанную Коном и Умансом.

Пусть $G$ --- группа и пусть $\mathbb{C}[G]$ --- групповая алгебра, то есть каждый элемент $\mathbb{C}[G]$ является формальной суммой $\sum_{g \in G} a_g g$, где $a_g \in \mathbb{C}$, и произведение двух элементов будет вычисляться следующим образом:
\[
	\left( \sum_{g \in G} a_g g \right) \cdot \left( \sum_{h \in G} b_h h \right) = \sum_{f \in G} \left( \sum_{gh=f} a_g b_h \right) f.
\]
Подробнее групповые алгебры описаны на странице \pageref{group_algebra}.

Мы часто будем отождествлять элемент $\sum_{g \in G} a_g g$ с вектором его коэффициентов. Если $G$ --- циклическая группа порядка $m$, то произведение двух элементов $a=(a_g)_{g \in G}$ и $b=(b_g)_{g \in G}$ является циклической свёрткой векторов $a$ и $b$. Важно заметить, что циклическая свёртка --- это почти то, что нам нужно для вычисления произведения $A(x)B(x)$. Единственная проблема в том, что она закольцовывается. Чтобы избежать этого, мы представим $A(x)$ и $B(x)$ как элементы $\overline{A}, \overline{B} \in \mathbb{C}[G]$ следующим образом: пусть $z$ является порождающим элементом группы $G$, которая будет циклической группой порядка $m > 2n-1$, и определим
\[
	\overline{A}=\sum_{i=0}^{n-1} a_i z^i \text{ и } \overline{B}=\sum_{i=0}^{n-1} b_i z^i.
\]
Так как порядок $m$ группы достаточно большой, чтобы избежать закольцовывания, мы можем получить коэффициенты произведения многочленов из $\overline{A} \cdot \overline{B} \in \mathbb{C}[G]$: коэффициент при $x^i$ в $A(x)B(x)$ будет равен коэффициенту при $z^i$ в $\overline{A} \cdot \overline{B}$. Казалось бы мы зря потратили столько слов, чтобы описать такое простое соответствие, но вознаграждение близко. Дискретное преобразование Фурье (ДПФ) для $\mathbb{C}[G]$ будет обратимым линейным преобразованием $D:\mathbb{C}[G] \to \mathbb{C}^{|G|}$, которое переводит умножение в $\mathbb{C}[G]$ в поточечное умножение векторов в $\mathbb{C}^{|G|}$. Поэтому мы можем получить $\overline{A} \cdot \overline{B}$, вычислив сперва $D(\overline{A})$ и $D(\overline{B})$, а затем вычислив обратное ДПФ их поточечного произведения. Таким образом, используя $O(m \log m)$-алгоритм быстрого преобразования Фурье, можно произвести умножение в $\mathbb{C}[G]$ (и поэтому умножение многочленов через выше описанное вложение) за $O(m \log m)$ операций.

Одним из основных результатов статьи 2003 года, стало то, что \textit{аналогичным способом можно матричное умножение вложить в умножение в групповой алгебре}. Это вложение будет не таким простым как в случае с многочленами, но оно будет просто и естественно описываться в терминах свойств подмножеств группы $G$ (которые часто будут подгруппами). В частности, если $S,T$ и $U$ --- подмножества $G$, а $A=(a_{s,t})_{s \in S, t \in T}$ и $B=(b_{t,u})_{t \in T, u \in U}$ --- $|S| \times |T|$ и $|T| \times |U|$-матрицы соответственно, тогда определим
\[
	\overline{A}=\sum a_{s,t} s^{-1}t \text{ и } \overline{B}=\sum b_{t,u} t^{-1}u.
\]
Если $S,T,U$ удовлетворяют свойству тройного произведения (определение \ref{def:tpp}), то элементы матричного произведения $AB$ можно получить из $\overline{A} \cdot \overline{B} \in \mathbb{C}[G]$: элемент $(AB)_{s,u}$ будет просто коэффициентом группового элемента $s^{-1}u$.

В случае умножения многочленов простота вложения скрывала тот факт, что если $G$ будет слишком большой (например, если $|G|=n^2$, а не $O(n)$), то всё преимущество схемы пропадает. Поэтому основным испытанием становится то, как бы не попасть в эту ловушку. Мы хотим вложить матричное умножение в групповую алгебру над небольшой группой $G$, так как размер $G$ будет нижней оценкой сложности умножения в $\mathbb{C}[G]$. Например, нет ничего удивительного в том, что умножение $n \times n$-матриц можно вложить в групповую алгебру группы порядка $n^3$. Кон и Уманс показали, что абелевы группы не смогут побить $n^3$, но существуют семейства неабелевых групп размера $n^{2+o(1)}$, которые допускают такое вложение.

Может показаться, что этот результат вместе с выше описанной уловкой для выполнения умножения в групповой алгебре (то есть взятие ДПФ, умножение в области Фурье и, наконец, обратное преобразование) будут означать, что $\omega=2$. Однако существуют две сложности, которые появляются из-за того, что приходится работать с неабелевыми группами. Первая в том, что \textit{быстрые алгоритмы} для вычисления ДПФ известны только для ограниченного класса неабелевых групп (смотри раздел 13.5 в \cite{bur}). Однако \textit{обычное} ДПФ будет линейным, и из-за рекурсивной структуры алгоритма <<разделяй и властвуй>> для матричного умножения, линейные преобразования, применяемые до и после рекурсивного шага, будут <<бесплатны>>. Например, в первоначальном алгоритме Штрассена число матричных сложений и скалярных умножений на рекурсивном шаге не влияло на оценку $\omega$. Поэтому эта потенциальная сложность не будет являться проблемой.

Вторая сложность в том, что для $\mathbb{C}[G]$, когда $G$ неабелева, умножение в области Фурье не будет простым поточечным умножением векторов в $\mathbb{C}^{|G|}$. Вместо этого оно будет блочно-диагональным матричным умножением, где размеры блоков являются степенями неприводимых представлений $G$. Таким образом, умножение $n \times n$-матриц сводится к нескольким умножениям матриц меньшего размера, что даёт неравенство с участием матричной экспоненты $\omega$. Если размер $G$ был бы в точности $n^2$, то из этого неравенства получалось бы, что $\omega=2$. Однако наименьший возможный порядок $G$ будет $n^{2+o(1)}$, и тогда вопрос о том, будет ли неравенство означать $\omega=2$, переводится в область теории представлений $G$. Кон и Уманс показали, что, когда $|G|=n^{2+o(1)}$, даже небольшого контроля над степенью наибольшего неприводимого представления будет достаточно, чтобы достичь $\omega=2$. Какой-то контроль будет необходим всегда, чтобы избегать тривиальностей вроде сведения матричного произведения к задаче перемножения матриц большего размера.


