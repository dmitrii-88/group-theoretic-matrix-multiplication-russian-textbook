\section{Оценка экспоненты матричного умножения с помощью $s$-ранга}

В этом разделе будет показано, что верхняя оценка того, что авторы назвали <<рангом носителя>> или $s$-рангом тензора матричного умножения, будет давать оценку $\omega$.
\begin{definition}
 \textbf{Носителем тензора $T$} (обозначается $supp(T)$, сокращение от английского слова support) называется множество одночленов, которые имеют ненулевые коэффициенты. Например, пусть $T=3 \widehat{x}_1 \widehat{y}_2+\widehat{x}_2 \widehat{y}_1$, тогда $supp(T)=\left\{  \widehat{x}_1 \widehat{y}_2, \widehat{x}_2 \widehat{y}_1\right\}$.  
\end{definition}
Конечно, $supp(T)$ зависит от выбора базиса, и поэтому не будет инвариантой относительно изоморфизмов, тоже самое будет верно и для других концепций, использующих носитель в своём определении. Однако эта зависимость от базиса не будет проблемой для теории алгебраической сложности. В конце концов любая вычислительная задача должна определять выбор базиса, который используется для входных и выходных значений, также запись тензора как мультилинейной формы уже неявно подразумевает выбор базиса (выбор переменных).
\begin{definition}
 \textbf{$s$-ранг} $R_s(T)$ тензора $T$ является минимальным рангом для всех тензоров $T'$ таких, что $supp(T)=supp(T')$.  
\end{definition}

Очевидно, что $s$-ранг не может превосходить обычный ранг. Далее следует простой пример, показывающий, что $s$-ранг может быть значительно меньше как обычного так и граничного рангов:
\begin{prop}
  \label{prop:12:3.2} Пусть $J$ --- матрица из одних единиц, а $I$ --- единичная матрица, тогда $n \times n$-матрица $J-I$ имеет ранг $n$ и граничный ранг $n$, в то время как её $s$-ранг равен 2.
\end{prop}
\begin{proof}
 Ранг и граничный ранг совпадают для матриц (матрицы, не превосходящие ранга $r$, характеризуются условием на их определитель и поэтому образуют замкнутое множество), и $J-I$ имеет ранг $n$. Рассмотрим матрицу 
 \[
   M = 
   \begin{pmatrix}
   	1 & \zeta^{n-1} & \cdots & \zeta^1 \\
   	\zeta^1 & 1 & \cdots & \zeta^2\\
   	\vdots & \vdots & \ddots & \vdots \\
   	\zeta^{n-1} & \zeta^{n-2}  & \cdots & 1
   \end{pmatrix},
 \] 
задаваемую как $M_{i,j}=\zeta^{i-j}$, где $\zeta$ --- примитивный $n$-ый корень из 1 (корни из 1 образуют циклическую группу по умножению, все элементы которой будут степенями примитивного элемента). Ранг $M$ равен 1, так как если домножить её $i$-ую строку на $\zeta^{j-i}$, получается её $j$-ая строка. 

 Тогда $M-J$ имеет тот же носитель, что и $J-I$, так как
 \begin{align*}
   M-J & = 
   \begin{pmatrix}
   	0 & \zeta^{n-1} - 1 & \cdots & \zeta^1 -1 \\
   	\zeta^1 -1 & 0 & \cdots & \zeta^2 -1\\
   	\vdots & \vdots & \ddots & \vdots \\
   	\zeta^{n-1} -1 & \zeta^{n-2} -1 & \cdots & 0
   \end{pmatrix} \\
   J - I & = 
   \begin{pmatrix}
   	0 & 1 & \cdots & 1 \\
   	1 & 0 & \cdots & 1 \\
   	\vdots & \vdots & \ddots & \vdots \\
   	1 & 1 & \cdots & 0
   \end{pmatrix}. 
 \end{align*}
 Потому как $M$ и $J$ обе имеют ранг 1, $s$-ранг $J-I$ не превосходит 2. Также легко убедиться, что не существует матрицы с носителем как у $J-I$ и рангом 1, поэтому $s$-ранг будет в точности 2.
\end{proof}

С другой стороны граничный ранг может быть меньше $s$-ранга, поэтому два эти ослабления нельзя сравнивать:
\begin{prop}
  \label{prop:12:3.3} Тензор $T = \widehat{x}_0 \widehat{y}_0 \widehat{z}_0 + \widehat{x}_0 \widehat{y}_1 \widehat{z}_1 + \widehat{x}_1 \widehat{y}_0 \widehat{z}_1$ имеет граничный ранг 2 и $s$-ранг 3.
\end{prop}
\begin{proof}
  Простое доказательство того, что граничный ранг равен 2 есть на странице \pageref{border_rank_2} данного пособия. Чтобы показать, что $s$-ранг больше равен 3, будем использовать метод подстановок так же как при доказательстве, что обычный ранг равен 3. У любого тензора $T'$ с носителем как у $T$ в декомпозиции в сумму тензоров ранга 1 один из тензоров должен зависеть от $\widehat{x}_1$. Мы можем сделать этот тензор нулевым, если подставим скалярное кратное $\widehat{x}_0$ вместо $\widehat{x}_1$. После этой замены $T'$ всё ещё зависит от $\widehat{y}_1$. Можно сделать его нулевым, подставив скалярное кратное $\widehat{y}_0$ вместо $\widehat{y}_1$. После двух этих подстановок $T'$ всё ещё зависит от $\widehat{z}_0$, поэтому по меньшей мере должен быть ещё один тензор с рангом 1 в декомпозиции. Очевидно, что $s$-ранг меньше равен 3. В итоге получим, что $R_s(T) = 3$.
\end{proof}

Также как обычный ранг, $s$-ранг является субаддитивным и субмультипликативным в том смысле, что для тензоров $T$ и $T'$ получим $R_s(T \oplus T') \leq R_s(T) + R_s(T')$ и $R_s(T \otimes  T') \leq R_s(T) R_s(T')$. Для тензоров матричного умножения имеем $R_s(\left\langle \ell, m, n \right\rangle) = R_s(\left\langle \ell', m', n' \right\rangle)$ для любой перестановки $(\ell', m', n')$ тройки $(\ell, m, n)$. По аналогии с экспонентой матричного умножения $\omega$, определим $\omega_s$.
\begin{definition}
 \textbf{$s$-ранговой экспонентой} матричного умножения (обозначается $\omega_s$) называется 
 \[
 	\omega_s = \inf \left\{ \tau \in \mathbb{R} \mid R_s(\left\langle n,n,n \right\rangle) = O(n^\tau) \right\}.
 \]
\end{definition}

Экспоненту $\omega$ можно определить тем же способом, если заменить в этом определении $s$-ранг на обычный ранг. Так как каждый тензор с тем же носителем, что и у $\left\langle n,n,n \right\rangle$, имеет $n^2$ линейно независимых слоёв, получим, что $R_s(\left\langle n,n,n \right\rangle) \geq n^2$, и поэтому $2 \leq \omega_s \leq \omega$.

Как и следовало ожидать, верхняя оценка $s$-ранга даст верхнюю оценку $\omega_s$. Ниже приведена $s$-ранговая версия стандартного доказательства:
\begin{prop}
  \label{prop:12:3.5}Для всех $\ell, m, n$ получим, что $(\ell m n)^{\omega_s / 3} \leq R_s(\left\langle \ell, m, n \right\rangle)$.
\end{prop}
\begin{proof}
  Пусть $r = R_s(\left\langle \ell, m, n \right\rangle)$ и $M = \ell m n$. Используя симметрию, получим, что $R_s(\left\langle M,M,M \right\rangle) \leq r^3$. Тогда для всех $N \geq 1$, при помощи дополнения до следующей наибольшей степени $M^i$ числа $M$,
  \[
  	R_s(\left\langle N,N,N \right\rangle) \leq R_s(\left\langle M^i,M^i,M^i \right\rangle) \leq r^{3i} = (M^i)^{3 \log_M r} = O(N^{3 \log_M r}).
  \]
  Поэтому 
  \begin{align*}
    \omega_s & \leq 3 \log_M r   \\
    M^{\omega_s/3} & \leq r \\
    (\ell m n)^{\omega_s/3} & \leq R_s(\left\langle \ell, m, n \right\rangle).\qedhere
  \end{align*}
\end{proof}

Отмечу, что можно определить граничный $s$-ранг $T$ как минимальный граничный ранг среди всех тензоров с носителем как у $T$, и затем, используя аргумент Бини \cite{Bini}, получить утверждение аналогичное утверждению \ref{prop:12:3.5}, но где вместо $s$-ранга будет стоять граничный $s$-ранг.

Тогда как верхняя оценка ранга тензора матричного умножения влечёт за собой билинейный алгоритм для матричного умножения, верхняя оценка $s$-ранга влечёт за собой билинейный алгоритм для взвешенной версии матричного умножения: если даны матрицы $A$ и $B$, алгоритм вычислит значения 
\[
	C_{i,k} = \sum_j \lambda_{i,j,k} A_{i,j} B_{j,k},
\]
где веса $\lambda_{i,j,k}$ это какие-то определённые ненулевые скаляры (зависящие от построения, с помощью которого стремимся достичь низкого $s$-ранга). Другими словами каждый элемент матрицы $C$ является взвешенным скалярным произведением, где разные веса применяются для разных скалярных произведений. Кажется, что не существует очевидных преобразований, которые убирали бы эти веса.

Как отмечалось выше $2 \leq \omega_s \leq \omega$, поэтому верхние оценки $\omega$ будут влечь за собой верхние оценки $\omega_s$ ($\omega=2$ влечёт $\omega_s=2$). Теорема \ref{th:12:3.6}, идущая ниже, покажет, что и верхние оценки $\omega_s$ влекут за собой верхние оценки $\omega$. В связи с этим $s$-ранг будет полезным ослаблением понятия ранга для оценки экспоненты матричного умножения.

\begin{theorem}
  \label{th:12:3.6} Для экспонент $\omega$ и $\omega_s$ будет верно $\omega \leq (3 \omega_s - 2)/2$.
\end{theorem}
Другими словами $\omega_s = 2 + \varepsilon$ влечёт $\omega \leq 2 + (3/2) \varepsilon$. В частности, если $\omega_s=2$, то $\omega=2$.
\begin{proof}
  По определению $\omega_s$ имеем, $R_s(\left\langle n,n,n \right\rangle)=n^{\omega_s + o(1)}$. Для данного значения $n$, пусть $T$ является трилинейной формой соответствующей этому (взвешенному) $n \times n$-матричному умножению:
  \[
  	T = \sum_{i,j,k \in [n]} \lambda_{i,j,k} \widehat{x}_{i,j} \widehat{y}_{j,k} \widehat{z}_{k,i},
  \]
  где $0 \neq \lambda_{i,j,k} \in \mathbb{C}$ для всех $i,j,k$. Пусть
  \[
  	S \subseteq \Delta_n = \left\{ (s_1, s_2, s_3) \mid s_1,s_2,s_3 \in [n] \text{ и } s_1+s_2+s_3=n+2 \right\}
  \]
  является \hyperref[def:triangle-free]{свободным от треугольников множеством}. Такое множество обладает свойством, что если $s,t,u \in S$ удовлетворяют $s_1=t_1, t_2=u_2, u_3=s_3$, то $s=t=u$. В статье 2005 года была дано простое построение свободного от треугольников множества $S$, где $|S|=n^{2-o(1)}$. Пусть $T'$ --- трилинейная форма, соответствующая $|S|$ независимым $n^2 \times n^2$-матричным умножениям, то есть
  \[
  	T'=\sum_{s \in S, i,j,k \in [n]^2} \widehat{u}_{s,i,j} \widehat{v}_{s,j,k} \widehat{w}_{s,k,i}.
  \]
  Покажем, что $T'$ является ограничением тензорной степени $T^{\otimes 3}$, которая задаётся как
  \[
  	T^{\otimes 3}=\sum_{i,j,k \in [n]^3} \lambda_{i_1,j_1,k_1} \lambda_{i_2,j_2,k_2} \lambda_{i_3,j_3,k_3} \widehat{x}_{i,j} \widehat{y}_{j,k} \widehat{z}_{k,i}.
  \]
  Другими словами, покажем, что $T'$ может быть получен заменой переменных в $T^{\otimes 3}$, что влечёт $R(T') \leq R(T^{\otimes 3})$. Чтобы это сделать определим (для $s,t,u \in S$ и $i,i',j,j',k,k' \in [n]^2$)
  \begin{align*}
    \widehat{u}_{s,i,j'} & =\lambda_{i_2,j_1',s_2} \widehat{x}_{(i_1,i_2,s_3),(s_1, j_1', j_2')}  \\
    \widehat{v}_{t,j,k'} & =\lambda_{t_3,j_2,k_2'} \widehat{y}_{(t_1,j_1,j_2),(k_1',t_2,k_2')} \\
    \widehat{w}_{u,k,i'} & =\lambda_{i_1',u_1,k_1} \widehat{z}_{(k_1,u_2,k_2),(i_1',i_2',u_3)}
  \end{align*}
  и положим переменные $\widehat{x}, \widehat{y}, \widehat{z}$, не встречающиеся в этих уравнениях, равными нулю. При такой замене переменных увидим, что $T^{\otimes 3}$ становится в точности тензором $T'$. Чтобы это проверить, нужно убедиться, что после подстановки $\widehat{u}, \widehat{v}, \widehat{w}$ вместо $\widehat{x}, \widehat{y}, \widehat{z}$ в $T^{\otimes 3}$ по выше упомянутым формулам, коэффициент при $\widehat{u}_{s,i,j'} \widehat{v}_{t,j,k'} \widehat{w}_{u,k,i'}$ равен 1, если $s=t=u, i=i', j=j'$ и $k=k'$, а иначе он равен 0. Так как носитель $T^{\otimes 3}$ будет тот же, что у $\left\langle n^3,n^3,n^3 \right\rangle$, одночлен 
  \[
  	\widehat{x}_{(i_1,i_2,s_3), (s_1,j_1',j_2')} \widehat{y}_{(t_1,j_1,j_2),(k_1',t_2,k_2')} \widehat{z}_{(k_1,u_2,k_2),(i_1',i_2',u_3)}
  \]
  имеет ненулевой коэффициент в $T^{\otimes 3}$ тогда и только тогда, когда 
  \begin{align*}
    (s_1,j_1',j_2') & = (t_1,j_1,j_2)\\
    (k_1',t_2,k_2') & = (k_1,u_2,k_2)\\
    (i_1',i_2',u_3) & = (i_1,i_2,s_3)
  \end{align*}
  Это произойдёт тогда и только тогда, когда $i=i', j=j',k=k',s_1=t_1,t_2=u_2$ и $u_3=s_3$. По определению свободного от треугольников множества последние три условия влекут $s=t=u$. Коэффициент в $T^{\otimes 3}$ при одночлене
  \[
  	\widehat{x}_{(i_1,i_2,s_3), (s_1,j_1,j_2)} \widehat{y}_{(s_1,j_1,j_2),(k_1,s_2,k_2)} \widehat{z}_{(k_1,s_2,k_2),(i_1,i_2,s_3)}
  \]
  равен $\lambda_{i_1,s_1,k_1} \lambda_{i_2,j_1,s_2} \lambda_{s_3,j_2,k_2}$, что в точности соответствует множителю $\lambda_{i_2,j_1',s_2} \lambda_{t_3,j_2,k_2'} \lambda_{i_1',u_1,k_1}$ из определения $\widehat{u}, \widehat{v}, \widehat{w}$, поэтому коэффициент $\widehat{u}_{s,i,j} \widehat{v}_{s,j,k} \widehat{w}_{s,k,i}$ после подстановки будет равен 1.
  
  Таким образом, обычный ранг прямой суммы $|S|=n^{2-o(1)}$ независимых $n^2 \times n^2$-матричных произведений не превосходит $\left(n^{\omega_s + o(1)}\right)^3$, применяя асимптотическое неравенство для сумм \eqref{eq:12:2.1}, получим
\begin{gather*}
	\rank{\ab{n^2, n^2, n^2} \oplus \dotsb \oplus \ab{n^2, n^2, n^2}} \leq \left(n^{\omega_s + o(1)}\right)^3\\
	n^{2-o(1)} \left( n^2 n^2 n^2 \right)^{\omega/3} \leq \left(n^{\omega_s + o(1)}\right)^3\\
	n^{2-o(1)} n^{2 \omega} \leq \left(n^{\omega_s + o(1)}\right)^3.
\end{gather*}
Взяв логарифм и устремив $n$ в бесконечность, получим $2+2\omega \leq 3 \omega_s$, что и требовалось. 
\end{proof}

Следующая теорема покажет как напрямую использовать взвешенное умножение для получения алгебраического алгоритма для умножения булевых матриц, где операция <<и>> действует как умножение, а <<или>> как сложение.

Даны булевы $n \times n$-матрицы $A$ и $B$, пусть $A'$ является очевидным отображением $A$ в комплексную матрицу из нулей и единиц, определим $B'$ аналогично, но со случайным выбором из 1 или 2 для каждого ненулевого элемента.

\begin{theorem}
  \label{th:12:3.7} Существует алгебраический алгоритм, требующий $n^{\omega_s + o(1)}$ операций для вычисления по $A',B'$ $n \times n$-матрицы $C$, обладающей свойством: $(i,j)$ элемент $C$ равен $0$, если $(i,j)$ элемент булевского матричного произведения $A$ и $B$ равен $0$; в противном случае он будет ненулевым с вероятностью не меньшей $1/2$.
  
  Процедуру можно повторить $O(\log n)$ раз, чтобы получить все элементы булевского произведения $A$ и $B$ с высокой вероятностью.
\end{theorem}
\begin{proof}
  Используя билинейный алгоритм, можно вычислить подходящее взвешенное произведение $A'$ и $B'$ за $n^{\omega_s+o(1)}$ операций. В результате получим матрицу, чьи элементы равны
  \[
  	\sum_\ell \lambda_{i,\ell,j} A_{i,\ell}' B_{\ell,j}',
  \]
  где $\lambda_{i,\ell,j} \neq 0$. Когда $(i,j)$ элемент булевского матричного произведения $A$ и $B$ равен 0, это значение очевидно также будет нулевым; иначе оно равно $\sum_{\ell \in L} \lambda_{i,\ell,j} r_{\ell}$ для непустого множества $L$, и каждый $r_{\ell}$ выбирается случайно из $\left\{ 1,2 \right\}$. Для данного $\ell \in L$ существует единственное значение $r_{\ell}$, делающее эту сумму нулевой, поэтому вероятность того, что эта сумма обнуляется не превосходит $1/2$.
\end{proof}

Если все веса, появляющиеся в приведённом выше доказательстве, положительны, тогда никакой случайности не нужно, так как $A'$ и $B'$ могут быть взяты как очевидные отображения $A$ и $B$ в матрицы из нулей и единиц.

Наконец, отметим, что все манипуляции, используемые в литературе по матричному умножению, для преобразования оценок ранга для определённых <<базовых>> тензоров в оценки ранга тензора матричного умножения будут также работать, если заменить ранг на $s$-ранг. Так, например, оценки $s$-ранга тензора частичного матричного умножения Бини, Каповани, Лотти и Романи \cite{Bini}, базового тензора, использовавшегося Шёнхаге \cite{Schonhage81}, базового тензора из статьи Штрассена о лазерном методе \cite{Strassen1987} или любого из базовых тензоров, введённых Копперсмитом и Виноградом \cite{Coppersmith:1990} в конце концов приводят к оценкам $s$-ранга тензора матричного умножения, если просто следовать известным доказательствам. Однако для большинства этих базовых тензоров есть явные тензорные разложения, и поэтому методом подстановок можно получить соответствующие нижние оценки для ранга (например, доказательство утверждения \ref{prop:12:3.3}). Эти нижние оценки также будут нижними оценками для $s$-ранга, поэтому кажется, что нет возможности для улучшения путём простого переключения на $s$-ранг. Однако улучшение при помощи переключения на граничный $s$-ранг может быть возможно. Для конкретного примера, неизвестно будет ли граничный $s$-ранг или даже просто $s$-ранг тензора $\left\langle 2,2,2 \right\rangle$ равным 6 или 7.























