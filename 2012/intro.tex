\section{Введение}

За 43 года, прошедших с появления статьи Штрассена \cite{Strassen:1969}, которая впервые улучшила очевидную оценку $\omega \leq 3$, было сделано несколько масштабных концептуальных улучшений для того, чтобы получить верхние оценки $\omega$, каждое из которых неформально можно считать ослаблением <<правил игры>>. Например, Бини \cite{Bini} показал, что верхняя оценка граничного ранга тензора будет означать верхнюю оценку его асимптотического ранга. В самом деле существуют полезные примеры тензоров с граничным рангом строго меньше чем их ранг, что приводит к улучшению первоначального алгоритма Штрассена. Шёнхаге \cite{Schonhage81} показал как перевести верхние оценки ранга прямой суммы нескольких тензоров матричного умножения в верхние оценки $\omega$. Его асимптотическое неравенство для сумм играло значительную роль практически во всех дальнейших улучшениях. Лазерный метод, разработанный Штрассеном \cite{Strassen1987}, дал способ перевода тензоров нематричного умножения (чья грубая структура содержит большую диагональ, и чьи компоненты изоморфны тензорам матричного умножения) в верхние оценки $\omega$. Этот метод был использован Копперсмитом и Виноградом \cite{Coppersmith:1990}, также как и в самых последних усовершенствованиях Стотерса \cite{stothers2010} и Вильямс \cite{Williams:2011}.

В этой статье авторы ввели дальнейшее ослабление правил игры, заключающееся в изучении взвешенной версии матричного умножения. Вместо вычисления произведения $AB$ двух матриц через
\[
	(AB)_{i,k} = \sum_j A_{i,j} B_{j,k},
\]
они используют 
\[
	\sum_j \lambda_{i,j,k} A_{i,j} B_{j,k},
\]
где коэффициенты $\lambda_{i,j,k}$ являются ненулевыми комплексными числами. Конечно, в некоторых случаях взвешенное матричное умножение будет тривиальным эквивалентом обычного матричного умножения. Например, если $\lambda_{i,j,k}$ можно записать как $\alpha_{i,j} \beta_{j,k} \gamma_{k,i}$, то взвешенное матричное умножение эквивалентно обычному умножению матриц, чьи элементы были умножены на скаляры. Однако умножение матриц на скаляры не является равнозначной заменой для взвешенного умножения со случайными весами.
\begin{question}
  Как перевести: For example, if $\lambda_{i,j,k}$ can be written as $\alpha_{i,j} \beta_{j,k} \gamma_{k,i}$, then weighted matrix multiplication amounts to ordinary multiplication of matrices whose
entries have been rescaled. However, rescaling does not yield an efficient equivalence for arbitrary weights. 
\end{question}

Сложность взвешенного матричного произведения оценивается при помощи новой экспоненты $\omega_s$, удовлетворяющей $2 \leq \omega_s \leq \omega$. Это наименьшее действительное число, для которого существуют веса (зависящие от размеров матриц) такие, что взвешенное произведение $n \times n$-матриц может быть произведено за $n^{\omega_s + o(1)}$ арифметических операций. <<$s$>> является сокращением для support (носитель), потому что мы имеем дело с тензорами, которые имеют тот же носитель, что и тензоры матричного умножения.

В статье 2003 года \cite{Cohn03} было показано как вложить матричное умножение в групповую алгебру, и этот способ был использован в статье 2005 года \cite{Cohn05} для доказательства нетривиальных оценок $\omega$. Замена групповых алгебр более общими алгебрами всегда была привлекательным обобщением, и в самом деле этот метод работает, за исключением того, что он приводит к вложению взвешенного матричного умножения. Таким образом, он даёт верхние оценки $\omega_s$, а не $\omega$. До этой статьи верхние оценки $\omega_s$ вызывали интерес только как аналоги оценок $\omega$, и не было известно, что они будут означать для $\omega$ самой по себе. В этой статье авторам удалось преодолеть все препятствия, и был разработан способ, которым можно связать оценки для этих различных экспонент.


