\section{Предварительные данные и предпосылки}

\subsection{Тензоры}

Все результаты будут выражены в терминах тензоров. Напомню, что тензоры являются обобщениями векторов и матриц на большие порядки. Тензорные произведения векторных пространств создают элегантное алгебраическое окружение для теории тензоров, но здесь авторы использовали более конкретный подход представления тензоров как мультилинейных форм. Например, матрица с элементами $A_{i,j}$ соответствует билинейной форме $\sum_{i,j} A_{i,j} \widehat{x}_i \widehat{y}_j$, где $\widehat{x}_i$ и $\widehat{y}_j$ --- формальные переменные. Можно представить трёхмерный тензор как $\sum_{i,j,k} A_{i,j,k} \widehat{x}_i \widehat{y}_j \widehat{z}_k$. Будем использовать шапку, чтобы было ясно какие символы обозначают формальные переменные. Применение обратимых линейных преобразований ко множествам переменных (здесь $\left\{ \widehat{x}_i \right\}, \left\{ \widehat{y}_j \right\}$ и $\left\{ \widehat{z}_k \right\}$) будет давать изоморфный тензор, но нельзя смешивать переменные из разных множеств. 

Ниже даны определения, которые уже встречались нам ранее, но теперь мы будем смотреть на тензоры как на мультилинейные формы.
\begin{definition}
  \textbf{Прямой суммой} $T \oplus T'$ двух тензоров будет просто их сумма, если у них нет общих переменных (иначе сначала нужно переименовать переменные, чтобы убрать все пересечения). 
\end{definition}

\begin{definition}
  \textbf{Тензорное произведение} $T \otimes T'$ тензоров $T = \sum_{i,j,k} T_{i,j,k} \widehat{x}_i \widehat{y}_j \widehat{z}_k$ и $T' = \sum_{\ell,m,n} T_{\ell,m,n}' \widehat{u}_\ell \widehat{v}_m \widehat{w}_n$ --- это
  \[
  	T \otimes T' = \sum_{i,j,k,\ell,m,n} T_{i,j,k} T_{\ell,m,n}' \widehat{r}_{i,\ell} \widehat{s}_{j,m} \widehat{t}_{k,n},
  \]
  с новыми переменными $\widehat{r}_{i,\ell}, \widehat{s}_{j,m}$ и $\widehat{t}_{k,n}$. Другими словами мы просто взяли произведение $T$ и $T'$, но смешали переменные как показано выше (например, $\widehat{x}_i \widehat{u}_\ell$ стало $\widehat{r}_{i,\ell}$). 
\end{definition}

Прямая сумма и тензорное произведение определены, только если тензоры имеют одинаковую размерность, и эти операции сохраняют размерность.

Ранг $R(T)$ тензора $T$ является одной из самых важных инвариант. Ненулевой тензор имеет ранг 1, если он является произведением линейных форм, и ранг $r$, если он является суммой $r$ тензоров с рангом 1, но не меньше. Другими словами $T = \sum_{i,j,k} T_{i,j,k} \widehat{x}_i \widehat{y}_j \widehat{z}_k$ имеет ранг $r$, если существуют линейные формы 
\begin{align*}
     \alpha_\ell(\widehat{x}) & = \alpha_{\ell 1} \widehat{x}_1 + \dotsb + \alpha_{\ell u} \widehat{x}_u,\\
     \beta_\ell(\widehat{y}) & = \beta_{\ell 1} \widehat{y}_1 + \dotsb + \beta_{\ell v} \widehat{y}_v,\\
     \gamma_\ell(\widehat{z}) & = \gamma_{\ell 1} \widehat{z}_1 + \dotsb + \gamma_{\ell w} \widehat{z}_w
\end{align*}
такие, что
\[
	\sum_{i,j,k} T_{i,j,k} \widehat{x}_i \widehat{y}_j \widehat{z}_k = \sum_{\ell=1}^r \alpha_\ell(\widehat{x}) \beta_\ell(\widehat{y}) \gamma_\ell(\widehat{z}).
\]
Тензорный ранг обобщает концепцию матричного ранга, но он является более тонким понятием. В то время как матрицы могут быть приведены к простому каноническому виду (ступенчатый вид по строкам), в котором их ранг становится виден, тензоры к такому виду привести нельзя, потому что симметрическая группа, действующая на них, имеет гораздо меньшую размерность по сравнению с размерностью пространства самих тензоров. 
\begin{question}
  что это могло бы значить? Tensor rank generalizes the concept of matrix rank, but it is more subtle. While matrices can be brought
into a simple canonical form (row echelon form) in which their rank is visible, tensors cannot, because the
symmetry group acting on them has far too low a dimension compared with the dimension of the space of
tensors itself.
\end{question}
В самом деле, вычисление ранга тензора будет $NP$-трудной задачей \cite{Hastad90}.

\subsection{Матричное умножение в терминах тензоров}

Тензор матричного умножения $\left\langle \ell,m,n \right\rangle$ --- это тензор 
\[
	\sum_{i=1}^{\ell} \sum_{j=1}^{m} \sum_{k=1}^{n} \widehat{x}_{i,j} \widehat{y}_{j,k} \widehat{z}_{k,i}.
\]
Обратите внимание, что коэффициент при $\widehat{z}_{k,i}$ выделяет элементы $\widehat{x}_{i,j} \widehat{y}_{j,k}$, которые находятся в $(i,k)$-ом элементе матричного произведения. Раньше уже упоминалось, что 
\[
	\left\langle \ell,m,n \right\rangle \otimes \left\langle \ell',m',n' \right\rangle \cong \left\langle \ell \ell', m m', n n' \right\rangle.
\]

Представив $\left\langle \ell,m,n \right\rangle$ в виде выражения с низким рангом, можно получить эффективный алгоритм для вычисления произведения $\ell \times m$ и $m \times n$-матриц. В частности, отсюда следует, что $(\ell m n)^{\omega/3} \leq R(\left\langle \ell, m, n \right\rangle)$ (теорема \ref{th:4.7}).

Фактически, хотя мы определили $\omega$ в терминах произвольного алгебраического алгоритма, она полностью характеризуется при помощи ранга
\[
	\omega = \inf \left\{ \tau \in \mathbb{R} \mid R(\left\langle n,n,n \right\rangle) = O(n^\tau) \right\}.
\]
(утверждение \ref{prop:bur:15.1}) Другими словами, билинейные алгоритмы имеют ту же экспоненту, что и произвольные алгебраические алгоритмы. Таким образом, весь предмет быстрого матричного умножения может быть сведён к оценке рангов тензоров матричного умножения.

Асимптотическое неравенство для сумм Шёнхаге \cite{Schonhage81} устанавливает, что
\begin{equation}\label{eq:12:2.1}
  (\ell_1 m_1 n_1)^{\omega/3} + \dotsb + (\ell_k m_k n_k)^{\omega/3} \leq R(\left\langle \ell_1, m_1, n_1 \right\rangle \oplus \dotsb \oplus \left\langle \ell_k, m_k, n_k \right\rangle),
\end{equation}
и более того, что тоже самое будет верно для граничного ранга. Таким образом, неожиданно эффективный способ выполнения нескольких независимых матричных умножений будет давать оценку на $\omega$.

Смотри \cite{bur} для более глубокого понимания тензоров матричного умножения и алгебраической сложности в целом. Важно помнить, что у всех тензорых операций есть в основе неявные алгоритмы. В принципе можно обойтись совсем без тензорного формализма, но он играет важную роль в привлечении внимания к центральным вопросам.
