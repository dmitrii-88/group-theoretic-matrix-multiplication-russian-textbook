\section{Семейства когерентных конфигураций}

В этом разделе мы обсудим пригодность широких классов когерентных конфигураций для получения оценок $\omega$.

\subsection{Когерентные конфигурации с несколькими слоями}

Из свойства (\ref{item:1:def:12:cc}) в определении когерентной конфигурации следует, что существует подмножество классов, образующих диагональ. Будем называть эти классы \textbf{слоями} (англ. fibers) когерентной конфигурации. Как отмечалось в подразделе \ref{ssub:12:4.2}, конфигурации, у которых больше одного слоя, являются некоммутативными. Для приложений интереснее то, что, как мы скоро увидим, $n$ слоёв достаточно для вложения умножения $n \times n$-матриц. Это наблюдение обобщает пример \ref{ex:12:4.6}.

Слои когерентной конфигурации $\mathscr{C}$ соответствуют разбиению множества точек на подмножества $\mathscr{C}_1, \dotsc, \mathscr{C}_n$. Тогда из свойства (\ref{item:3:def:12:cc}) определения следует, что классы $\mathscr{C}$ являются уточнением подмножеств $\mathscr{C}_i \times \mathscr{C}_j$, то есть классы $\mathscr{C}$ получаются дальнейшим дроблением подмножеств $\mathscr{C}_i \times \mathscr{C}_j$.

\begin{prop}\label{prop:12:6.1}
     Любая когерентная конфигурация с $n$ слоями реализует $\left\langle n,n,n \right\rangle$.
\end{prop}
\begin{proof}
  Пусть $\mathscr{C}$ --- когерентная конфигурация с $n$ слоями, которым соответствует разбиение $\mathscr{C}_1, \dotsc, \mathscr{C}_n$, и пусть $x_1, \dotsc, x_n$ --- система различных представителей $\mathscr{C}_1, \dotsc, \mathscr{C}_n$. Определим $\alpha(a,b)$ как класс $\mathscr{C}$, содержащий $(x_a, x_b)$, $\beta(b,c)$ --- класс, содержащий $(x_b,x_c)$, а $\gamma(c,a)$ --- класс, содержащий $(x_c,x_a)$. Легко проверить, что эти функции удовлетворяют определению \ref{def:12:4.1}. 
\end{proof}

Однако это универсальное вложение не даёт нетривиальных оценок $\omega_s$, потому что аналогичными рассуждениями можно показать, что степень одного из характеров должна быть не меньше $n$. Пусть $\mathscr{C}'$ --- когерентная конфигурация с точками как у $\mathscr{C}$ и классами $\mathscr{C}_i \times \mathscr{C}_j$. Тогда алгебра смежности $\mathscr{C}'$ является подалгеброй алгебры смежности $\mathscr{C}$ (матрицы смежности для $\mathscr{C}_i \times \mathscr{C}_j$ будут суммами матриц смежности для классов $\mathscr{C}$, содержащихся в $\mathscr{C}_i \times \mathscr{C}_j$), и нетрудно проверить, что алгебра смежности $\mathscr{C}'$ изоморфна $\mathbb{C}^{n \times n}$. Однако $\mathbb{C}^{n \times n}$ не может быть изоморфна подалгебре полупростой алгебры $\mathbb{C}^{d_1 \times d_1} \times \dotsm \times \mathbb{C}^{d_t \times d_t}$, если нет $d_i \geq n$ для какого-то $i$. Чтобы увидеть почему, обратите внимание, что проекция на факторы полупростой алгебры даст представления размеров $d_i$ для $\mathbb{C}^{n \times n}$. Потому как $\mathbb{C}^{n \times n}$ --- простая алгебра, эти проекции будут нулевыми кроме случая, когда $d_i \geq n$. 

\subsection{Шуровы когерентные конфигурации}

Напомню, что шуровы когерентные конфигурации получаются из действий групп на множества. Такая конфигурация будет ассоциативной схемой тогда и только тогда, когда действие транзитивно.

В случае, когда группа действует на себя правыми умножениями, эта конструкция эквивалентна свойству тройного произведения. Поэтому гипотезы \ref{conj:05:3.4} и \ref{conj:05:4.7}, каждая из которых означает $\omega=2$ при помощи свойства тройного произведения в группах, означают также, что шуровых когерентных конфигураций достаточно для достижения $\omega_s=2$.

Более интересно то, что когерентные конфигурации, появляющиеся в теореме \ref{th:12:5.6} и гипотезе \ref{conj:12:5.7}, на самом деле будут коммутативными шуровыми когерентными конфигурациями. Всё потому, что симметрические степени когерентных конфигураций, возникающие из абелевых групп, которые коммутативны, также будут и шуровыми через действие сплетением: если $G$ --- группа, и $\mathscr{C}$ --- связанная с ней когерентная конфигурация, то $Sym^k \mathscr{C}$ --- шурова когерентная конфигурация, возникающая из действия $S_k \ltimes G^k$ на $G^k$.

Таким образом, коммутативные шуровы когерентные конфигурации, возникающие из транзитивных групповых действий, уже доказывают нетривиальные оценки $\omega_s$ (и $\omega$), и если хотя бы одна из двух гипотез статьи 2005 года верна, то их будет достаточно, чтобы доказать $\omega_s = 2$.

\subsection{Групповые ассоциативные схемы}

Другим универсальным способом получения шуровых когерентных конфигураций будет рассмотрение действия $G \times G$ на $G$, следующим образом $(x, y) \cdot g = x g y^{-1}$. Оно порождает коммутативную когерентную конфигурацию (независимо от того, была $G$ коммутативной или нет, что будет привлекательным для приложений), называемую \textbf{групповой ассоциативной схемой} (англ. group association scheme), чьи классы отождествляются с классами сопряжённых элементов $G$ (то есть класс $R_i = \left\{ (g,h) \mid g h^{-1} \in C_i \right\}$, где $C_i$ --- это $i$-ый класс сопряжённых элементов). Сейчас покажем, что групповых ассоциативных схем достаточно для доказательства нетривиальных оценок $\omega_s$, и, что при верности одной из гипотез статьи 2005 года, их достаточно, чтобы доказать $\omega_s = 2$.

Пусть подмножества $A_i, B_i, C_i$ группы $H$ удовлетворяют \hyperref[def:05:5.1]{свойству совместно тройного произведения}, тогда когерентная конфигурация, связанная с правым действием $H$ на себя, реализует $\bigoplus_i \left\langle |A_i|, |B_i|, |C_i| \right\rangle$ при помощи функций $\alpha_i$, $\beta_i$ и $\gamma_i$, заданных на $A_i \times B_i$, $B_i \times C_i$ и $C_i \times A_i$ соответственно. В частности $\alpha_i(a,b)$ --- это класс, содержащий пару $(a,b)$ и так далее. Тогда определение \ref{def:12:5.1} эквивалентно свойству совместного тройного произведения.

В статье 2005 года в числе других были описаны следующие конструкции:
\begin{enumerate}
     \item Из теоремы \ref{th:05:3.3} и подраздела \ref{ssub:05:6.3} следует, что для всех $m > 2$ и достаточно больших $\ell$ существует $n$ троек $A_i,B_i,C_i$ подмножеств $(\mathbb{Z}/ m \mathbb{Z})^{3 \ell}$, удовлетворяющих свойству совместного тройного произведения, где $n = (27/4)^{\ell - o(1)}$ и $|A_i||B_i||C_i| = (m-2)^{3 \ell}$ для всех $i$. Так как шурова когерентная конфигурация, полученная с помощью правого действия группы $(\mathbb{Z}/ m \mathbb{Z})^{3 \ell}$ на саму себя, будет иметь ранг равный порядку этой группы, то применив теорему \ref{th:12:5.5} и взяв предел при $\ell \to \infty$, получим
     \begin{gather*}
     	\sum_{i=1}^n \left( |A_i||B_i||C_i| \right)^{\omega_s/3} \leq |(\mathbb{Z}/ m \mathbb{Z})^{3 \ell}|\\
     	(27/4)^{\ell - o(1)} (m-2)^{\ell \omega_s} \leq m^{3 \ell}\\
     	\frac{\ell - o(1)}{\ell} \ln(27/4) + \omega_s \ln(m-2) \leq 3 \ln m\\
     	\omega_s \leq \frac{3 \ln m - \ln(27/4)}{\ln(m-2)},
     \end{gather*}
     которое становится оптимальным при $m = 10$ (давая $\omega_s \leq 2.41$).
     \item \label{it:2:gas} Любая из гипотез статьи 2005 года означает существование подмножеств, удовлетворяющих свойству совместного тройного произведения в абелевой группе $H$, где
     \[
     	|A_i| = |B_i| = |C_i| = t \geq n^\varepsilon
     \]
     для $1 \leq i \leq n$ и $|H| = (t^2 n)^{1 + o(1)}$ (при $n \to \infty$ и фиксированном $\varepsilon > 0$), что доказало бы $\omega = 2$. 
\end{enumerate}

\begin{theorem}\label{th:12:6.3} 
  Пусть $H$ --- абелева группа, и пусть $n$ троек подмножеств $A_i, B_i, C_i$ в $H$ удовлетворяют свойству совместного тройного произведения. Пусть $G = S_n \ltimes H^n$ и определим
  \begin{align*}
    A & = A_1 \times A_2 \times \dotsm \times A_n \\
    B & = B_1 \times B_2 \times \dotsm \times B_n \\
    C & = C_1 \times C_2 \times \dotsm \times C_n   
  \end{align*}
  на которые будем смотреть как на подмножества $G$ при помощи естественного вложения $H^n$ в $G$. Пусть $\mathscr{C}$ --- групповая ассоциативная схема для $G$. Тогда подмножества $A,B,C$ удовлетворяют условиям из утверждения \ref{prop:12:4.7} в отношении $\mathscr{C}$ (то есть для действия $G \times G$ на $G$), поэтому $\mathscr{C}$ реализует $\left\langle |A|, |B|, |C| \right\rangle$.
\end{theorem}
\begin{proof}
  Будем записывать элементы $G$ как $h \pi$, где $h \in H^n, \; \pi \in S_n$, и будем через $\pi \cdot h$ обозначать действие перестановок из $S_n$ на $H^n$. Полупрямое произведение удовлетворяет равенству $\pi h = (\pi \cdot h) \pi$ для $\pi \in S_n$ и $h \in H^n$.
  
  Предположим, что есть $f = (f_1, f_2), g = (g_1, g_2), h = (h_1, h_2)$ в $G \times G$ и $a, a' \in A, \; b,b' \in B, \; c,c' \in C$, для которых
  \begin{equation}\label{eq:12:6.1}
	  \begin{aligned}
	  fgh & = 1\\
	  f_1 a f_2^{-1} & = a'\\
	  g_1 b g_2^{-1} & = b'\\
	  h_1 c h_2^{-1} & = c'.
	  \end{aligned}
  \end{equation}
  Мы хотим показать, что $a = a', b = b', c = c'$. Из последних трёх уравнений \eqref{eq:12:6.1} видно, что $f_1 = x_1 \pi, f_2 = x_2 \pi$ для каких-то $x_1, x_2 \in H^n$ и $\pi \in S_n$. 
  \begin{align*}
    f_1 a f_2^{-1} & = x_1 \pi a \pi^{-1} x_2^{-1} \\
    & = x_1 (\pi \cdot a) \pi \pi^{-1} x_2^{-1} \\
    & = x_1 (\pi \cdot a) x_2^{-1} \in H^n
  \end{align*}
  Подобным же образом получим, что $g_1 = y_1 \rho, g_2 = y_2 \rho$ и $h_1 = z_1 \tau, h_2 = z_2 \tau$. Теперь, используя коммутативность $H$, три уравнения примут вид
  \begin{equation}\label{eq:12:6.2}
	  \begin{aligned}
		x_1 x_2^{-1} & = a' (\pi \cdot a)^{-1}	 \\
		& = a' (\pi \cdot a^{-1})\\
		y_1 y_2^{-1} & = b' (\rho \cdot b^{-1}) \\
		z_1 z_2^{-1} & = c' (\tau \cdot c^{-1}).
	  \end{aligned}
  \end{equation}
  Из $fgh=1$ получим $f_1 g_1 h_1 = 1$, что означает
  \begin{align*}
       1 = f_1 g_1 h_1 & = x_1 \pi y_1 \rho z_1 \tau\\
       & = x_1 (\pi \cdot y_1) \pi \rho z_1 \tau\\
       & = x_1 (\pi \cdot y_1) ((\pi \rho) \cdot z_1) \pi \rho \tau\\
       & = x_1 (\pi \cdot y_1)((\pi \rho) \cdot z_1).
  \end{align*}
  Точно также из $fgh=1$ получим $f_2 g_2 h_2 = 1$ и следовательно
  \[
  	1 = f_2 g_2 h_2 = x_2 (\pi  \cdot y_2)((\pi \rho) \cdot z_2).
  \]
  Используя коммутативность $H$, из этих двух уравнений получаем
  \[
  	x_1 x_2^{-1} (\pi \cdot (y_1 y_2^{-1}))((\pi \rho) \cdot (z_1 z_2^{-1})) = 1,
  \]
  что в сочетании с \eqref{eq:12:6.2} даёт
  \[
  	a' (\pi \cdot a^{-1}) (\pi \cdot b')((\pi \rho) \cdot b^{-1})((\pi \rho) \cdot c')((\pi \rho \tau) \cdot c^{-1}) = 1.
  \]
  $f g h = 1$ означает $\pi \rho \tau = 1$, поэтому получаем
  \[
  	(\pi \cdot (a^{-1} b'))((\pi \rho) \cdot (b^{-1} c'))(c^{-1} a') = 1.
  \]
  С помощью свойства совместного тройного произведения получим $\pi = \pi \rho = 1$. Откуда можно заключить, что $\pi = \rho = \tau = 1$, и затем, что $a=a', b=b', c=c'$, что и требовалось.
\end{proof}

Чтобы определить какие оценки $\omega_s$ следует ожидать, нам нужно знать ранг групповой ассоциативной схемы, то есть число классов сопряжённых элементов в $S_n \ltimes H^n$:
\begin{lemma}\label{lem:12:6.4}
  Существует константа $C$ такая, что для любой абелевой группы $H$, если $n \leq |H|$, то число классов сопряжённых элементов $S_n \ltimes H^n$ не превосходит $C^n |H|^n/ n^n$.
\end{lemma}
Это грубая оценка, но её будет достаточно для наших целей.
\begin{proof}
  Не сложно доказать следующее описание классов сопряжённых элементов в группе $S_n \ltimes H^n$. Любой элемент этой группы может быть записан как $h \pi$, где $h \in H^n$ и $\pi \in S_n$. (Напомню, что любую перестановку $\pi$ можно представить в виде произведения непересекающихся циклов $\pi = (i_1 \; i_2 \; \dotso i_k) \dotso (i_t \; \dotso \; i_n)$. Такое представление будет называться \textbf{циклическим разложением}. Информация о количестве циклов каждой из длин в циклическом разложении называется \textbf{циклическим типом перестановки}.) Циклический тип $\pi$ не изменяется при сопряжениях, и сумма элементов из $H$ в координатах элемента $h$, соответствующих любому из циклов $\pi$, также сохраняется. Более того эти инварианты полностью определяют класс сопряжённых элементов. Таким образом, любой класс сопряжённых элементов определяется мультимножеством пар, состоящих из длины цикла и элемента группы $H$, где длины циклов должны в сумме давать $n$.
  
  Возможные циклические типы соответствуют разбиениям $n$. Их число растёт субэкспоненциально при $n \to \infty$. Проще говоря, существует $2^{n-1}$ композиций числа $n$ (то есть способов записи $n$ в виде \textit{упорядоченной} суммы положительных целых чисел) и поэтому существует не более чем $2^{n-1}$ разбиений $n$. Напомню, что \textbf{разбиением числа} называют его представление в виде суммы положительных целых чисел, называемых частями. При этом порядок следования частей не учитывается (в отличие от композиций), то есть разбиения, отличающиеся только порядком частей, считаются равными.
  
  Пусть перестановка имеет $c_i$ циклов длины $i$, где $\sum_{i=1}^n i c_i = n$. Тогда существует
  \[
  	\prod_{i=1}^n \dbinom{|H| + c_i - 1}{c_i}
  \]
  способов выбрать элементы из $H$, соответствующие этим циклам. Здесь используется формула для числа сочетаний с повторением $\multiset{|H|}{c_i} = \binom{c_i + |H| - 1}{|H| - 1} = \binom{|H| + c_i - 1}{c_i}$. Таким образом, оценка числа классов сопряжённых элементов в $G$ эквивалентна оценке того на сколько велико это произведение.
  
  Так как следующие оценки верны для $\binom{n}{k}$:
  \[
  	\left( \frac{n}{k} \right)^k \leq \binom{n}{k} \leq \frac{n^k}{k!} \leq \left( \frac{n \cdot e}{k} \right)^k \text{ для } 1 \leq k \leq n,
  \]
  имеем
  \begin{align*}
      \prod_{i=1}^n \dbinom{|H| + c_i - 1}{c_i} & \leq \prod_{i=1}^n \left( \frac{e(|H| + c_i - 1)}{c_i} \right)^{c_i} \\
      & \leq (2e)^n  \prod_{i=1}^n \frac{|H|^{c_i}}{c_i^{c_i}} \text{ (так как $c_i - 1 \leq n \leq |H|$)} \\
      & \leq (2e)^n \prod_{i=1}^n \frac{|H|^{i c_i}}{c_i^{c_i} n^{(i-1) c_i}} \text{ (так как $\frac{|H|}{n} \geq 1$)} \\
      & \leq (2e)^n \frac{|H|^n}{n^n} \prod_{i=1}^n \left( \frac{n}{c_i} \right)^{c_i}.
  \end{align*}
  Если положим $x_i = c_i / n$, тогда
  \[
  	\prod_{i=1}^n \left( \frac{n}{c_i} \right)^{c_i} = e^{-n \sum_{i=1}^n x_i \ln x_i},
  \]
  Таким образом, для завершения доказательства нужно показать, что $-\sum_{i=1}^n x_i \ln x_i$ ограничен независимо от $n$ всякий раз, когда $x_i \geq 0$ и $\sum_{i=1}^n i x_i = 1$. Максимум может быть найден при помощи метода множителей Лагранжа. Приходится иметь дело с крайними случаями, когда $x_i = 0$ для какого-то $i$, ниже приведены подробности.
  
  Пусть $x_1, \dotsc, x_n$ максимизируют $-\sum_{i=1}^n x_i \ln x_i$ при условии, что $\sum_{i=1}^n i x_i = 1$ и $x_i \geq 0$. Желаемый результат будет тривиальным, когда только один из $x_1, \dotsc, x_n$ является ненулевым. В противном случае, пусть $z_1, \dotsc, z_m$ --- ненулевые элементы среди $x_1, \dotsc, x_n$. Уравнение $\sum_{i=1}^n i x_i = 1$ преобразуется в $\sum_{i=1}^m y_i z_i = 1$, где $y_1 < y_2 < \dotsb < y_m$ --- положительные целые. Для нахожденя условного максимума будем использовать метод множителей Лагранжа. В данном случае функция Лагранжа имеет вид:
  \[
  	L(z_1, \dotsc, z_m, \lambda) = -\sum_{i=1}^m z_i \ln z_i - \lambda \left( \sum_{i=1}^m y_i z_i - 1 \right),
  \]  
  где $\lambda$ --- это множитель Лагранжа. Необходимым условием точки экстремума будет равенство нулю в ней всех частных производных функции $L(z_1, \dotsc, z_m, \lambda)$. Для $z_i$ это условие примет вид
  \[
       \frac{\partial L(z_1, \dotsc, z_m, \lambda)}{\partial z_i} = \frac{\partial (-z_i \ln z_i - \lambda y_i z_i)}{\partial z_i} = -\ln z_i - z_i \frac{1}{z_i} - \lambda y_i = 0
  \]
  Поэтому существует множитель Лагранжа $\lambda$ такой, что $-1-\ln z_i = \lambda y_i$ для всех $i$ и отсюда
  \[
  	- \sum_{i=1}^m z_i \ln z_i = \sum_{i=1}^m z_i (\lambda y_i + 1) = \lambda + \sum_{i=1}^m z_i \leq \lambda + 1.
  \]
  Чтобы оценить $\lambda$, заметим, что $z_i = e^{-1-\lambda y_i}$ и поэтому
  \[
  	\sum_{i=1}^m y_i e^{-1-\lambda y_i} = 1,
  \]
  в то время как для $\lambda > 1$ имеем
  \[
  	\sum_{i=1}^m y_i e^{-1-\lambda y_i} < \sum_{j=1}^\infty j e^{-1-j} < 1.
  \]
  Таким образом, $\lambda < 1$ и поэтому $-\sum_{i=1}^m z_i \ln z_i \leq 2$. Сочетая полученные оценки, покажем, что число классов сопряжённых элементов $S_n \ltimes H^n$ не превосходит
  \begin{align*}
    2^{n-1} (2e)^n \frac{|H|^n}{n^n} e^{-n \sum_{i=1}^n x_i \ln x_i} & \leq 2^{n-1} (2e)^n \frac{|H|^n}{n^n} e^{2n} \\
    & = 2^{n-1} (2 e^3)^n \frac{|H|^n}{n^n} \\
    & < (4 e^3)^n \frac{|H|^n}{n^n}.
  \end{align*}
  Поэтому мы можем взять $C = 4 e^3$ в утверждении леммы. (Самая лучшая из возможных констант будет, конечно, гораздо меньше.)
\end{proof}

Эта оценка ранга групповых ассоциативных схем именно то, что нужно для повторного получения желанных оценок из приведённых ранее построений, удовлетворяющих свойству совместного тройного произведения. Например, применив теорему \ref{th:12:6.3} и лемму \ref{lem:12:6.4} ко \hyperref[it:2:gas]{второму примеру}, получим
\[
	t^{n \omega_s} \leq C^n \frac{(t^2 n)^{n(1+o(1))}}{n^n},
\]
что эквивалентно
\[
	\omega_s \ln t \leq \ln C + (2 + o(1)) \ln t + o(\ln n).
\]
Так как $t \geq n^\varepsilon$, то мы получим $\omega_s=2$ в пределе при $n \to \infty$.

\begin{corollary}
  Существуют групповые ассоциативных схемы, которые доказывают $\omega_s \leq 2.41$ (и следовательно $\omega \leq 2.62$). Если любая из гипотез \ref{conj:05:3.4} и \ref{conj:05:4.7} верна, то существуют групповые ассоциативные схемы, которые доказывают $\omega_s = 2$ (и следовательно $\omega = 2$).
\end{corollary}

В более общем смысле, можно скопировать переход от следствия \ref{cor:12:5.4} (которое аналогично теореме \ref{th:12:6.3}) к теореме \ref{th:12:5.5}, чтобы доказать теорему \ref{th:05:5.5}, используя групповые ассоциативные схемы.























