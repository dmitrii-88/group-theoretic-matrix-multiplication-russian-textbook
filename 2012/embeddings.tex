\section{Совместные вложения и симметрические степени}

Лучшие техники построения до настоящего времени были основаны на одновременной реализации нескольких независимых матричных умножений. В работе 2005 года это было названо свойством совместного тройного произведения. Для когерентных конфигураций это определение эквивалентно следующему (и, конечно, можно дать аналогичное определение для любой алгебры):
\begin{definition}\label{def:12:5.1}
     Когерентная конфигурация $\mathscr{C}$ ранга $r$ \textbf{реализует} $\bigoplus_i \left\langle \ell_i, m_i, n_i  \right\rangle$, если существуют инъективные функции
     \begin{align*}
     	   \alpha_i &:[\ell_i] \times [m_i] \to [r]\\
     	   \beta_i &:[m_i] \times [n_i] \to [r]\\
     	   \gamma_i &:[n_i] \times [\ell_i] \to [r]
     \end{align*}
     такие, что $\alpha_i(a,b'), \beta_j(b,c'), \gamma_k(c,a')$ образуют треугольник тогда и только тогда, когда $i=j=k$ и $a=a',b=b',c=c'$.
\end{definition}

Если $\mathscr{C}$ реализует $\left\langle \ell_1,m_1,n_1 \right\rangle \oplus \dotsb \oplus \left\langle \ell_k,m_k,n_k \right\rangle$, и $T$ --- структурный тензор $\CC[\mathscr{C}]$, то
\[
	R_s(\left\langle \ell_1,m_1,n_1 \right\rangle \oplus \dotsb \oplus \left\langle \ell_k,m_k,n_k \right\rangle) \leq R(T).
\]
Можно скопировать доказательство асимптотического неравенства для сумм, чтобы показать
\begin{equation}\label{eq:12:5.1}
	(\ell_1 m_1 n_1)^{\omega_s/3} + \dotsb + (\ell_k m_k n_k)^{\omega_s/3} \leq R_s(\left\langle \ell_1,m_1,n_1 \right\rangle \oplus \dotsb \oplus \left\langle \ell_k,m_k,n_k \right\rangle),
\end{equation}
из чего можно получать оценки на $\omega_s$. Вместо этого в данном разделе будет разработан эффективный алгебраический метод для объединения этих независимых матричных умножений в одно. Это даст те же оценки, но также покажет, что эти оценки можно получить реализацией единственного матричного умножения в когерентной конфигурации. Для начала приведём пример. Он будет экстремальным потому, что он реализует прямую сумму $n^{1-o(1)}$ копий $\left\langle n,n,n \right\rangle$ при помощи когерентной конфигурации ранга $n^3$, и это нельзя сделать, если ранг будет меньше, чем $n^{3-o(1)}$ (потому что образы вложений не должны пересекаться). Если бы когерентная конфигурация была коммутативной, можно было бы заключить, что $\omega=2$, но она далека от коммутативности.

\begin{example}\label{ex:12:5.2}
  Пусть $\mathscr{C}$ --- когерентная конфигурация, соответствующая диагональному действию $\mathbb{Z}/n \mathbb{Z}$ на $(\mathbb{Z}/n \mathbb{Z})^2$ (то есть действию $s \cdot (a,b) = (s + a, s + b)$, где $s \in \mathbb{Z}/n \mathbb{Z},\; (a,b) \in (\mathbb{Z}/n \mathbb{Z})^2$), и пусть $S \subseteq \mathbb{Z}/n \mathbb{Z}$ --- множество размера $|S|=n^{1-o(1)}$, не содержащее арифметических прогрессий из трёх элементов \cite{Salem}. Можно проиндексировать классы в $\mathscr{C}$ так
 \[
 	R_{(a,b,c)}=\left\{ ((s,s+a),(s+a+b,s+a+b+c)) \mid s \in \mathbb{Z}/n \mathbb{Z} \right\},
 \]
  где $a,b,c \in \mathbb{Z}/n \mathbb{Z}$. Тогда $\mathscr{C}$ реализует $\bigoplus_{i \in S} \left\langle n,n,n \right\rangle$ через отображения $\alpha_i, \beta_i, \gamma_i$, определённые для $i \in S$ следующим образом
  \begin{align*}
    	\alpha_i (x,y) & = (x, i-x, y)   \\
    	\beta_i (y,z) & = (y,i-y,z) \\
    	\gamma_i (z,x) & = (z, -2i - z, x)
  \end{align*}
  Классы
  \begin{align*}
       R_{\alpha_i (x, y')} = R_{(x, i-x, y')} & = \left\{ ((s,s+x),(s+i,s+i+y')) \mid s \in \mathbb{Z}/n \mathbb{Z} \right\},\\
       R_{\beta_j (y, z')} = R_{(y,j-y,z')} & = \left\{ ((s,s+y),(s+j,s+j+z')) \mid s \in \mathbb{Z}/n \mathbb{Z} \right\},\\
       R_{\gamma_k (z, x')} = R_{(z, -2k - z, x')} & = \left\{ ((s,s+z),(s-2k,s-2k+x')) \mid s \in \mathbb{Z}/n \mathbb{Z} \right\}
  \end{align*}
  образуют треугольник тогда и только тогда, когда существуют $s',s'',s''' \in \mathbb{Z}/n \mathbb{Z}$ такие, что
  \begin{empheq}[left=\empheqlbrace]{align*}
	&(s'+i,s'+i+y') = (s'',s''+y)\\
	&(s''+j,s''+j+z') = (s''',s'''+z) \\
	&(s'''-2k,s'''-2k+x') = (s',s'+x).
  \end{empheq}
  Откуда получим, что $x=x',y=y',z=z'$ и $i+j=2k$ (в этом случае $i=j=k$ потому, что $S$ не содержит арифметических прогрессий из трёх элементов). Однако этот пример не доказывает нетривиальных оценок на $\omega$ потому, что на самом деле все степени характеров $\mathscr{C}$ равны $n$ (повторяются $n$ раз).
\end{example}

Как и обещалось, сейчас будет дано конструктивное доказательство \eqref{eq:12:5.1}. Доказательство преобразует когерентную конфигурацию, реализующую несколько независимых матричных умножений, в одну когерентную конфигурацию, реализующую \textbf{единственное} матричное умножение. Более того результирующая когерентная конфигурация будет коммутативной, если первоначальная была таковой. Так как это доказательство на самом деле строит когерентную конфигурацию, а не просто выводит оценки $\omega_s$, можно использовать его для получения коммутативных когерентных конфигураций (даже ассоциативных схем), которые доказывают нетривиальные оценки $\omega$. Это является одним из основных моментов статьи 2012 года. Некоммутативности, которая была необходима в теоретико-групповом подходе, можно избежать в обобщении, использующем когерентные конфигурации.

Авторы также нашли, что следствием любой из двух гипотез \ref{conj:05:3.4} или \ref{conj:05:4.7} из статьи 2005 года будет то, что коммутативных когерентных конфигураций (даже ассоциативных схем) будет достаточно для доказательства $\omega=2$. Это усиливает веру в то, что можно найти коммутативную конфигурацию ранга $n^{2+o(1)}$, которая реализует $\left\langle n,n,n \right\rangle$, и таким образом доказать $\omega=2$.

\begin{theorem}\label{th:12:5.3} 
  Пусть $\mathscr{C}$ --- когерентная конфигурация ранга $r$, которая реализует $\bigoplus_{i=1}^k \left\langle \ell_i, m_i, n_i  \right\rangle$. Тогда симметрическая степень $Sym^k \mathscr{C}$ реализует $\left\langle \prod_i \ell_i, \prod_i m_i, \prod_i n_i \right\rangle$ и имеет ранг $\binom{r+k-1}{k}$.
\end{theorem}
\begin{proof}
  Классы $R_I$ $k$-кратного прямого произведения $\mathscr{C}$ проиндексированы векторами $I \in [r]^k$. Симметрическая группа $S_k$ действует на $[r]^k$, переставляя $k$ координат, и орбиты этого действия естественным образом соответствуют $k$-мультиподмножествам множества $[r]$, то есть множествам из $k$ элементов, допускающим включение одного и того же элемента из $[r]$ по нескольку раз. Напомню, что $Sym^k \mathscr{C}$ --- это конфигурация слияния с классом для каждой орбиты, то есть для каждого $k$-мультиподмножества $S$ множества $[r]$ мы имеем класс $R_S'$, который будет объединением $R_I$ для всех $I$ из орбиты, соответствующей $S$. Ранг $Sym^k \mathscr{C}$ --- это число различных орбит (то же самое, что число различных $k$-мультиподмножеств множества $[r]$), которое равно числу сочетаний с повторением из $r$ по $k$, то есть $\binom{r+k-1}{r-1} = \binom{r+k-1}{k}$.
  
 Положим $L=\prod_i \ell_i,\; M=\prod_i m_i$ и $N=\prod_i n_i$. Теперь $\mathscr{C}^k$ реализует $\left\langle L,M,N \right\rangle$ поэтому существуют инъективные функции
 \begin{question}
 	Мы не доказывали утверждения, что если $\mathscr{C}$ реализует $\bigoplus_{i=1}^k \left\langle \ell_i, m_i, n_i  \right\rangle$, то $\mathscr{C}^k$ реализует $\left\langle L,M,N \right\rangle$.
 \end{question}
 \begin{align*}
 	\alpha &: [L] \times [M] \to [r]^k\\
 	\beta &: [M] \times [N] \to [r]^k\\
 	\gamma &: [N] \times [L] \to [r]^k
 \end{align*}
 удовлетворяющие условиям из определения \ref{def:12:4.1}, а именно
 \begin{align*}
   \alpha(A,B) & = (\alpha_1(A_1,B_1), \dotsc, \alpha_k(A_k,B_k))   \\
   \beta(B,C) & = (\beta_1(B_1,C_1), \dotsc, \beta_k(B_k,C_k))\\
   \gamma(C,A) & = (\gamma_1(C_1,A_1), \dotsc, \gamma_k(C_k,A_k)),
 \end{align*}
 где $\mathscr{C}$ реализует $\bigoplus_{i=1}^k \left\langle \ell_i, m_i, n_i  \right\rangle$ через $\alpha_i,\beta_i,\gamma_i$.
 
 Утверждается, что на самом деле $\alpha,\beta,\gamma$ инъективны даже в конфигурации слияния, где мы деформируем орбиты. Действительно, предположим, что $\alpha(A,B)=\pi \alpha(A',B')$ для какого-то $\pi \in S_k$. Тогда $\pi$ должна быть тождественной, так как отображения $\alpha_i$ имеют непересекающиеся образы в $[r]$ (это следует немедленно из определения \ref{def:12:5.1}), и затем благодаря инъективности $\alpha_i$ получим $(A,B)=(A',B')$. Тоже самое верно для $\beta$ и $\gamma$.
 
Более того орбиты $\alpha(A,B'), \beta(B,C'), \gamma(C,A')$ образуют треугольник в $Sym^k \mathscr{C}$ тогда и только тогда, когда $A=A',B=B'$ и $C=C'$. Действительно, предположим, что существуют точки $X,Y,Z$ и перестановки $\pi_1,\pi_2,\pi_3 \in S_k$, для которых $(X,Y) \in R_I, (Y,Z) \in R_J$ и $(Z,X) \in R_K$, где $I=\pi_1 \alpha(A,B'),J=\pi_2 \beta(B,C')$ и $K=\pi_3 \gamma(C,A')$. Тогда мы должны получить $\pi_1=\pi_2=\pi_3$ потому, что $\alpha_i(A_i,B_i'), \beta_j(B_j,C_j'), \gamma_k(C_k,A_k')$ могут образовывать треугольник, только если $i=j=k$, и затем $A'=A,B=B'$ и $C=C'$ следуют из факта, что эти равенства верны для каждой координаты (из-за свойств $\alpha_i,\beta_i,\gamma_i$). Таким образом, $Sym^k \mathscr{C}$ реализует $\left\langle L,M,N \right\rangle$, как и утверждалось.
\end{proof}

\begin{corollary}\label{cor:12:5.4}
  Пусть $\mathscr{C}$ --- коммутативная когерентная конфигурация ранга $r$, реализующая $\bigoplus_{i=1}^k \left\langle \ell_i, m_i, n_i  \right\rangle$. Тогда симметрические степени прямых степеней $\mathscr{C}$ доказывают оценку
  \[
  	k \cdot \left( \left( \prod_{i=1}^k \ell_i m_i n_i \right)^{1/k} \right)^{\omega_s/3} \leq r.
  \]
\end{corollary}
Точнее они подходят произвольно близко к этой оценке.
\begin{proof}
  Прямые произведения $\mathscr{C}^t$ реализуют $\bigoplus_{I \in [k]^t} \left\langle \prod_{j \in [t]} \ell_{I_j}, \prod_{j \in [t]} m_{I_j}, \prod_{j \in [t]} n_{I_j}  \right\rangle$. Присвоив $L =\prod_{i \in [k]} \ell_i,\; M = \prod_{i \in [k]} m_i$ и $N = \prod_{i \in [k]} n_i$, получим из теоремы \ref{th:12:5.3}, что $Sym^k \mathscr{C}^t$ реализует 
  \[
  	\left\langle 
  		\prod_{I \in [k]^t} \prod_{j \in [t]} \ell_{I_j},  
  		\prod_{I \in [k]^t} \prod_{j \in [t]} m_{I_j},
  		\prod_{I \in [k]^t} \prod_{j \in [t]} n_{I_j}
  	\right\rangle = \left\langle L^{t k^{t-1}}, M^{t k^{t-1}}, N^{t k^{t-1}} \right\rangle
  \]
  и имеет ранг $\binom{r^t+k^t-1}{k^t}$. Из утверждения \ref{prop:12:3.5} получим
  \[
  	(LMN)^{\omega_s t k^{t-1}/3} \leq \dbinom{r^t+k^t-1}{k^t} \leq \left( \frac{e (r^t+k^t-1)}{k^t} \right)^{k^t} \leq \left( \frac{2 e r^t}{k^t} \right)^{k^t},
  \]
  где последнее неравенство использует факт, что $k \leq r$. Взяв $t k^t$-ый корень и устремив $t$ в бесконечность, получим оценку $(LMN)^{\omega_s/(3k)} \leq r/k$.  
\end{proof}

При помощи подходящих взвешиваний независимых матричных умножений найдём, что геометрическое среднее может быть заменено арифметическим средним, для получения оценки на $\omega_s$ аналогичной асимптотическому неравенству для сумм \eqref{eq:12:2.1}.

\begin{theorem}\label{th:12:5.5} 
  Пусть $\mathscr{C}$ --- коммутативная когерентная конфигурация ранга $r$, реализующая $\bigoplus_{i=1}^k \left\langle \ell_i, m_i, n_i  \right\rangle$. Тогда симметрические степени прямых степеней $\mathscr{C}$ доказывают оценку $\sum_i ( \ell_i m_i n_i)^{\omega_s/3} \leq r$.
\end{theorem}
\begin{proof}
  Зафиксируем целое $N$ и $\mu = (\mu_1, \dotsc, \mu_k)$ такие, что $\mu_i \geq 0$ и $\sum_i \mu_i = N$. Тогда прямое произведение $\mathscr{C}^N$ реализует $L=\binom{N}{\mu}$ независимых копий $\left\langle \prod_i \ell_i^{\mu_i},  \prod_i m_i^{\mu_i}, \prod_i n_i^{\mu_i}\right\rangle$ (ключевой момент в том, что теперь они все имеют одинаковый размер). Применяя следствие \ref{cor:12:5.4}, найдём, что симметрические степени прямых степеней $\mathscr{C}$ доказывают оценку 
  \[
  	L \cdot \left( \prod_{j=1}^L \left( \prod_i \ell_i^{\mu_i} \cdot  \prod_i m_i^{\mu_i} \cdot  \prod_i n_i^{\mu_i} \right) \right)^{\omega_s/3L} \leq r^N
  \]
  \begin{equation}\label{eq:12:5.2}
  	\dbinom{N}{\mu} \prod_i (\ell_i m_i n_i)^{\mu_i \omega_s/3} \leq r^N.
  \end{equation}
  Суммирование этого неравенства по всем $\mu$ даёт
  \[
  	\left( \sum_i (\ell_i m_i n_i)^{\omega_s/3} \right)^N \leq \dbinom{N+k-1}{k-1} \cdot r^N,
  \]
  и утверждение теоремы получается после взятия $N$-ых корней и устремления $N$ к бесконечности. Обратите внимание, что для любого $N$  с помощью аргумента усреднения получается, что должно быть определенное распределение $\mu$, для которого левая часть уравнения \eqref{eq:12:5.2} по меньшей мере равна $\left( \sum_i (\ell_i m_i n_i)^{\omega_s/3} \right)^N / \binom{N+k-1}{k-1}$, и это будет конкретной последовательностью когерентных конфигураций, доказывающей ту же оценку (в пределе).
\end{proof}

Эти результаты не будут специфичны для когерентных конфигураций. Если взять любую алгебру $A$, то примером аналогичной конструкции будет подалгебра алгебры $A^{\otimes n}$ инвариантная относительно действия $S_n$.

Прежде чем использовать теорему \ref{th:12:5.5} для получения оценок $\omega$, кратко сравним другие доказательства асимптотического неравенства для сумм с доказательством, приведённым выше, которое окажется конструктивно другим, как сейчас будет объяснено. Стандартное доказательство асимптотического неравенства для сумм берёт тензор $T$, реализующий $\bigoplus_{i=1}^k \left\langle \ell_i, m_i, n_i  \right\rangle$, и находит $k!$ независимых копий $\left\langle \prod_i \ell_i^{\mu_i},  \prod_i m_i^{\mu_i}, \prod_i n_i^{\mu_i}\right\rangle$ в $T^{\otimes k}$, которые способны через выполнение блочного матричного умножения реализовать случай большего матричного умножения $\left\langle K \cdot \prod_i \ell_i^{\mu_i}, K \cdot \prod_i m_i^{\mu_i}, K \cdot \prod_i n_i^{\mu_i}\right\rangle$ (где $K \approx k!^{1/\omega}$), и общая оценка получается после нескольких преобразований аналогичных тем, что были в следствии \ref{cor:12:5.4} и теореме \ref{th:12:5.5}. Напротив, доказательство, данное авторами статьи, находит единственную копию $\left\langle \prod_i \ell_i^{\mu_i},  \prod_i m_i^{\mu_i}, \prod_i n_i^{\mu_i}\right\rangle$ в $T^{\otimes k}$ и затем использует тот факт, что $T$ имеет специальное строение --- он является структурным тензором алгебры --- чтобы показать, что тот же случай матричного умножения сможет пережить симметризацию $k$-ой степени. Симметризация уменьшает ранг, и таким образом $s$-ранг на самом деле сокращается достаточно для получения той же оценки. Авторам не известно других доказательств, которые используют сокращение ранга.

\subsection{Нетривиальные оценки $\omega$}

Используя теорему \ref{th:12:5.5}, можно преобразовать любой результат из статьи 2005 года в реализацию тензора единственного матричного умножения в коммутативной когерентной конфигурации, а именно в симметрической степени абелевой группы. Хотя изначальные построения не новы, конечный алгоритмы таковыми являются (они используют аппарат введённый в статье 2012 года). Они устанавливают, что коммутативных когерентных конфигураций достаточно для доказательства нетривиальных оценок $\omega$, и даже указывают на особое семейство коммутативных когерентных конфигураций, которое, как предполагают авторы, способно доказать $\omega=2$.

\begin{theorem}\label{th:12:5.6} 
  Существуют коммутативные когерентные конфигурации, которые доказывают оценки $s$-ранговой экспоненты $\omega_s \leq 2.48, \omega_s \leq 2.41$ и $\omega_s \leq 2.376$, и таким образом соответствующие оценки экспоненты $\omega \leq 2.72, \omega \leq 2.62$ и $\omega \leq 2.564$.
\end{theorem}
\begin{proof}
  Применим теорему \ref{th:12:5.5} к построениям абелевых групп из утверждения \ref{prop:05:3.8}, из теорем \ref{th:05:3.3} и \ref{th:05:6.6}, и из обобщения, сочетающегося с \cite{Coppersmith:1990} (как указано в статье 2005 года, но не описано подробно), соответственно. Каждое из этих построений, если смотреть на группы как на когерентные конфигурации и позаимствовать язык статьи 2012 года, даёт когерентную конфигурацию, удовлетворяющую определению \ref{def:12:5.1}. Применив теорему \ref{th:12:3.6} к полученным оценкам $s$-ранговой экспоненты, придём к требуемым оценкам $\omega$.
\end{proof}

Все конкретные оценки экспоненты, указанные выше, страдают от 50 процентного штрафа, введённого в теореме \ref{th:12:3.6}. Но сами по себе числа не должны скрывать ту главную мысль, что матричное умножение с использованием когерентных конфигураций является жизнеспособным подходом к доказательству $\omega=2$. В самом деле, авторы полагают, что коммутативных когерентных конфигураций достаточно, чтобы доказать $\omega=2$:
\begin{conj}\label{conj:12:5.7}
  Существуют коммутативные когерентные конфигурации $\mathscr{C}_n$, реализующие $\left\langle n,n,n \right\rangle$, и имеющие ранг $n^{2+o(1)}$.
\end{conj}

Такое семейство коммутативных когерентных конфигураций будет доказывать $\omega=2$. Если гипотеза \ref{conj:05:3.4} или гипотеза \ref{conj:05:4.7} верна, тогда через теорему \ref{th:12:5.5} будет верна и гипотеза \ref{conj:12:5.7}. Среди различных комбинаторных и алгебраических гипотез, влекущих $\omega=2$, гипотеза \ref{conj:12:5.7} самая слабая (она следует из других), что делает её <<самой простой>> среди возможных способов доказательства $\omega=2$.

