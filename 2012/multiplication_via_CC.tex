\section{Матричное умножение посредством когерентных конфигураций}

В этом разделе будет описано как вложить матричное умножение в умножение в алгебре. Для этого было бы полезно иметь представление об одной важной комбинаторной структуре. Как упоминалось во введении к статье 2012 года, алгебры смежности для когерентных конфигураций являются многообещающим семейством алгебр для такого использования. В подразделах \ref{ssub:12:4.2} и \ref{ssub:12:4.3} будет рассмотрена основная теория когерентных конфигураций и их алгебр смежности, в подразделе \ref{ssub:12:4.4} теоретико-групповой подход будет приспособлен к этому новому окружению.

\subsection{Выполнение матричного умножения в алгебрах}
	
Можно оценить $s$-ранг матричного умножения при помощи ограничения структурного тензора алгебры. Пусть $A$ --- конечномерная комплексная алгебра, и пусть $u_1, \dotsc, u_r$ --- базис $A$. Тогда существуют коэффициенты $\lambda_{i,j,k}$ такие, что 
\[
	u_i u_j = \sum_{k=1}^r \lambda_{i,j,k} u_k.
\]
Они называются \textbf{структурными константами} $A$ относительно этого базиса. \textbf{Структурный тензор} --- это трилинейная форма
\[
	\sum_{i,j,k} \lambda_{i,j,k} \widehat{x}_i \widehat{y}_j \widehat{z}_k.
\]

Он изоморфен тензору умножения (то есть элемент $A^* \otimes A^* \otimes A$ соответствует отображению умножения из $A \otimes A$ в $A$). 

В более общем смысле, если мы используем любые три базиса $u_1, \dotsc, u_r, \; v_1, \dotsc, v_r, \; w_1, \dots, w_r$ для $A$ и определим коэффициенты как 
\[
	u_i v_j = \sum_{k=1}^r \lambda_{i,j,k} w_k,
\]
тогда соответствующий тензор изоморфен структурному тензору.
\begin{definition}\label{def:12:4.1}
   Пусть $A$ --- $r$-мерная комплексная алгебра со структурными константами $\lambda_{i,j,k}$, соответствующими какому-то выбору базиса. Будем говорить, что \textbf{$A$ реализует} $\left\langle \ell, m, n \right\rangle$, если существуют три инъективные функции
   \[
   	\alpha: [\ell] \times [m] \to [r], \; \beta: [m] \times [n] \to [r], \; \gamma: [n] \times [\ell] \to [r]
   \]
   такие, что 
   \[
   	\lambda_{\alpha(a,b'), \beta(b,c'), \gamma(c,a')} \neq 0
   \]
   тогда и только тогда, когда $a=a',b=b'$ и $c=c'$.
\end{definition}

Обратите внимание, что это определение зависит от выбора базиса. Обычно мы не будем явно его задавать, потому что у обсуждаемых алгебр всегда имеется стандартный базис.

\begin{prop}\label{prop:12:4.2}
   Если алгебра $A$ реализует $\left\langle \ell,m,n \right\rangle$, то $s$-ранг $\left\langle \ell,m,n \right\rangle$ не превосходит ранг структурного тензора для $A$.  
\end{prop}
\begin{proof}
   Предположим, что $A$ реализует $\left\langle \ell,m,n \right\rangle$ через $\alpha, \beta, \gamma$, и рассмотрим структурный тензор
     \[
     	\sum_{i,j,k} \lambda_{i,j,k} \widehat{x}_i \widehat{y}_j \widehat{z}_k.
     \]
   Определим $\widehat{u}_{a,b'} = \widehat{x}_{\alpha(a,b')}, \widehat{v}_{b,c'}=\widehat{y}_{\beta(b,c')}, \widehat{w}_{c,a'}=\widehat{z}_{\gamma(c,a')}$, более того положим $\widehat{x}_i=0$, когда $i$ не содержится в образе $\alpha$, $\widehat{y}_j=0$, когда $j$ не содержится в образе $\beta$, и $\widehat{z}_k=0$, когда $k$ не содержится в образе $\gamma$. При такой замене переменных структурный тензор принимает следующий вид
     \[
     	\sum_{a,a',b,b',c,c'} \lambda_{\alpha(a,b'), \beta(b,c'), \gamma(c,a')} \widehat{u}_{a,b'} \widehat{v}_{b,c'} \widehat{w}_{c,a'},
     \]
   и по предположению, члены не равны нулю, только когда $a=a', b=b'$ и $c=c'$. Таким образом,
   \[
   	\sum_{a,b,c} \lambda_{\alpha(a,b), \beta(b,c), \gamma(c,a)} \widehat{u}_{a,b} \widehat{v}_{b,c} \widehat{w}_{c,a}
   \]
   имеет ранг не превосходящий ранг структурного тензора. Этот новый тензор является взвешенной версией тензора матричного умножения $\left\langle \ell,m,n \right\rangle$, отсюда $s$-ранг $\left\langle \ell,m,n \right\rangle$ не превосходит ранга структурного тензора.
\end{proof}

Обратите внимание, что это доказательство на самом деле даёт очень простой алгоритм для сведения взвешенного матричного умножения к умножению в алгебре тем же способом, что и в статье 2003 года.

Напомню, что алгебра называется \textbf{полупростой}, если она является произведением матричных алгебр. Другими словами, она будет полупростой, если существуют степени характеров $d_1, \dotsc, d_t$ такие, что 
\[
	A \cong \mathbb{C}^{d_1 \times d_1} \times \dotsb \times \mathbb{C}^{d_t \times d_t}.
\]
В этом случае структурный тензор изоморфен $\left\langle d_1,d_1,d_1 \right\rangle \oplus \dotsb \oplus \left\langle d_t,d_t,d_t \right\rangle$.

\begin{prop}\label{prop:12:4.3}
     Если полупростая алгебра $A$ со степенями характеров $d_1, \dotsc, d_t$ реализует $\left\langle \ell,m,n \right\rangle$, тогда
     \[
     	(\ell m n)^{\omega_s / 3} \leq d_1^{\omega} + \dotsb + d_t^{\omega}.
     \]
\end{prop}

Это утверждение также влечёт за собой естественный алгоритм, который сводит взвешенное матричное умножение к коллекции невзвешенных матричных умножений.
\begin{proof}
  Для любого $\varepsilon > 0$ существует константа $C$ такая, что $R(\left\langle d,d,d \right\rangle) \leq C d^{\omega + \varepsilon}$ для всех $d$. Отсюда следует, что
  \begin{align*}
       (\ell m n)^{\omega_s / 3} & \leq R_s(\left\langle \ell, m, n \right\rangle) \\
        & \leq \text{ ранга структурного тензора алгебры } A \text{ (утверждение \ref{prop:12:4.2}) }\\
        & = R(\left\langle d_1,d_1,d_1 \right\rangle \oplus \dotsb \oplus \left\langle d_t,d_t,d_t \right\rangle) \text{ (так как $A$ полупростая) }\\
        & \leq R(\left\langle d_1,d_1,d_1 \right\rangle) + \dotsb + R(\left\langle d_t, d_t, d_t \right\rangle) \text{ (лемма \ref{lem:bi:4.5}) }\\
        & \leq C d_1^{\omega+\varepsilon} + \dotsc + C d_t^{\omega + \varepsilon},
  \end{align*}
  но $C$ и $\varepsilon$ создают проблемы. Чтобы убрать их, используем трюк с вычислением асимптотического ранга для больших тензорных степеней. Алгебра $A^{\otimes N}$ реализует взвешенную версию $\left\langle \ell,m,n \right\rangle^{\otimes N}=\left\langle \ell^N,m^N,n^N \right\rangle$, и она имеет степени характеров, даваемые $N$-кратными произведениями $d_{i_1}, \dotsc, d_{i_N}$. Таким образом,
  \[
  	(\ell m n)^{N \omega_s / 3} \leq C (d_1^{\omega+\varepsilon} + \dotsc + d_t^{\omega + \varepsilon})^N.
  \]
  Взяв теперь $N$-ый корень и устремив $N$ к бесконечности получим
  \[
  	(\ell m n)^{\omega_s/3} \leq d_1^{\omega+\varepsilon} + \dotsc + d_t^{\omega + \varepsilon}
  \]
  и так как это верно для всех $\varepsilon > 0$, это будет верно и при $\varepsilon=0$ по непрерывности.
\end{proof}

Определение \ref{def:12:4.1} обобщает свойство тройного произведения. Чтобы понять почему, предположим, что $A$ --- это групповая алгебра конечной группы, и возьмём сами групповые элементы как базис. Тогда $\alpha(a,b'), \beta(b,c')$ и $\gamma(c,a')$ соответствуют групповым элементам $g_{a,b'}, h_{b,c'}$ и $k_{a',c}$ таким, что
\begin{equation}\label{eq:12:4.1}
	g_{a,b'} h_{b,c'} = k_{a',c}
\end{equation}
тогда и только тогда, когда $a=a',b=b'$ и $c=c'$. Мы хотим найти групповые элементы $s_a,t_b,u_c$ такие, что $g_{a,b}=s_a^{-1} t_b, h_{b,c}=t_b^{-1} u_c$ и $k_{a,c}=s_a^{-1} u_c$. Тогда $\left\{ s_a \right\}, \left\{ t_b \right\}, \left\{ u_c \right\}$ удовлетворяют свойству тройного произведения, так как тогда
\[
	s_{a'} s_a^{-1} t_{b'} t_b^{-1} u_{c'} u_c^{-1} = 1 \implies s_a^{-1} t_{b'} t_b^{-1} u_{c'} = s_{a'}^{-1} u_c \implies g_{a, b'} h_{b, c'} = k_{a',c} \implies a = a', b = b', c = c'.
\]

Чтобы их найти, зафиксируем $b_0$ и пусть $s_a=g_{a,b_0}^{-1}$ и $u_c=h_{b_0,c}$. Тогда $s_a^{-1} u_c = k_{a,c}$ автоматически. Более того \eqref{eq:12:4.1} означает, что $g_{a,b_0}^{-1} g_{a,b} = h_{b_0,c} h_{b,c}^{-1}$ (так как 
$g_{a,b} h_{b,c}  = k_{a,c} = g_{a,b_0} h_{b_0,c}$), где этот групповой элемент не зависит от $a$ и $c$. Назвав его $t_b$, завершим построение. В качестве проверки
\begin{align*}
  s_a^{-1} t_b & = \left( g_{a, b_0}^{-1} \right)^{-1} g_{a,b_0}^{-1} g_{a,b} = g_{a,b}\\
  t_b^{-1} u_c & = \left( h_{b_0,c} h_{b,c}^{-1} \right)^{-1} h_{b_0,c} = h_{b,c}\\
  s_a^{-1} u_c & = \left( g_{a,b_0}^{-1} \right)^{-1} h_{b_0,c} = k_{a,c}.
\end{align*}


\subsection{Когерентные конфигурации}\label{ssub:12:4.2}

Когерентные конфигурации являются весьма примечательными структурами, которые вобрали в себя многое из теории групп и алгебраической комбинаторики \cite{Hig70,Hig72,Hig75}. 

\begin{definition}\label{def:12:cc}
   \textbf{Когерентная конфигурация} (англ. coherent configuration) ранга $r$ --- это конечное множество $\mathscr{C}$, чьи элементы называются точками, вместе с разбиением $\mathscr{C}^2$ на подмножества $R_1, \dotsc, R_r$, называемыми классами, такими, что
   \begin{enumerate}
     \item \label{item:1:def:12:cc} диагональ $\left\{ (x,x) \mid x \in \mathscr{C} \right\}$ является объединением каких-то классов
     \item \label{item:2:def:12:cc} для любого $i \in [r]$ существует $i^* \in [r]$ такой, что $R_i^* = R_{i^*}$, где
     \[
     	R_i^* = \left\{ (b,a) \mid (a,b) \in R_i \right\}
     \]
     \item \label{item:3:def:12:cc} существуют целые $p_{i,j}^k$ для $i,j,k \in [r]$ такие, что для всех $x,y \in \mathscr{C}$, где $(x,y) \in R_k$
     \[
     	\#\left\{ z \in \mathscr{C} \mid (x,z) \in R_i \text{ и } (z,y) \in R_j \right\} = p_{i,j}^k.
     \]
   \end{enumerate}
\end{definition}
Будем говорить, что $\mathscr{C}$ \textbf{симметрична}, если $R_i^*=R_i$ для всех $i$, и \textbf{коммутативна}, если $p_{i,j}^k=p_{j,i}^k$ для всех $i,j,k$. (Симметрия влечёт коммутативность, но не наоборот.) Числа $p_{i,j}^k$ называются \textbf{числами пересечений} (англ. intersection numbers) конфигурации. (Они так названы потому, что $p_{i,j}^k$ является мощностью пересечения $R_i(x) \cap R_j^*(y)$ для $(x, y) \in R_k$, где $R(x) = \cb{z \in \mathscr{C} \mid (x, z) \in R}$.) Конфигурация называется \textbf{ассоциативной схемой} (англ. association scheme), если диагональ сама по себе является одним из классов. Легко доказать, что коммутативная когерентная конфигурация должна быть ассоциативной схемой \cite[14]{Hig75}.

Любая конечная группа $G$ определяет ассоциативную схему, где $G$ будет множеством точек, а $G^2$ разбивается на подмножества $R_g = \left\{ (h,hg) \mid h \in G \right\}$, где $g \in G$. Тогда для $g,h,k \in G$
\[
	p_{g,h}^k = 
	\begin{cases} 
		1, & \mbox{если } gh=k\mbox{ и} \\ 
		0 & \mbox{иначе}.
	\end{cases}
\]
Числа пересечений кодируют таблицу умножения группы, поэтому группа и соответствующая ассоциативная схема являются полностью эквивалентными структурами. Обратите внимание, что эта ассоциативная схема коммутативна тогда и только тогда, когда $G$ коммутативна; она симметрична тогда и только тогда, когда $g=g^{-1}$ для всех $g \in G$. Можно показать, что ассоциативная схема получается из группы таким способом тогда и только тогда, когда все её числа пересечений не превосходят 1.

В более общем смысле, пусть $G$ действует на конечном множестве $X$. Тогда разбиение $X^2$ на орбиты под диагональным действием $G$ определяет когерентную конфигурацию, называемую \textbf{шуровой когерентной конфигурацией} (англ. Schurian coherent configuration). Она является ассоциативной схемой тогда и только тогда, когда $G$ действует транзитивно на $X$.

Многие важные примеры из комбинаторики вписываются в это построение. Например, схема Хемминга состоящая из точек в $\left\{ 0,1 \right\}^n$ с классами, определяемыми при помощи расстояния Хемминга. С точки зрения теории групп, она будет шуровой ассоциативной схемой, определяемой при помощи действия полупрямого произведения $S_n \ltimes (\mathbb{Z}/2 \mathbb{Z})^n$ на $\left\{ 0,1 \right\}^n$, хотя эта формулировка будет чрезмерной для большинства приложений.

Если $G$ действует транзитивно на $X$, то можно отождествить $X$ с $G/H$, где $H$ является стабилизатором точки в $X$. Обратите внимание, что $G/H$ будет группой, только если $H$ нормальна в $G$, зато она всегда будет ассоциативной схемой. В определённых случаях, называемых парами Гельфанда $(G,H)$, фактор $G/H$ будет коммутативной ассоциативной схемой (хотя обычно группы $G$ и $H$ не являются коммутативными). Например, это так для схемы Хемминга.

Существуют также многочисленные комбинаторные примеры ассоциативных схем и когерентных конфигураций, которые не происходят от групп симметрий. Например, сильно регулярный граф --- это то же самое, что и симметрическая ассоциативная схема ранга 3. В более общем смысле, любой дистанционно-регулярный граф (англ. distance-regular graph) будет ассоциативной схемой, когда классы определяются при помощи метрик графа. Некоторые из этих графов будут шуровыми ассоциативными схемами, но большинство --- нет. 

\textbf{Слияние} (англ. fusion) когерентной конфигурации $\mathscr{C}$ --- это конфигурация $\mathscr{C}'$ с тем же множеством точек и с классами, получаемыми объединением классов $\mathscr{C}$. (Обратите внимание, что делать объединение классов нужно аккуратно потому, что не любое объединение даст когерентную конфигурацию.) Другой важной конструкцией является \textbf{прямое произведение}: для двух когерентных конфигураций $\mathscr{C}$ и $\mathscr{C}'$, их произведение $\mathscr{C} \times \mathscr{C}'$ имеет в качестве множества точек прямое произведение точек исходных конфигураций, а класс пары точек $((c_1,c_1'),(c_2,c_2'))$ из $\mathscr{C} \times \mathscr{C}'$ определяется классом пары $(c_1,c_2)$ из $\mathscr{C}$ и классом пары $(c_1',c_2')$ из $\mathscr{C}'$. \textbf{Симметрической степенью} (обозначается $Sym^k \mathscr{C}$) (англ. symmetric power) называется слияние, получаемое путём объединения классов прямой степени $\mathscr{C}^k$ в соответствии с действием симметрической группы $S_k$ на факторы.

\subsection{Алгебра смежности}\label{ssub:12:4.3}

Любая когерентная конфигурация имеет ассоциированную с ней алгебру, которая играет ту же роль что и групповая алгебра для группы. Пусть $A_1, \dotsc, A_r$ --- матрицы смежности для отношений $R_1, \dotsc, R_r$. Другими словами $A_i$ проиндексированы множеством $\mathscr{C}$ следующим образом
\[
	(A_i)_{x,y} =
	\begin{cases}
	  1 & \text{ если } (x,y) \in R_i \text{ и}\\
	  0 & \text{ иначе}
	\end{cases}
\]
для $x,y \in \mathscr{C}$. \textbf{Алгебра смежности} $\mathbb{C}[\mathscr{C}]$ когерентной конфигурации $\mathscr{C}$ --- это комплексная алгебра, порождаемая этими матрицами смежности. (Обратите внимание, что она содержит единицу, так как диагональ будет объединением классов. Например, пусть $R'$ и $R''$ два класса $\mathscr{C}$, составляющие её диагональ, тогда у матриц смежности этих отношений $A'$ и $A''$ единицы будут стоять исключительно на диагонали, но в различных строках, поэтому $A' + A'' = I$.) Простые вычисления показывают, что
\[
	A_i A_j = \sum_k p_{i,j}^k A_k,
\]
поэтому $\mathbb{C}[\mathscr{C}]$ является линейной оболочкой матриц $A_1, \dotsc, A_r$. Она будет коммутативной алгеброй тогда и только тогда, когда $\mathscr{C}$ --- коммутативна.

\textbf{Эрмитовым сопряжением} (англ. conjugate transpose) называют операцию, при которой исходную матрицу транспонируют и заменяют каждый её элемент на комплексно-сопряжённый. 
Алгебра смежности замкнута относительно этого действия, поэтому она будет полупростой алгеброй. Таким образом, существуют степени характеров $d_1, \dotsc, d_k$ такие, что
\[
	\mathbb{C}[\mathscr{C}] \cong \mathbb{C}^{d_1 \times d_1} \times \dotsb \times \mathbb{C}^{d_k \times d_k}.
\]
Конечно, сумма $d_1^2 + \dotsb + d_k^2$ должна быть равна размерности $\mathbb{C}[\mathscr{C}]$, которая совпадает с рангом $\mathscr{C}$. Алгебра смежности коммутативной когерентной конфигурации ранга $r$ изоморфна $\mathbb{C}^r$, так как в этом случае $d_i=1$ для любого $i$.

Структурный тензор $\mathbb{C}[\mathscr{C}]$ будет следующим 
\[
	\sum_{i,j,k \in [r]} p_{i,j}^k \widehat{x}_i \widehat{y}_j \widehat{z}_k,
\]
но (как вскоре увидим) часто удобнее использовать
\[
	\sum_{i,j,k \in [r]} p_{i,j}^{k^*} \widehat{x}_i \widehat{y}_j \widehat{z}_k.
\]
Этот изоморфный тензор эквивалентен простому переупорядочиванию переменных $\widehat{z}_k$. Обратите внимание, что ранг $r$ когерентной конфигурации не обязательно будет таким же как ранг структурного тензора: они равны тогда и только тогда, когда конфигурация коммутативна.

\subsection{Вложение матричного умножения в алгебру смежности}\label{ssub:12:4.4}

Пусть $\mathscr{C}$ --- когерентная конфигурация ранга $r$ с обозначениями как в предыдущем подразделе.

\begin{definition}\label{def:12:4.4}
     Три класса $i,j,k$ образуют \textbf{треугольник}, если существуют точки $x,y,z$ такие, что $(x,y) \in R_i,\; (y,z) \in R_j$ и $(z,x) \in R_k$.
\end{definition}

В терминах чисел пересечений классы $i,j,k$ образуют треугольник тогда и только тогда, когда $p_{i,j}^{k^*} > 0$. (Обратите внимание, что здесь используется $k^*$ вместо $k$, чтобы изменить порядок $x$ и $z$.) Именно поэтому мы предпочли использовать $k^*$ вместо $k$ в структурном тензоре: иначе циклическая симметрия между $x,y,z$ была бы нарушена.

\begin{definition}\label{def:12:4.5}
     Когерентная конфигурация $\mathscr{C}$ ранга $r$ \textbf{реализует} $\left\langle \ell,m,n \right\rangle$, если существуют три инъективные функции
     \[
     	\alpha: [\ell] \times [m] \to [r], \; \beta: [m] \times [n] \to [r], \; \gamma: [n] \times [\ell] \to [r]
     \]
     такие, что $\alpha(a,b'), \beta(b,c'), \gamma(c,a')$ образуют треугольник тогда и только тогда, когда $a=a', b=b'$ и $c=c'$.
\end{definition}

Конечно, это определение подразумевает, что нумерация классов $\mathscr{C}$ зафиксирована. Это эквивалентно общему определению реализации в алгебре, приспособленному к нашему выбору структурного тензора.

\begin{example}\label{ex:12:4.6}
  В качестве простого примера: пусть $\mathscr{C}$ --- когерентная конфигурация на $n$ точках, для которой любая пара точек определяет отдельный класс. Если мы проиндексируем классы парами точек, то $(a,b'), (b,c')$ и $(c,a')$ образуют треугольник тогда и только тогда, когда $a=a',b=b'$ и $c=c'$, поэтому $\mathbb{C}[\mathscr{C}]$ тривиально реализует $\left\langle n,n,n \right\rangle$. Как можно было ожидать от такого тривиального примера, это вложение не даёт никакого преимущества для матричного умножения потому, что на самом деле $\mathbb{C}[\mathscr{C}] \cong \mathbb{C}^{n \times n}$.
\end{example}

В следующем разделе будут построены менее тривиальные примеры. В то же время, упомянем следующее утверждение, которое определяет условия достаточные для реализации матричного умножения в когерентных конфигурациях, возникающих из групповых действий.

\begin{prop}\label{prop:12:4.7}
	Пусть $G$ --- конечная группа, действующая на множестве $X$, и пусть $\mathscr{C}$ --- соответствующая шурова когерентная конфигурация. Предположим, что существуют подмножества $A,B,C \subseteq X$ такие, что для всех $f,g,h \in G$, для которых $fgh=1$ и всех $a \in A,\; b \in B$ и $c \in C$
	\begin{center}
	  если $f \cdot a \in A,\; g \cdot b \in B$ и $h \cdot c \in C$, то $f \cdot a = a,\; g \cdot b = b$ и $h \cdot c = c$.
	\end{center}    
	Тогда $\mathscr{C}$ реализует $\left\langle |A|, |B|, |C| \right\rangle$.
\end{prop}
\begin{proof}
  Напомню, что классы $\mathscr{C}$ --- это орбиты $G$ на $X^2$. Если отождествить $A$ с $[|A|]$ и так далее, то $\left\langle |A|, |B|, |C| \right\rangle$ реализуется через отображения $\alpha, \beta, \gamma$ такие, что для $a \in A$ и $b' \in B$, $\alpha(a,b')$ является орбитой $(a,b')$ в $X^2$ и так далее. Мы хотим показать, что классы $(a,b'),(b,c')$ и $(c,a')$ образуют треугольник тогда и только тогда, когда $a=a',\; b=b'$ и $c=c'$ (где $a,a' \in A,\; b,b' \in B$ и $c,c' \in C$). Говоря, что они образуют треугольник, имеем в виду, что существуют $x,y,z \in X$ и $s,t,u \in G$ такие, что $(x,y) = (s \cdot a,s \cdot b'), (y,z) = (t \cdot b,t \cdot c')$ и $(z,x) = (u \cdot c,u \cdot a')$. Если положим $f=u^{-1}s,\; g=s^{-1}t$ и $h=t^{-1}u$, то $fgh=1$ и $f \cdot a=a'$, так как 
\[
	f \cdot a = (u^{-1} s) \cdot a = u^{-1} \cdot (s \cdot a) = u^{-1} \cdot x = u^{-1} \cdot (u \cdot a') = (u^{-1} u) \cdot a' = 1 \cdot a' = a',
\] 
аналогично $g \cdot b=b'$ и $h \cdot c=c'$. В условии утверждения сказано, что если $f \cdot a = a'\in A$, то $f \cdot a = a$, откуда имеем $a = a'$, аналогично получим $b=b'$ и $c=c'$, что и требовалось.
\end{proof}

Если позволим $G$ действовать на себя левыми сдвигами, то условие из утверждения \ref{prop:12:4.7} просто говорит, что $A,B$ и $C$ удовлетворяют свойству тройного произведения. Таким образом, это утверждение даёт естественное обобщение свойства тройного произведения из групп на групповые действия.


























