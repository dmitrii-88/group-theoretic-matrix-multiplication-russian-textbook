Задача умножения матриц является одной из самых фундаментальных проблем в алгоритмической линейной алгебре. Матричное умножение важно само по себе, но оно становится ещё важнее из-за множества задач, которые к нему сводятся.
 
В 1969 Штрассен \cite{Strassen:1969} создал \hyperref[Strassen_algorithm]{алгоритм} для умножения $n \times n$ матриц, который требовал $O(n^{2.81})$ операций. За этим открытием последовала серия улучшений, которые смогли дать лучшие оценки экспоненты матричного умножения. Копперсмит и Виноград \cite{Coppersmith:1990} в 1990 показали, что $\omega \leq 2.376$. Небольшим улучшением их метода стала работа Вильямс в 2011 \cite{Williams:2011}, которая показала, что $\omega \leq 2.3727$, что является наилучшей оценкой сверху на данный момент. 

Лучшей нижней оценкой является $2 \leq \omega$, её можно получить, если заметить, что для матричного произведения необходимо вычислить $n^2$ элементов. Многие исследователи верят, что эту нижнюю оценку можно достичь, то есть $\omega = 2$.

Недавно американские математики Кон и Уманс \cite{Cohn03} предложили новый теоретико-групповой подход для создания алгоритмов матричного умножения. В их схеме выбирается конечная группа $G$, удовлетворяющая определённому свойству, которое позволяет свести умножение $n \times n$ матриц к умножению элементов из \hyperref[group_algebra]{групповой алгебры} $\mathbb{C}[G]$. Это последнее умножение производится через преобразование Фурье, которое сводит его к нескольким меньшим матричным умножениям, чьи размеры зависят от \hyperref[def:characters_degree]{степеней характеров} $G$. Данное построение естественным образом даёт рекурсивный алгоритм, чьё время выполнения зависит от степеней характеров. Таким образом, проблема создания алгоритма матричного умножения в этой схеме переводится в область теории групп и \hyperref[representation]{теории представлений}.

