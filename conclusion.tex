На данный момент не существует практических приложений каких-либо быстрых алгоритмов матричного умножения кроме алгоритма Штрассена. Алгоритм Коппер\-смита-Винограда и его потомки (Стотерс, Вильямс) являются очень сложными, зависящими от вероятностных конструкций, и так далее. С другой стороны не существует никаких препятствий для их реализации, но в этом очень мало смысла потому, что, во-первых, это сложно, а, во-вторых, точка, за которой они становятся лучше наивного $O(n^3)$-алгоритма, находится очень далеко (поэтому на самом деле никакого улучшения вы никогда не увидите). Существуют другие алгоритмы, которые будет проще реализовать, правда за простоту придётся расплачиваться худшей асимптотической производительностью, но и они также совершенно непрактичны.

Существует и более глубокая проблема, когда вы пытаетесь использовать алгебраические алгоритмы на практике. Модель алгебраической сложности, обычно используемая в этих задачах, считает только арифметические операции и рассматривает доступ к памяти бесплатным. Раньше это имело смысл, так как операции с плавающей точкой были сравнительно дорогими, но сегодня управление памятью может стать настоящим узким местом. Алгебраическая сложность красива и теоретически важна, но она игнорирует важные практические проблемы.

