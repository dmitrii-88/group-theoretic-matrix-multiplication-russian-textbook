\section{Подсолнечники}

Существуют следствия из различных открытых проблем, которые говорят, что гипотеза \ref{conj:05:3.4} скорее всего неверна, то есть УОР-ёмкость не достигает ${3 \over 2^{2 / 3}}$.
Эти следствия рассматриваются в статье Алона и др. \cite{Alon11} и касаются комбинаторного объекта, называемого подсолнечником.

\begin{definition}
  \textbf{$k$-подсолнечник} --- это коллекция $k$ подмножеств $S_1, \dotsc, S_k$, имеющих одинаковые попарные пересечения. То есть для любого $x \in \bigcup_{i=1}^k S_i$ либо $x$ находится в одном множестве, либо во всех $k$ подмножествах.
\end{definition}

Наиболее интересным вопросом, связанным с подсолнечниками будет следующий: дано семейство $\mathcal{F}$, на сколько большим должно быть семейство чтобы в нём содержался подсолнечник. Эту задачу поставили Эрдёш и Радо в статье 1960 года. В ней они доказали следующую фундаментальную теорему о подсолнечниках:
\begin{theorem}
  Пусть $\mathcal{F}$ --- семейство множеств, каждое с мощностью $s$. Если $|\mathcal{F}| > (k-1)^s s!$, тогда $\mathcal{F}$ содержит $k$-подсолнечник.
\end{theorem}
В той же статье была выдвинута гипотеза, являющаяся одной из самых изучаемых в комбинаторике
\begin{conj}
  Пусть $\mathcal{F}$ --- семейство множеств, каждое с мощностью $s$. Существует константа $c_k$, зависящая только от $k$ такая, что если $|\mathcal{F}| \geq c_k^s$, то $\mathcal{F}$ содержит $k$-подсолнечник.
\end{conj}

Хотя эта гипотеза не связана напрямую с УОР-матрицами, Алон и др. \cite{Alon11} представили серию очень сложных её обобщений, чтобы описать которые потребовалось бы привести всю их статью, поэтому заинтересованным читателям следует обратиться к первоисточнику.
