\section{Использование свойства совместного тройного произведения}

Все групповые построения, с помощью которых авторы статьи доказывают нетривиальные оценки на $\omega$, имеют в своей основе свойство совместного тройного произведения в абелевой группе. Каждое построение также включает сплетение, как объясняется в разделе \ref{wreath_construction}, это будет основным инструментом для работы со свойством совместного тройного произведения. Так как теорема \ref{th:05:5.5} может быть доказана либо через построение со сплетением из раздела \ref{wreath_construction}, либо при помощи асимптотического неравенства для сумм, то выходит, что можно обойтись без неабелевых групп. В этом разделе объясняется, как встречавшиеся ранее конструкции будут выглядеть с новых позиций.

\subsection{Локально усиленные ОР-матрицы}

\begin{definition}
  \label{def:LSUSP} \textbf{Локально усиленная ОР-матрица} (сокр. ЛУОР-матрица) ширины $k$ --- это подмножество $U \subseteq \left\{ 1,2,3 \right\}^k$ такое, что для любой упорядоченной тройки $(u,v,w) \in U^3$, где $u,v$ и $w$ не все равны, существует $i \in [k]$ такое, что $(u_i, v_i, w_i)$ является элементом из
  \[
  	\left\{ (1,2,1), (1,2,2), (1,1,3), (1,3,3), (2,2,3), (3,2,3) \right\}.
  \]
\end{definition}

\begin{lemma}
  \label{lem:05:6.1} Любая ЛУОР-матрица является УОР-матрицей.
\end{lemma}
\begin{proof}
Пусть $U$ - ЛУОР-матрица, и пусть $\pi_1, \pi_2, \pi_3 \in S_U$. Если $\pi_1, \pi_2$ и $\pi_3$ не все равны, то существует $u \in U$ такой, что $\pi_1(u), \pi_2(u)$ и $\pi_3(u)$ не все равны. Также существует $i \in [k]$ такой, что $((\pi_1(u))_i, (\pi_2(u))_i, (\pi_3(u))_i)$ содержится в $\left\{ (1,2,1), (1,2,2), (1,1,3), (1,3,3), (2,2,3), (3,2,3) \right\}$, и следовательно в точности два из $(\pi_1(u))_i=1, (\pi_2(u))_i=2$ и $(\pi_3(u))_i=3$ выполняются, что и требовалось.
\end{proof}

Причина, по которой используется слово <<локально>>, в том, что ЛУОР-матрицы удовлетворяют условию на любую тройку строк, а не более слабому глобальному условию на перестановки. Преимущество ЛУОР-матриц в том, что они естественным путём приводят к построению, удовлетворяющему свойству совместного тройного произведения:
\begin{theorem}
  \label{th:05:6.2} Пусть $U$ --- ЛУОР-матрица ширины $k$, и для любого $u \in U$ определим подмножества $A_u, B_u, C_u \subseteq Cyc_\ell^k$ следующим образом
  \begin{align*}
  	A_u & = \left\{ x \in Cyc_\ell^k \mid x_j \neq 0 \iff u_j = 1 \right\}, \\
  	B_u & = \left\{ x \in Cyc_\ell^k \mid x_j \neq 0 \iff u_j = 2 \right\}, \\
  	C_u & = \left\{ x \in Cyc_\ell^k \mid x_j \neq 0 \iff u_j = 3 \right\}. 
  \end{align*}
  Тогда тройки $A_u, B_u, C_u$ удовлетворяют свойству совместного тройного произведения.
\end{theorem}

Обратите внимание, что это построение подчёркивает основную идею утверждения \ref{prop:05:3.5}.

\begin{proof}
Положим $u,v,w \in U$ не все равны и 
\[
	a_u - a_v' + b_v - b_w' + c_w - c_u' = 0,
\]
где $a_u \in A_u, a_v' \in A_v, b_v \in B_v, b_w' \in B_w, c_w \in C_w$ и $c_u' \in C_u$. По определению ЛУОР-матрицы существует $i \in [k]$ такой, что $(u_i,v_i,w_i)$ содержится в
\[
  	\left\{ (1,2,1), (1,2,2), (1,1,3), (1,3,3), (2,2,3), (3,2,3) \right\}.
\]
В каждом из этих случаев в точности один из $a_u,a_v',b_v,b_w',c_w,c_u'$ будет ненулевым, а именно $a_v',b_v,a_u,c_u',b_w'$ и $c_w$ соответственно. 
\begin{question}
  Как это понимать, ведь:
  \[
  	\begin{array}{c|c}
  		(u_i,v_i,w_i) & \text{ элементы, у которых в $i$-ой координате стоит ненулевое значение }\\
  		\hline
  		(1,2,1) & a_u, b_v \\
  		(1,2,2) & a_u, b_v, b_w'\\
  		(1,1,3) & a_u, a_v', c_w\\
  		(1,3,3) & a_u, c_w\\
  		(2,2,3) & b_v, c_w\\
  		(3,2,3) & c_u', b_v, c_w
  	\end{array}
  \]  
\end{question}
Таким образом, в каждом из случаев уравнение $a_u+b_v+c_w=a_v'+b_w'+c_u'$ невозможно, поэтому $u=v=w$, что и требовалось. 

Осталось показать, что для каждого $u$ множества $A_u, B_u, C_u$ удовлетворяют свойству совместного тройного произведения, что очевидно, так как эти множества поддерживаются на непересекающихся наборах координат.
\end{proof}

На первый взгляд определение ЛУОР-матрицы гораздо сильнее, чем определение УОР-матрицы. Например, УОР-матрицы, построенные в подразделе \ref{triangle_construction}, не являются ЛУОР-матрицами. Однако оказывается, что любая оценка на $\omega$, которую можно доказать с помощью УОР-матриц, можно также доказать, используя ЛУОР-матрицы:
\begin{prop}
  \label{prop:05:6.3} УОР-ёмкость достигается ЛУОР-матрицами. В частности, пусть дана любая УОР-матрица ширины $k$, тогда существует ЛУОР-матрица размера $|U|!$ и ширины $|U|k$.
\end{prop}
\begin{proof}
Пусть $U$ --- УОР-матрица ширины $k$, зафиксируем произвольный порядок $u_1,u_2,\dotsc,u_{|U|}$ элементов из $U$. Для любого $\pi \in S_U$, пусть $U_\pi \in \left\{ 1,2,3 \right\}^{|U|k}$ будет конкатенацией $\pi(u_1), \pi(u_2), \dotsc, \pi(u_{|U|})$. Тогда множество всех векторов $U_{\pi}$ будет ЛУОР-матрицей: пусть даны три элемента $U_{\pi_1}, U_{\pi_2}, U_{\pi_3}$, где $\pi_1, \pi_2, \pi_3$ не все равны, по определению УОР-матрицы существует $u \in U$ и $i \in [k]$ такие, что в точности два из $(\pi_1(u))_i=1, (\pi_2(u))_i=2$ и $(\pi_3(u))_i=3$ выполняются. Тогда у векторов $U_{\pi_1}, U_{\pi_2}, U_{\pi_3}$ элементы в координате с индексами $u$ и $i$ будут из 
\[
\left\{ (1,2,1), (1,2,2), (1,1,3), (1,3,3), (2,2,3), (3,2,3) \right\},
\] что и требовалось.
\end{proof}

Утверждение \ref{prop:05:6.3} объясняет выбор слова <<ёмкость>>: оптимизация размера ЛУОР-матрицы эквивалентна определению ёмкости Спернера какого-то направленно гиперграфа (смотри \cite{Sim01} для справки о ёмкости Спернера).

\subsection{Свободные от треугольников множества}

Построение из теоремы \ref{th:05:4.3} также легко проинтерпретировать в терминах свойства совместного тройного произведения. Вспомните о построении троек 
\begin{align*}
  \widehat{A}_v & = A_{v_1} \times \left\{ 1 \right\} \times B_{v_3},\\
  \widehat{B}_v & = B_{v_1} \times A_{v_2} \times \left\{ 1 \right\},\\
  \widehat{C}_v & = \left\{ 1 \right\} \times B_{v_2} \times A_{v_3}
\end{align*}
с индексами $v \in \Delta_n$, определённых перед теоремой \ref{th:05:4.3}. Эти тройки почти удовлетворяют свойству совместного тройного произведения в следующем смысле: если
\[
	\widehat{a}_u (\widehat{a}_v')^{-1} \widehat{b}_v (\widehat{b}_w')^{-1} \widehat{c}_w (\widehat{c}_u')^{-1} = 1,
\]
то, вспомнив о том, что $\widehat{a}_{u} = (a_{u_1}, 1, b_{u_3}), \; \widehat{b}_v = (b_{v_1}, a_{v_2}, 1), \; \widehat{c}_w = (1, b_{w_2}, a_{w_3})$ и так далее, получим
\begin{empheq}[left=\empheqlbrace]{align*}
     a_{u_1}(a_{v_1}')^{-1}b_{v_1}(b_{w_1}')^{-1} & = 1\\
     a_{v_2}(a_{w_2}')^{-1}b_{w_2}(b_{u_2}')^{-1} & = 1\\
     b_{u_3}(b_{v_3}')^{-1}a_{w_3}(a_{u_3}')^{-1} & = 1.
\end{empheq}
По свойству совместного двойного произведения из $a_i (a_j')^{-1} b_j (b_k')^{-1} = 1$ следует, что $i=k$. Поэтому $u_1 = w_1, v_2 = u_2, w_3 = v_3$. 
\begin{definition}\label{def:triangle-free}
Назовём подмножество $S$ множества $\Delta_n$ \textbf{свободным от треугольников}, если для всех $u,v,w \in S$, удовлетворяющих $u_1 = w_1, v_2 = u_2$ и $w_3 = v_3$, будет следовать, что $u=v=w$.    
\end{definition}
Таким образом, тройки $\widehat{A}_v, \widehat{B}_v, \widehat{C}_v$ с $v$ из свободного от треугольников подмножества $\Delta_n$ удовлетворяют свойству совместного тройного произведения.

Главным вопросом будет существование свободного от треугольников подмножества $\Delta_n$ размера $|\Delta_n|^{1-o(1)}$. Приведём простое построение, использующее множества Салема-Спенcера (смотри \cite{Salem}), позволяющее этого добиться. Пусть $T$ подмножество $[\lfloor n/2 \rfloor]$ размера $n^{1-o(1)}$, которое не содержит три последовательных члена арифметической прогрессии. Легко можно доказать следующую лемму:
\begin{lemma}
  \label{lem:05:6.4} Подмножество $\left\{ (a,b,c) \in \Delta_n \mid b-a \in T \right\}$ будет свободным от треугольников и иметь размер $|\Delta_n|^{1-o(1)}$.
\end{lemma}

\subsection{Локальные ОР-матрицы и обобщения}\label{ssub:05:6.3}

Так же как УОР-матрицы обычные ОР-матрицы имеют локальную версию. \textbf{Локальная ОР-матрица} определяется аналогично ЛУОР-матрице, за исключением того, что $(1,2,3)$ добавляется к тройкам $(1,2,1), (1,2,2), (1,1,3), (1,3,3), (2,2,3)$ и $(3,2,3)$. ЛОР-матрицы будут ОР-матрицами, и они будут достигать ОР-ёмкости: доказательство аналогично доказательствам леммы \ref{lem:05:6.1} и утверждения \ref{prop:05:6.3}. Далее описывается, как поместить это построение в более широкий контекст:

\begin{definition}
  \label{def:05:6.5} Пусть $H$ --- конечная абелева группа. \textbf{$H$-схема} (англ. $H$-chart) $\mathcal{C}=(\Gamma, A,B,C)$ состоит из конечного множества символов $\Gamma$, вместе с тремя отображениями $A,B,C: \Gamma \to 2^H$ такими, что для любого $x \in \Gamma$, множества $A(x), B(x), C(x)$ удовлетворяют свойству тройного произведения. Пусть $\mathcal{H}(\mathcal{C}) \subseteq \Gamma^3$ обозначает множество упорядоченных троек $(x,y,z)$ таких, что 
  \[
  	0 \notin A(x) - A(y) + B(y) - B(z) + C(z) - C(x).
  \]
\textbf{Локальная $\mathcal{C}$-ОР-матрица} ширины $k$ --- это подмножество $U \subseteq \Gamma^k$ такое, что для всех упорядоченных троек $(u,v,w) \in U^3$, где $u,v,w$ не все равны, существует $i \in [k]$ такое, что $(u_i, v_i, w_i) \in \mathcal{H}(\mathcal{C})$.
\end{definition}

Например, ЛОР-матрица является $\mathcal{C}$-ОР-матрицей для $Cyc_\ell$-схемы $\mathcal{C}=(\left\{ 1,2,3 \right\}, A, B, C)$, где $A, B, C$ определены следующим образом (ниже $\widehat{H} = Cyc_\ell \setminus \left\{ 0,1 \right\}$)
\begin{align*}
  	A(1) & = \left\{ 0 \right\} & B(1) & = -\widehat{H} & C(1) & = \left\{ 0 \right\}\\
  	A(2) & = \left\{ 1 \right\} & B(2) & = \left\{ 0 \right\} & C(2) & = \widehat{H}\\
  	A(3) & = \widehat{H} & B(3) & = \left\{ 0 \right\} & C(3) & = \left\{ 0 \right\}
\end{align*}

\begin{theorem}
  \label{th:05:6.6} Пусть $H$ --- конечная абелева группа, $\mathcal{C}$ --- $H$-схема, и $U$ --- локальная $\mathcal{C}$-ОР-матрица ширины $k$. Для любого $u \in U$ определим подмножества $A_u, B_u, C_u \subseteq H^k$ так
  \begin{align*}
  	A_u & = \prod_{i=1}^k A(u_i), & B_u & = \prod_{i=1}^k B(u_i), & C_u & = \prod_{i=1}^k C(u_i).
  \end{align*}
  Тогда эти тройки подмножеств удовлетворяют свойству совместного тройного произведения.
\end{theorem}

Вместе с примером, приведённым выше, эта теорема даст аналог теоремы \ref{th:05:6.2} для локальных ОР-матриц. Используя теорему \ref{th:05:3.3}, этот пример покажет, что $\omega < 2.41$.

Используя более сложную схему с 24 символами, из теоремы \ref{th:05:6.6} можно вывести оценку $\omega < 2.376$, аналогичную оценке из работы \cite{Coppersmith:1990}.






















