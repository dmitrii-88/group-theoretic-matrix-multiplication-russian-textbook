\section{Свойство совместного двойного произведения}

Существует по меньшей мере два естественных направления для улучшения конструкции из раздела \ref{triangle_construction}. В комбинаторном направлении можно надеяться заменить УОР-матрицу из утверждения \ref{prop:05:3.8} на большую, что позволит достичь экспоненты 2, если гипотеза \ref{conj:05:3.4} верна. В алгебраическом направлении, можно оставить комбинаторную треугольную конструкцию на месте, а вместо этого модифицировать лежащую в основе группу. Такая модификация может быть произведена при помощи свойства совместного двойного произведения, которое приведено ниже, и есть основания полагать, что этим способом также можно достичь $\omega=2$ (гипотеза \ref{conj:05:4.7}).

\begin{definition}
 Будем говорить, что подмножества $S_1, S_2$ группы $H$ удовлетворяют \textbf{свойству двойного произведения} (англ. double product property), если
 \[
 	q_1 q_2 = 1 \implies q_1 = q_2 = 1,
 \]  
 где $q_i \in Q(S_i)$.
\end{definition}

\begin{definition}\label{def:05:4.1}
  Будем говорить, что $n$ пар подмножеств $A_i, B_i$ (для $1 \leq i \leq n$) группы $H$ удовлетворяют \textbf{свойству совместного двойного произведения} (англ. simultaneous double product property), если
  \begin{itemize}
    \item для всех $i$ пары $A_i, B_i$ удовлетворяют свойству двойного произведения и 
    \item для всех $i,j,k$ 
    \[
    	a_i (a_j')^{-1} b_j (b_k')^{-1} = 1 \text{ означает } i=k,
    \]
    где $a_i \in A_i, a_j' \in A_j, b_j \in B_j, b_k' \in B_k$.
  \end{itemize}
\end{definition}

Можно переформулировать последнее условие так: 

\textit{если рассматривать множества
\[
	A_i^{-1} B_j = \left\{ a^{-1}b \mid a \in A_i, b \in B_j \right\}
\]
те из них, у которых $i=j$, не пересекаются с теми, у которых $i \neq j$.}

Тривиальным примером будет $H = Cyc_n^k \times Cyc_n$ и множества $A_i = \left\{ (x,i) \mid x \in Cyc_n^k \right\}$ и $B_i = \left\{ (0,i) \right\}$. Тогда пары $A_i, B_i$ для $i \in Cyc_n$ удовлетворяют свойству совместного двойного произведения.

\begin{lemma}\label{lem:05:4.2}
  Если $n$ пар подмножеств $A_i, B_i \subseteq H$ удовлетворяют свойству совместного двойного произведения, и $n'$ пар подмножеств $A_i', B_i' \subseteq H'$ удовлетворяют свойству совместного двойного произведения, тогда $n n'$ пар $A_i \times A_j', B_i \times B_j' \subseteq H \times H'$ также будут ему удовлетворять.
\end{lemma}

Пары $A_i, B_i$, удовлетворяющие свойству совместного двойного произведения в группе $H$, могут быть преобразованы в подмножества, удовлетворяющие свойству тройного произведения. Напомню, что
\[
	\Delta_n = \left\{ (a,b,c) \in \mathbb{Z}^3 \mid a+b+c=n-1 \text{ и } a,b,c \geq 0\right\}.
\]

Пусть дано $n$ пар подмножеств $A_i, B_i$ в $H$  для $0 \leq i \leq n-1$, определим тройки подмножеств в $H^3$ с индексами $v=(v_1,v_2,v_3) \in \Delta_n$ следующим образом:
\begin{align*}
  \widehat{A}_v & = A_{v_1} \times \left\{ 1 \right\} \times B_{v_3}\\
  \widehat{B}_v & = B_{v_1} \times A_{v_2} \times \left\{ 1 \right\}\\
  \widehat{C}_v & = \left\{ 1 \right\} \times B_{v_2} \times A_{v_3}
\end{align*}

\begin{theorem}\label{th:05:4.3}
  Если $n$ пар подмножеств $A_i, B_i \subseteq H$ (с $0 \leq i \leq n-1$) удовлетворяют свойству совместного двойного произведения, то следующие подмножества $S_1,S_2,S_3$ группы $G=(H^3)^{|\Delta_n|} \rtimes S_{\Delta_n} $ удовлетворяют свойству тройного произведения:
  \begin{align*}
    S_1 & = \left\{ \widehat{a} \pi \mid \pi \in S_{\Delta_n},  \widehat{a}_v \in \widehat{A}_v \text{ для всех } v \right\}\\
    S_2 & = \left\{ \widehat{b} \pi \mid \pi \in S_{\Delta_n},  \widehat{b}_v \in \widehat{B}_v \text{ для всех } v \right\}\\
    S_3 & = \left\{ \widehat{c} \pi \mid \pi \in S_{\Delta_n},  \widehat{c}_v \in \widehat{C}_v \text{ для всех } v \right\}
  \end{align*}
\end{theorem}

Увы, авторы статьи не привели доказательства этой теоремы, известно лишь, что в нём используется теорема \ref{th:05:1.7}, и оно подобно доказательству утверждения \ref{prop:05:3.5}.

\begin{theorem}\label{th:05:4.4}
  Если $H$ --- конечная группа со степенями характеров $\left\{ d_k \right\}$ и парами подмножеств $A_i, B_i \subseteq H$, удовлетворяющих свойству совместного двойного произведения, то
  \[
  	\sum\limits_{i=1}^{n} (|A_i||B_i|)^{\omega/2} \leq \left( \sum\limits_{k} d_k^\omega  \right)^{3/2}.
  \]
\end{theorem}
\begin{proof}
	Пусть $A_j', B_j'$ --- всевозможные $N$-мерные прямые произведения пар $A_i, B_i$ как в лемме \ref{lem:05:4.2}, и пусть $\mu$ --- случайный $n$-вектор неотрицательных целых чисел, для которых $\sum_{i=1}^{n} \mu_i = N$.

	Среди $A_j', B_j'$ всего $M = \binom{N}{\mu}$ пар, для которых 
	\[
		|A_j'||B_j'| = \prod\limits_{i=1}^n (|A_i||B_i|)^{\mu_i}.
	\]
	Назовём эту величину $L$. Зададим $P:=|\Delta_M|$, откуда $P=M(M+1)/2$. Каждое из трёх подмножеств из теоремы \ref{th:05:4.3} имеет размер $P!L^P$.

	Пусть $\left\{ d_k \right\}$ --- степени характеров $H$, $\left\{ f_k \right\}$ --- $H^3$, $\left\{ c_k \right\}$ --- $(H^3)^{|\Delta_n|} \rtimes S_{\Delta_n}$. Из теоремы \ref{th:05:1.8} получаем, что
	\[
		((P!L^P)^3)^{\omega/3} = (P!L^P)^\omega \leq \sum_i c_i^\omega.
	\]
	из леммы \ref{lem:05:1.2} имеем, что
	\[
		\sum_i c_i^\omega \leq (|\Delta_n|!)^{\omega-1} \left( \sum_k f_k^\omega \right)^{|\Delta_n|} = (P!)^{\omega-1}\left( \sum_k f_k^\omega \right)^{P} \overset{?}{=} (P!)^{\omega-1}\left( \sum_k d_k^\omega \right)^{3NP}.
	\]
	\begin{question}
	  Как выполнить переход от $\left\{ f_k \right\}$ к $\left\{ d_k \right\}$?
	\end{question}
	Итого
	\begin{align*}
		(P!L^P)^{\omega} & \leq (P!)^{\omega-1}\left( \sum_k d_k^\omega \right)^{3NP}\\
		P!L^{P \omega} & \leq \left( \sum_k d_k^\omega \right)^{3NP}.
	\end{align*}
	После извлечения $2P$-ого корня и взятия предела при $N \to \infty$ получим
	\begin{align*}
	  (P!)^{\frac{1}{2P}} L^{\frac{\omega}{2}} & \leq  \left( \sum_k d_k^\omega \right)^{3N/2}\\
	  \dbinom{N}{\mu} \left( \prod\limits_{i=1}^n (|A_i||B_i|)^{\mu_i} \right)^{\frac{\omega}{2}} & \leq \left( \sum_k d_k^\omega \right)^{3N/2}
	\end{align*}
	\begin{question}
	  Почему $\left( P! \right)^{\frac{1}{2P}} = \binom{N}{\mu}$?
	\end{question}
	Наконец, применив лемму \ref{lem:05:1.1} с $s_i = \left( |A_i| |B_i| \right)^{\omega/2}$ и $C=\left( \sum_k d_k^\omega \right)^{3/2}$, получим то, что утверждалось. 
\end{proof}

Удобно использовать два параметра $\alpha$ и $\beta$ для того, чтобы описывать пары, удовлетворяющие свойствам совместного двойного произведения: если существует $n$ пар, выберем $\alpha$ и $\beta$ такими, что $|A_i||B_i| \geq n^\alpha$ для всех $i$ и $|H|=n^\beta$. Если $H$ --- абелева, то теорема \ref{th:05:4.4} будет означать, что $\omega \leq \frac{3\beta - 2}{\alpha}$, так как
\begin{align*}
	\sum_{i=1}^{n} (|A_i||B_i|)^{\omega/2} & \leq \left( \sum_{k} d_k^\omega  \right)^{3/2} \\
	\sum_{i=1}^{n} n^{\alpha \omega/2} & \leq \left( \sum_{k=1}^{|H|} 1  \right)^{3/2}\\
	n^{\alpha \omega/2} \sum_{i=1}^{n} 1 & \leq |H|^{3/2} \\
	n^{\frac{\alpha \omega + 2}{2}}	& \leq n^{\frac{3 \beta}{2}} \\
	\alpha \omega + 2 & \leq 3 \beta \\
	\omega & \leq \frac{3\beta - 2}{\alpha}.
\end{align*}
Во второй строке вывода использовали то, что у абелевой группы все степени характеров будут равны 1.

Наилучшая известная конструкция будет следующей:
\begin{prop}\label{prop:05:4.5}
  Для всех $m \geq 2$ существует конструкция в $Cyc_m^{2 \ell}$, удовлетворяющая свойству совместного двойного произведения, где $\alpha=\log_2(m-1)+o(1)$ и $\beta=\log_2 m+o(1)$ при $\ell \to \infty$.
\end{prop}

Если взять $m=6$, получим ту же оценку, что и в разделе \ref{triangle_construction} ($\omega < 2.48$).

\begin{proof}
	Пусть $n=\binom{2 \ell}{\ell}$. Тогда $n=2^{2\ell(1-o(1))}$, 
	\begin{question}
	  Как вывести $\binom{2 \ell}{\ell} = 2^{2\ell(1-o(1))}$?
	\end{question}
	поэтому $\beta=\log_2 m+o(1)$, так как
	\[
		m^{2\ell} = |Cyc_m^{2\ell}| = |H| = n^\beta=2^{2\ell(1-o(1))(\log_2 m+o(1))}=(2^{\log_2 m})^{2\ell} = m^{2\ell}.
	\]
	\begin{question}
	  Верно ли тут обращаюсь с $o(1)$?
	\end{question}

	Для каждого подмножества $S$ множества $2\ell$ координат группы $Cyc_m^{2\ell}$, для которого $|S|=\ell$, зададим $A_S$ как множество элементов, у которых ненулевые элементы стоят в координатах $S$, а в остальных нули. Через $\overline{S}$ обозначим дополнение множества $S$ и зададим $B_S = A_{\overline{S}}$. Для любого $S$ получим, что $|A_S||B_S|=(m-1)^{2\ell}$, откуда $\alpha = \log_2 (m-1)+o(1)$, так как
	\[
		|A_S||B_S| \geq n^\alpha = 2^{2\ell(1-o(1))(\log_2 (m-1)+o(1))} = (2^{\log_2 (m-1)})^{2\ell} = (m-1)^{2\ell}.
	\]

	Покажем, что пары $A_S, B_S$ удовлетворяют свойству совместного двойного произведения. Ясно, что каждая пара удовлетворяет свойству двойного произведения, потому что элементы $A_S$ и $B_S$ имеют ненулевые элементы на непересекающихся подмножествах координат. Каждый элемент $B_S-A_S$ будет ненулевым во всех координатах, но если $Q \neq R$ тогда будет существовать координата в $\overline{R} \cap Q$, в которой стоит 0 (замечу, что именно поэтому требовалось $|Q|=|R|$). Любой элемент $B_Q-A_R$ имеет ноль в этой координате, поэтому 
	\[
		(B_Q-A_R) \cap (B_S-A_S) = \emptyset
	\]
	что и требовалось доказать. 
\end{proof}

Единственное известное ограничение на возможные значения $\alpha$ и $\beta$ будут следующими:
\begin{prop}\label{prop:05:4.6}
  Если $n$ пар подмножеств $A_i, B_i \subseteq H$ удовлетворяет свойству совместного двойного произведения, где $|A_i||B_i| \geq n^\alpha$ для всех $i$ и $|H|=n^\beta$, то $\alpha \leq \beta$ и $\alpha+2 \leq 2\beta$.
\end{prop}
\begin{proof}
	Свойство двойного произведения означает, что фактор-отображение $(a,b) \mapsto a^{-1}b$ из $A_i \times B_i$ в $H$ инъективно и поэтому $|A_i \times B_i| \leq |H| \implies n^\alpha \leq n^\beta \implies \alpha \leq \beta$.

	Для другого неравенства сперва заметим, что $A_1, \dotsc, A_n$ не пересекаются (если $x \in A_i \cap A_j$, где $i \neq j$ и $y \in B_i$, то $x^{-1}y \in (A_i^{-1}B_i) \cap (A_j^{-1}B_i)$, что невозможно). Точно также можно показать, что $B_1, \dotsc, B_n$ не будут пересекаться. Отсюда следует, что отображение
	\[
		S_n \times S_n \times \prod_{i=1}^n A_i \times \prod_{i=1}^n B_i \to (H^n)^2,
	\]
	определяемое как $(\pi, \rho, a, b) \mapsto (\pi a, \rho b)$, будет инъективным. Здесь группа $S_n$ действует переставляя $n$ координат. Сравнивая размеры этих подмножеств, получим, что 
	\begin{align*}
		(n!)^2 (n^\alpha)^n & \leq (n^\beta)^{2n} \\
		\ln((n!)^2 (n^\alpha)^n) & \leq \ln ((n^\beta)^{2n}) \\
		n \alpha \ln n + 2 \ln(n!) & \leq 2n\beta \ln n \\
		n \alpha \ln n + 2 n \ln n - 2 n + O(\ln n) & \leq 2n\beta \ln n \text{ (используя формулу Стирлинга) }\\
		\alpha + 2 - \frac{2}{\ln n} + \frac{O(\ln n)}{n \ln n} & \leq 2 \beta\\
		\alpha + 2 & \leq 2\beta \text{ при } n \to \infty. \qedhere
	\end{align*}
\end{proof}

Обратите внимание, что если начать брать прямые произведения $H$, используя лемму \ref{lem:05:4.2}, то можно сделать $n$ произвольно большим без изменения $\alpha$ и $\beta$. 

Самым важным случаем будет тот, когда $H$ является абелевой группой. В этом случае оценка на $\omega$ будет $\omega \leq (3\beta-2)/ \alpha$, и утверждение \ref{prop:05:4.6} показывает, что единственным способом достичь $\omega=2$ будет $\alpha=\beta=2$. Есть основания полагать, что это возможно:

\begin{conj}\label{conj:05:4.7}
  Для произвольно большого $n$ существует абелева группа $H$ с $n$ парами подмножеств $A_i, B_i$, удовлетворяющих свойству совместного двойного произведения такими, что $|H|=n^{2+o(1)}$ и $|A_i||B_i| \geq n^{2-o(1)}$.
\end{conj}



















































