\section{Треугольная конструкция}\label{triangle_construction}

У УОР-матриц, построенных в утверждении \ref{prop:05:3.1}, было свойство: каждый столбец содержал только два символа из возможных трёх $\left\{ 1,2,3 \right\}$. Любая ОР-матрица, обладающая этим свойством, будет также УОР-матрицей. Ниже будет рассмотрен вопрос насколько большими могут быть матрицы такого вида.

Пусть $U \subseteq \left\{ 1,2,3 \right\}^k$ --- матрица, у которой каждый столбец содержит только два символа. Через $U_{1,2}$ обозначим подматрицу $U$, образованную всеми столбцами, содержащими только символы 1 и 2. Аналогично определим матрицы $U_{2,3}$ и $U_{1,3}$. Без ограничения общности можно считать, что $U$ получается в результате объединения матриц $U_{1,2}, U_{2,3}$ и $U_{1,3}$, поэтому
\[
	U = ||U_{1,2} U_{2,3} U_{1,3}||.
\]

Через $H_1$ обозначим подгруппу $S_U$, сохраняющую столбцы $U$, содержащие только 1 и 2. Иными словами $H_1$ образована всеми перестановками $\pi$, обладающими свойством:\\
\textit{для любой строки $u \in U$ строка $\pi(u)$ совпадает с $u$ во всех координатах, соответствующим столбцам, содержащим только 1 и 2 (таким образом, если $U = ||U_{1,2} U_{2,3} U_{1,3}||$, то $\pi(U) = ||U_{1,2} \widetilde{U}_{2,3} \widetilde{U}_{1,3}||$).}

Аналогично $H_2$ сохраняет столбцы, содержащие только 2 и 3, а $H_3$ --- 1 и 3.

\begin{lemma}\label{lem:05:3.7}
  Множество $U$ является ОР-матрицей тогда и только тогда, когда $H_1,H_2$ и $H_3$ удовлетворяют свойству тройного произведения внутри $S_U$.
\end{lemma}
\begin{proof}
Пусть $\pi_1, \pi_2, \pi_3 \in S_U$. Перестановка $\pi_1 \pi_2^{-1}$ не содержится в $H_1$ тогда и только тогда, когда существует $v \in U$ и координата $i$ такие, что $v_i = 2$ и $((\pi_1 \pi_2^{-1})(v))_i=1$. Если положим $u=\pi_2^{-1}(v)$, то это эквивалентно тому, что $(\pi_2(u))_i=2$ и $(\pi_1(u))_i=1$. Аналогично выводятся условия для $\pi_2 \pi_3^{-1}$ и $\pi_3 \pi_1^{-1}$. Формально все три условия выглядят следующим образом:
\begin{align}
	\label{eq:tc1}
	\pi_1 \pi_2^{-1} \notin H_1  \iff \exists u, i :  (\pi_1(u))_i=1 \text{ и } (\pi_2(u))_i=2\\
	\label{eq:tc2}
	\pi_2 \pi_3^{-1} \notin H_2  \iff \exists u, i :  (\pi_2(u))_i=2 \text{ и } (\pi_3(u))_i=3\\
	\label{eq:tc3}
	\pi_3 \pi_1^{-1} \notin H_3  \iff \exists u, i :  (\pi_3(u))_i=3 \text{ и } (\pi_1(u))_i=1
\end{align}

\begin{question}
  Верны ли мои доказательства, представленные ниже?
\end{question}
$(\Longrightarrow)$ Пусть $U$ --- ОР-матрица, у которой каждый столбец содержит только два символа, докажем, что тогда $H_1, H_2, H_3$ удовлетворяют свойству тройного произведения в $S_U$.
Пусть $h_i \in H_i$ такие, что 
\[
	h_1 h_2 h_3 = 1.
\]
Любые элементы, удовлетворяющие этому равенству, можно записать в виде $h_1 = \pi_1 \pi_2^{-1}, h_2 = \pi_2 \pi_3^{-1}$ и $h_3 = \pi_3 \pi_1^{-1}$. Раз $U$ является ОР-матрицей, то для любых $\pi_1, \pi_2, \pi_3$ верно, что либо $\pi_1= \pi_2= \pi_3$, откуда $h_1=h_2=h_3=1$, либо существуют $u \in U, i \in [k]$ такие, что по крайней мере два равенства из $(\pi_1(u))_i=1, (\pi_2(u))_i=2, (\pi_3(u))_i=3$ выполняются, откуда получим противоречие, так как выходит, что для какого-то $j$ $h_j \notin H_j$. Подробнее, скажем, пусть верны $(\pi_1(u))_i=1$ и $(\pi_2(u))_i=2$, отсюда из \eqref{eq:tc1} будет следовать, что $\pi_1 \pi_2^{-1} = h_1 \notin H_1$.

$(\Longleftarrow)$ $H_1,H_2,H_3$ удовлетворяют свойству тройного произведения. Докажем, что $U$ --- ОР-матрица. Предположим противное, $U$ не является ОР-матрицей. Из этого будет следовать, что существуют перестановки $\pi_1, \pi_2, \pi_3 \in S_U$, хотя бы две из которых не равны и такие, что только 1 из равенств $(\pi_1(u))_i=1, (\pi_2(u))_i=2, (\pi_3(u))_i=3$ выполняется для всех $u \in U$ и всех $i \in [k]$. Из условий \eqref{eq:tc1}, \eqref{eq:tc2}, \eqref{eq:tc3} следует, что в этом случае $\pi_1 \pi_2^{-1} =: h_1 \in H_1, \pi_2 \pi_3^{-1} =: h_2 \in H_2$ и $\pi_3 \pi_1^{-1} =: h_3 \in H_3$. Исходя из задания элементов $h_1, h_2, h_3$ получим, что $h_1 h_2 h_3 = 1$, и так как $H_1, H_2$ и $H_3$ удовлетворяют свойству тройного произведения имеем, что $h_1=h_2=h_3=1$, следовательно $\pi_1= \pi_2= \pi_3$ --- противоречие с тем, что хотя бы две перестановки из $\pi_1, \pi_2, \pi_3$ не равны между собой. 
\end{proof}

\begin{prop}\label{prop:05:3.8}
  Для любого $k \geq 1$ существует УОР-матрица размера ${2^{k-1}(2^k + 1)}$ и ширины $3k$.
  
  Из этого будет следовать, что УОР-ёмкость не меньше, чем $2^{2/3}$ и $\omega < 2.48$.
\end{prop}
\begin{proof}
Рассмотрим треугольник
\[
	\Delta_n = \left\{ (a,b,c) \in \mathbb{Z}^3 \mid a+b+c=n-1 \text{ и } a,b,c \geq 0\right\},
\]
где $n=2^k$, и пусть $H_1, H_2$ и $H_3$ являются подгруппами $S_{\Delta_n}$, сохраняющими первую, вторую и третью координаты у элементов $(a,b,c)$ соответственно. По теореме \ref{th:05:1.7} эти подгруппы удовлетворяют свойству тройного произведения в $S_{\Delta_n}$.

Чтобы построить требуемую УОР-матрицу, выберем подмножество $U \subseteq \left\{ 1,2,3 \right\}^{3k}$ следующим образом. В первых $k$ координатах будут встречаться только 1 и 2, во вторых $k$ координатах только 2 и 3, в третьих --- только 1 и 3. В каждом из этих блоков по $k$ координат существует $2^k$ возможных способов расставить два доступных символа. Пронумеруем эти способы числами от 0 до $2^k-1$ (причём номер получает сам способ чередования двух символов, то есть, какие именно символы чередуются, значения не имеет. Например, при $k=3$ блоки $\begin{array}{|c|c|c|} \hline 2 & 1 & 1 \\ \hline\end{array}, \; \begin{array}{|c|c|c|} \hline 3 & 2 & 2 \\ \hline\end{array}$ и $\begin{array}{|c|c|c|} \hline 3 & 1 & 1 \\ \hline\end{array}$ имеют один и тот же номер.) Элементы из $U$ будут соответствовать элементам из $\Delta_n$ следующим образом: $u \in U$ соответствует $(a,b,c) \in \Delta_n$, если в первых $k$ координатах 2 символа чередуются способом под номером $a$, во вторых --- способом $b$, а в третьих --- способом $c$. Из леммы \ref{lem:05:3.7} следует, что $U$ будет УОР-матрицей. Всего существует $n(n+1)/2$ троек $(a,b,c)$, где $a,b,c \geq 0$ и $a + b + c = n-1$, поэтому  
\[
	|U|=|\Delta_n| = \frac{n(n+1)}{2} = \frac{2^k (2^k + 1)}{2} = 2^{k-1}(2^k + 1).\qedhere
\]
\end{proof}

Можно показать, используя лемму \ref{lem:05:3.7}, что это построение оптимально:
\begin{corollary}
  Если $U$ --- ОР-матрица ширины $k$ такая, что только 2 символа встречаются в каждой координате, то $|U| \leq (2^{2/3} + o(1))^k$.
\end{corollary}

Условие о том, что в каждой координате встречаются только 2 из 3 символов сильно ограничивает выбор, но улучшить утверждение \ref{prop:05:3.8} пока не удалось. Однако также не известно верхних оценок на размер УОР-матриц кроме леммы \ref{lem:05:3.2}. 





































