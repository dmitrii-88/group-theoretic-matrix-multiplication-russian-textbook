\section{Свойство совместного тройного произведения}

Каждое из групповых построений, дающих нетривиальные оценки на $\omega$, имеет один и тот же вид, а именно является полупрямым произведением группы перестановок и абелевой группы. Ключевая часть таких построений --- это то, как абелев фактор распределён между подмножествами, удовлетворяющими свойству тройного произведения.

Это распределение можно рассматривать как сведение нескольких независимых задач матричного умножения к единственному умножению в групповой алгебре, используя тройки подмножеств, удовлетворяющих свойству совместного тройного произведения:

\begin{definition}\label{def:05:5.1}
  Будем говорить, что подмножества $A_i, B_i, C_i$ (для $1 \leq i \leq n$) группы $H$ удовлетворяют \textbf{свойству совместного тройного произведения}, если
  \begin{itemize}
    \item для всех $i$ тройки $A_i, B_i, C_i$ удовлетворяют свойству тройного произведения 
    \item для всех $i,j,k$ 
    \[
    	a_i (a_j')^{-1} b_j (b_k')^{-1} c_k (c_i')^{-1} = 1 \text{ означает } i=j=k,
    \]
    где $a_i \in A_i, a_j' \in A_j, b_j \in B_j, b_k' \in B_k, c_k \in C_k$ и $c_i' \in C_i$.
  \end{itemize}  
\end{definition}
Будем говорить, что такая группа одновременно реализует 
\[
	\left\langle |A_1|, |B_1|,|C_1| \right\rangle, \dotsc, \left\langle |A_n|, |B_n|,|C_n| \right\rangle.
\]

В большинстве приложений группа $H$ будет абелевой, в этом случае удобно пользоваться аддитивной записью, тогда второе из условий в определении принимает вид
\[
	a_i-a_j'+b_j-b_k'+c_k-c_i'=0 \text{ влечёт } i=j=k.
\]

Например, пусть $H=Cyc_n^3$, назовём три её фактора как $H_1, H_2, H_3$. Определим множества
\[
	A_1=H_1 \setminus \left\{ 0 \right\},\; B_1=H_2 \setminus \left\{ 0 \right\}, \; C_1=H_3 \setminus \left\{ 0 \right\}
\]
и
\[
	A_2=H_2 \setminus \left\{ 0 \right\},\; B_2=H_3 \setminus \left\{ 0 \right\}, \; C_2=H_1 \setminus \left\{ 0 \right\}.
\]

\begin{prop}\label{prop:05:5.2}
  Две тройки $A_1,B_1,C_1$ и $A_2,B_2,C_2$ удовлетворяют свойству совместного тройного произведения.
\end{prop}
\begin{proof}
	Очевидно, что каждая тройка по отдельности удовлетворяет свойству тройного произведения, поэтому необходимо доказать только второе условие из определения. Для $i \in \left\{ 1,2 \right\}$ обозначим $U_i=A_i-C_i,\; V_i=B_i-A_i$ и $W_i=C_i-B_i$. Нужно доказать, что если $u_i+v_j+w_k=0$, где $u_i \in U_i,\; v_j \in V_j$ и $w_k \in W_k$, то $i=j=k$.

	Имеем 
	\[
		\begin{array}{lclcl}
			U_1 & = & W_2 & = & \left\{ (x,0,z) \in Cyc_n^3 \mid x \neq 0, z \neq 0 \right\}, \\
			V_1 & = & U_2 & = & \left\{ (x,y,0) \in Cyc_n^3 \mid x \neq 0, y \neq 0 \right\}, 
		\end{array}
	\]
	и
	\[
		\begin{array}{lclcl}
			W_1 & = & V_2 & = & \left\{ (0,y,z) \in Cyc_n^3 \mid y \neq 0, z \neq 0 \right\}.
		\end{array}
	\]
	Если $i,j$ и $k$ не равны между собой, то два из них должны быть равны, но отличаться от третьего. В любом случае $U_i, V_j$ и $W_k$ заключают в себе в точности два из трёх подмножеств $Cyc_n^3$, определённых в уравнениях выше, причём одно из двух подмножеств встречается дважды, поэтому сумма $u_i+v_j+w_k$ не может обратиться в ноль, так как у двух элементов в одной из координат будет стоять ноль, в то время как у третьего элемента в этой координате будет всегда ненулевое значение. 
\end{proof}

Причина, по которой в определении свойства совместного тройного произведения есть такое странное условие в том, что именно оно необходимо для того, чтобы свести несколько независимых матричных умножений к одному умножению в групповой алгебре.

\begin{theorem}
  \label{th:05:5.3} Пусть $R$ --- произвольная алгебра над $\mathbb{C}$. Если $H$ одновременно реализует $\left\langle n_1,m_1,p_1  \right\rangle , \dotsc, \left\langle n_k,m_k,p_k \right\rangle$, тогда число операций в кольце, необходимых для выполнения $k$ независимых матричных умножений размеров $n_1 \times m_1$ на $m_1 \times p_1, \dotsc, n_k \times m_k$ на $m_k \times p_k$, не превосходит числа операций необходимых для умножения двух элементов из $R[H]$.
\end{theorem}
  
Доказательство этой теоремы подобно доказательству теоремы \ref{th:03:2.3}.

Другие результаты, полученные для свойства тройного произведения, также легко обобщаются для свойства совместного тройного произведения, например, лемма:
\begin{lemma}
  \label{lem:05:5.4} Если $n$ троек подмножеств $A_i, B_i, C_i \subseteq H$ удовлетворяет свойству совместного тройного произведения, и $n'$ троек $A_j', B_j', C_j' \subseteq H'$ удовлетворяет свойству совместного тройного произведения, тогда $n n'$ троек подмножеств $A_i \times A_j', B_i \times B_j', C_i \times C_j' \subseteq H \times H'$ также будут ему удовлетворять.
\end{lemma}

Используя асимптотическое неравенство для сумм Шёнхаге \eqref{eq:12:2.1}, можно получить оценку на $\omega$ из свойства совместного тройного произведения:
\begin{theorem}
  \label{th:05:5.5} Если группа $H$ одновременно реализует $\left\langle a_1,b_1,c_1 \right\rangle, \dotsc, \left\langle a_n,b_n,c_n \right\rangle$ и имеет степени характеров $\left\{ d_k \right\}$, то
  \[
  	\sum_{i=1}^n (a_i b_i c_i)^{\omega/3} \leq \sum_k d_k^\omega.
  \]
\end{theorem}

Часто $H$ будет абелевой, в этом случае $\sum_k d_k^\omega=|H|$. Так было в примере из утверждения \ref{prop:05:5.2}, которое доказывает $\omega < 2.93$, если применить теорему \ref{th:05:5.5}. В разделе \ref{wreath_construction} будет показано, что любая оценка на $\omega$, которую можно доказать, используя свойство совместного тройного произведения, можно также доказать, используя свойство обычного тройного произведения. Таким образом, свойство совместного тройного произведения не добавляет общности, зато является важным организующим принципом.




























