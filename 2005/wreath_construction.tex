\section{Построение, использующее сплетение}\label{wreath_construction}

Осталось доказать теорему \ref{th:05:5.5}, используя только теоретико-групповые средства. Кроме самого доказательства покажем также, что обычное свойство тройного произведения из определения \ref{def:tpp} настолько же сильно насколько свойство совместного тройного произведения в том смысле, что оценки, полученные из теоремы \ref{th:05:5.5}, можно также получить, используя теорему \ref{th:05:1.8}.

Для доказательства теоремы \ref{th:05:5.5} воспользуемся построением со сплетением. Пусть $H$ --- группа, определим $G = S_n \ltimes H^n$, где $S_n$ действует на $H^n$ справа, переставляя соответствующим образом координаты $(h^\pi)_i = h_{\pi(i)}$. Запишем элементы $G$ как $h \pi$, где $h \in H^n$ и $\pi \in S_n$.
\begin{question}
  Почему элемент $G = S_n \ltimes H^n$ записывается как $h \pi$, а не $\pi h$, хотя так полагается? Тогда получается, что операция в $G$ выглядит как:
  \[
  	h_1 \pi_1 h_2 \pi_2 = h_1^{\pi_2} h_2 \pi_1 \pi_2?
  \]
\end{question}

\begin{theorem}
  \label{th:05:7.1} Если $n$ троек подмножеств $A_i, B_i, C_i \subseteq H$ удовлетворяют свойству совместного тройного произведения, то следующие подмножества $H_1, H_2, H_3$ группы $G = S_n \ltimes H^n$ удовлетворяют свойству тройного произведения.
  \begin{align*}
    H_1 & = \left\{ h \pi \mid \pi \in S_n, h_i \in A_i \text{ для всех } i\right\}\\
    H_2 & = \left\{ h \pi \mid \pi \in S_n, h_i \in B_i \text{ для всех } i\right\}\\
    H_3 & = \left\{ h \pi \mid \pi \in S_n, h_i \in C_i \text{ для всех } i\right\}
  \end{align*}
\end{theorem}
\begin{proof}
  Оно аналогично доказательству утверждения \ref{prop:05:3.5}. Рассмотрим тройное произведение 
  \begin{equation}\label{eq:7.1}
    h_1 \pi_1 \pi_1'^{-1} h_1'^{-1}  h_2 \pi_2 \pi_2'^{-1} h_2'^{-1}  h_3 \pi_3 \pi_3'^{-1} h_3'^{-1}  = 1,
  \end{equation}
  где $h_i \pi_i, h_i' \pi_i' \in H_i$. (Замечу, что $h_1, h_2, h_3$ --- это различные элементы $H^n$, а не координаты одного элемента $h \in H^n$. Как только вы это поймёте, путаницы возникнуть не должно.) Чтобы \eqref{eq:7.1} выполнялось нужно 
  \begin{equation}\label{eq:7.2}
    \pi_1 \pi_1'^{-1}  \pi_2 \pi_2'^{-1}  \pi_3 \pi_3'^{-1} = 1.
  \end{equation}
  Присвоим $\pi = \pi_1 \pi_1'^{-1}$ и $\rho = \pi_1 \pi_1'^{-1} \pi_2 \pi_2'^{-1}$. Тогда для того чтобы выполнялось \eqref{eq:7.1}, нужно
  \[
  	h_3'^{-1} h_1 (h_1'^{-1} h_2)^\pi (h_2'^{-1} h_3)^\rho = 1,
  \]
  Другими словами для каждой координаты $i$
  \[
  	(h_3'^{-1})_i (h_1)_i (h_1'^{-1})_{\pi(i)} (h_2)_{\pi(i)} (h_2'^{-1})_{\rho(i)} (h_3)_{\rho(i)} = 1.
  \]
  
  Из свойства совместного тройного произведения получим $\pi(i)=\rho(i)=i$. Таким образом, $\pi=\rho=1$, что вместе с \eqref{eq:7.2} означает $\pi_i = \pi_i'$ для всех $i$. Наконец, получим
  \[
  	h_1 h_1'^{-1} h_2 h_2'^{-1} h_3 h_3'^{-1} = 1,
  \]
  что означает $h_1 = h_1', h_2 = h_2'$ и $h_3 = h_3'$, так как каждая тройка $A_i, B_i, C_i$ удовлетворяет свойству тройного произведения.
\end{proof}

Как первый шаг к доказательству теоремы \ref{th:05:5.5} докажем более слабую оценку, в которой геометрическое среднее заменяет арифметическое среднее:
\begin{lemma}
  \label{lem:05:7.2} Если $H$ --- конечная группа со степенями характеров $\left\{ d_k \right\}$, и есть $n$ троек подмножеств $A_i, B_i, C_i \subseteq H$, удовлетворяющих свойству совместного тройного произведения, тогда
  \[
  	n \left( \prod_i (|A_i| |B_i| |C_i|)^{\omega/3} \right)^{1/n} \leq \sum_k d_k^\omega.
  \]
\end{lemma}
\begin{proof}
  Размеры трёх подмножеств $G$ из теоремы \ref{th:05:7.1} равны $n!\prod_i |A_i|, n!\prod_i |B_i|$ и $n!\prod_i |C_i|$ соответственно. Применив теорему \ref{th:05:1.8}, получим неравенство
  \[
  	\left((n!)^3 \prod_i |A_i| |B_i| |C_i|\right)^{\omega/3} \leq \sum_j c_j^\omega,
  \]
  где $\left\{ c_j \right\}$ --- это степени характеров $G$.  По лемме \ref{lem:05:1.2} правая часть не превосходит $(n!)^{\omega-1} \left( \sum_k d_k^\omega \right)^n$, затем, поделив обе части на $(n!)^\omega$, получим
  \[
  	\left(\prod_i |A_i| |B_i| |C_i|\right)^{\omega/3} \leq (n!)^{-1} \left( \sum_k d_k^\omega \right)^n.	
  \]
  
  Это неравенство немного слабее того, что нам требуется, но это легко исправить, используя прямые степени $H$ с помощью леммы \ref{lem:05:5.4}. Заменив $H$ на $H^t$ (и $n$ на $n^t$), получим
  \[
  	\left(\prod_i |A_i| |B_i| |C_i|\right)^{t n^{t-1} \omega/3} \leq (n^t!)^{-1} \left( \sum_k d_k^\omega \right)^{t n^t}.
  \]
  \begin{question}
    Не понял природу этой замены. 
  \end{question}
  Взяв $t n^t$-ый корень и устремив $t \to \infty$, получим требуемое неравенство, так как
  \[
  	\left(\prod_i (|A_i| |B_i| |C_i|)^{\omega/3}\right)^{1/n} \leq (n^t!)^{-\frac{1}{t n^t}} \left( \sum_k d_k^\omega \right),
  \]
  а 
  \begin{align*}
       \lim_{t \to \infty} (n^t!)^{-\frac{1}{t n^t}} & = \lim_{t \to \infty} \left( \sqrt{2 \pi n} \left( \frac{n^t}{e} \right)^{n^t} \right)^{-\frac{1}{t n^t}}\\
       & = \lim_{t \to \infty} \underbrace{\left(\sqrt{2 \pi n}\right)^{-\frac{1}{t n^t}}}_{\to 1} \left( \frac{n^t}{e} \right)^{-\frac{1}{t}}\\
       & = \lim_{t \to \infty} \frac{e^{\frac{1}{t}}}{n} = \frac{1}{n}. \qedhere
  \end{align*}
  \begin{question}
    Верно ли я подсчитал $\lim_{t \to \infty} (n^t!)^{-\frac{1}{t n^t}}$?
  \end{question}
\end{proof}

\begin{proof}[Доказательство теоремы \ref{th:05:5.5}]
  Пусть $A_j', B_j', C_j'$ --- всевозможные $N$-кратные прямые произведения троек $A_i, B_i, C_i$ (при помощи леммы \ref{lem:05:5.4}), и пусть $\mu$ --- случайный $n$-вектор неотрицательных целых чисел, для которых $\sum_{i=1}^{n} \mu_i = N$. Среди троек $A_j', B_j', C_j'$ всего $\binom{N}{\mu}$ троек, для которых
  \[
  	|A_j'| |B_j'| |C_j'| = \prod_{i=1}^n (|A_i| |B_i| |C_i|)^{\mu_i}.
  \]
  Применив лемму \ref{lem:05:7.2} к этим тройкам получим
  \[
  	\dbinom{N}{\mu} \prod_{i=1}^n (|A_i| |B_i| |C_i|)^{\mu_i \omega/ 3} \leq \left( \sum_k d_k^\omega \right)^N.
  \]
  \begin{question}
    Как из $\binom{N}{\mu} \left( \prod_j (|A_j'| |B_j'| |C_j'|)^{\omega/ 3} \right)^{1/ \binom{N}{\mu}} \leq \sum_k f_k^\omega $, где $\left\{ f_k \right\}$ --- степени характеров $H^N$, смогли получить $\binom{N}{\mu} \prod_{i=1}^n (|A_i| |B_i| |C_i|)^{\mu_i \omega/ 3} \leq \left( \sum_k d_k^\omega \right)^N$?
  \end{question}
  Применив лемму \ref{lem:05:1.1}, где $s_i = (|A_i| |B_i| |C_i|)^{\omega/ 3}$ и $C = \sum_k d_k^\omega$, получим требуемую оценку.
\end{proof}
