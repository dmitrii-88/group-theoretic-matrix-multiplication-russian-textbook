\section{$O$-символика}\label{o_notation}

$O$-большое описывает предельное поведение функции в терминах более простой функции, когда её аргумент стремится к определённому значению или бесконечности. Более формально, пусть $f$ и $g$ --- две функции, определённые на каком-то подмножестве действительных чисел. Тогда
\[
	f(x)=O(g(x)) \mbox{ при } x \rightarrow \infty
\]
тогда и только тогда, когда существует положительная константа $M$ такая, что для достаточно больших значений $x$, $f(x)$ не превосходит $M$, умноженную на $g(x)$, в абсолютных значениях. То есть 
\[
	f(x) = O(g(x)) \iff \exists M > 0 \; \exists x_0: |f(x)| \leq M |g(x)| \mbox{ для любого } x > x_0.
\]

Когда не сказано к чему стремится $x$, будем считать, что он стремится к $\infty$. $O$"~символику можно использовать для описания поведения $f$ вблизи какого-то действительного числа $a$:
\[
	f(x)=O(g(x)) \mbox{ при } x \rightarrow a
\]
тогда и только тогда, когда существуют положительные числа $\delta$ и $M$ такие, что
\[
	|f(x)| \leq M |g(x)| \mbox{ для } |x - a| < \delta.
\]

Рассмотрим небольшой пример. Пусть $f(x) = 4 x^2 - 2 x + 2$ при $x \rightarrow \infty$. В этом выражении $x^2$ будет доминировать, поэтому оставшимися членами можно пренебречь. Коэффициент при $x^2$ можно перенести в константу $M$ из определения $O$-большого. В итоге получим:
\[
	f(x) = O(x^2) \mbox{ или } f(x) \in O(x^2).
\]
Мы будем говорить, что $f$ имеет сложность порядка $x^2$. Обратите внимание, что <<=>> не означает равенства в математическом смысле, а скорее эквивалентно слову <<является>>. Поэтому второй способ технически точнее, тогда как первый --- это распространённое злоупотребление обозначениями.

Основание логарифма внутри символа $O()$ не пишется. Причина этого весьма проста. Пусть у нас есть $O(\log_2 n)$. Но $\log_2 n = \log_3 n / \log_3 2$, а $\log_3 2$, как и любую константу, асимптотика не учитывает. Таким образом, $O(\log_2 n) = O(\log_3 n)$. К любому основанию мы можем перейти аналогично, а значит и писать его не имеет смысла.

Далее нам потребуется также понятие $o$-малого. \phantomsection \label{little-oh} Отношение $f(x) = o(g(x))$  читается как "$f(x)$ является $o$-малым от $g(x)$". Интуитивно это означает, что $g$ растёт гораздо быстрее чем $f$, или по-другому --- рост $f$ ничто в сравнении с $g$. Формально, 
\[
	f(x) = o(g(x)) \mbox{ при }  x \rightarrow \infty
\] 
означает, что для любого положительного $\varepsilon$ существует $x_0$ такое, что
\[
	|f(x)| \leq \varepsilon |g(x)| \mbox{ для всех } x \geq x_0.
\]

В определении $O$-большого существовала хотя бы одна константа $M$, здесь же неравенство верно для любой сколь угодно малой положительной константы  $\varepsilon$. Поэтому то, что функция является $o$-малым от $g$, означает также, что она будет $O$-большим от $g$. 

В тексте часто можно встретить выражение $o(1)$. Примером функции, являющейся $o$-малым от $1$, будет $1/x$ при $x \rightarrow \infty$.
