\section{Групповые кольца и алгебры}\label{group_algebra}

Зафиксируем коммутативное кольцо $R$ с единицей $1 \neq 0$ и пусть $G = \{ g_1, g_2, \ldots, g_n\}$ конечная группа с операцией, записываемой мультипликативно. Определим \textbf{групповое кольцо} $R[G]$ группы $G$ с коэффициентами из $R$ как множество формальных сумм вида 
\[
	\sum\limits_{i=1}^{n} \alpha_i g_i, \; \alpha_i \in R.
\]

Две формальные суммы равны тогда и только тогда, когда все соответствующие коэффициенты при групповых элементах равны. Сложение и вычитание в $R[G]$ определяется следующим образом:
\[
 \sum\limits_{i=1}^{n} \alpha_i g_i + \sum\limits_{i=1}^{n} \beta_i g_i = \sum\limits_{i=1}^{n} (\alpha_i + \beta_i) g_i \\
\]
\[  
 \left( \sum\limits_{i=1}^{n} \alpha_i g_i \right) \left(\sum\limits_{i=1}^{n} \beta_i g_i \right) =  
 \sum\limits_{k=1}^{n} \left(\sum_{\substack{i,j\\g_i g_j = g_k}} \alpha_i \beta_j\right) g_k
\]	
где сложение и умножение коэффициентов $\alpha_i, \beta_j$ производится в $R$.

Пусть $F$ --- поле. Напомню, что поле --- это коммутативное кольцо с $1 \neq 0$, в котором любой ненулевой элемент имеет мультипликативный обратный. Рассмотрим групповое кольцо $F[G]$, оно будет коммутативным кольцом тогда и только тогда, когда $G$ является абелевой группой.

Группа $G$ содержится в $F[G]$ (если отоджествить $g_i$ с $1 g_i$), и поле $F$ содержится в $F[G]$ (если отоджествить $\beta$ с $\beta g_1$, где $g_1$ - это единица в $G$). При этих отождествелениях 
\[
	\beta \left( \sum\limits_{i=1}^{n} \alpha_i g_i \right) = \sum\limits_{i=1}^{n} (\beta \alpha_i) g_i, \; \mbox{ для всех } \beta \in F.
\]

Таким образом, $F[G]$ является векторным пространством над $F$ с элементами из $G$ в качестве базиса, поэтому размерность $F[G]$ как векторного пространства будет равна $|G|$. 
Элементы $F$ коммутируют со всеми элементами $F[G]$, то есть $F$ находится в центре кольца $F[G]$. Напомню, что центром произвольного кольца $R$ называется множество $\{ z \in R \mid zr = rz \; \mbox{ для всех } r \in R \}$. Когда хотят подчеркнуть два последних свойства, то говорят, что $F[G]$ --- это $F$-алгебра.

В общем случае \textbf{$F$-алгебра} --- это кольцо $R$, которое содержит $F$ в своём центре, поэтому $R$ является одновременно и кольцом и векторным пространством над $F$.

