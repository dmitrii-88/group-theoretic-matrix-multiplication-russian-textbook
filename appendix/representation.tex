\section{Несколько слов о представлениях}\label{representation}

Напомню, что если $V$ --- векторное пространство над полем $F$, то $GL(V)$ --- группа невырожденных линейных преобразований из $V$ на себя (относительно операции композиции), и если $n \in \mathbb{Z}^+$, то $GL_n(F)$ --- группа обратимых $n \times n$ матриц с элементами из $F$ (относительно матричного умножения).

Пусть $G$ --- конечная группа, $F$ --- поле, а $V$ --- векторное пространство над $F$.
\begin{definition}
  \textbf{Линейным представлением} $G$ называется любой гомоморфизм из $G$ в $GL(V)$. \textbf{Степень представления} --- это размерность пространства $V$.
\end{definition}
\begin{definition}
  \textbf{Матричным представлением} $G$ называется любой гомоморфизм из $G$ в $GL_n(F)$.
\end{definition}

Напомню, что если $V$ --- это конечное векторное пространство размерности $n$, то, зафиксировав базис $V$, мы получим изоморфизм 
\[
	GL(V) \cong GL_n(F).	
\]
Таким образом, любое линейное представление $G$ на векторном пространстве конечной размерности даст матричное представления, и наоборот. 

\begin{definition}
  Представление $\varphi: G \to GL(V)$ называется \textbf{приводимым}, если в $V$ существует нетривиальное подпространство $W$, инвариантое относительно всех $\varphi(g)$, то есть 
  \[
  	\forall w \in W: \; \varphi(g)(w) \in W.
  \]
  Если в $V$ нет таких нетривиальных инвариантных подпространств, то представление $\varphi$ называется \textbf{неприводимым}.
\end{definition}
\begin{definition}\label{def:characters_degree}
  \textbf{Степенями характеров} группы $G$ (обозначается $\{ d_i \}$) называют множество степеней всех её неприводимых представлений, то есть, если 
  \[
  	\{ \varphi_i \mid \varphi_i: G \to GL(V_i) \}
  \] --- это множество неприводимых представлений, то $\{ d_i \mid d_i = dim(V_i) \}$. Замечу, что у конечной группы их будет конечное же количество.
\end{definition}

В тексте часто используется факт, что порядок конечной группы равен сумме квадратов её степеней характеров, то есть
\[
	|G|=\sum\limits_i d_i^2.
\]
