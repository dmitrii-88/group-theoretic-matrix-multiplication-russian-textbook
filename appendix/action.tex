\section{Действия и регулярные представления}\label{action}

\begin{definition}
  \textbf{Групповое действие} группы $G$ на множестве $A$ --- это отображение из $G \times A$ в $A$ (записывается как $g \cdot a$ для всех $g \in G$ и $a \in A$), удовлетворяющее следующим свойствам
  \begin{enumerate}[(i)]
  \item $g_1 \cdot (g_2 \cdot a) = (g_1 g_2) \cdot a$ для всех $g_1, g_2 \in G, a \in A$
  \item $1 \cdot a = a$ для всех $a \in A$
  \end{enumerate}
\end{definition}

Пусть группа $G$ действует на множестве $A$. Для любого фиксированного $g \in G$ мы можем определить отображение $\sigma_g$ следующим образом
\[
	\sigma_g: A \rightarrow A,\ \sigma_g(a)=g \cdot a.	
\]
Для этой конструкции будут верны перечисленные ниже факты
\begin{enumerate}[(i)]
  \item для любого фиксированного $g \in G$ $\sigma_g$ будет перестановкой элементов из $A$
  \item отображение из $G$ в $S_A$, задаваемое как $g \mapsto \sigma_g$, является гомоморфизмом, который называется \textbf{перестановочным представлением} ассоциированным с данным действием.
\end{enumerate}

Легко увидеть, что если нам будет дан любой гомоморфизм $\varphi: G \rightarrow S_A$, тогда отображение из $G \times A$ в $A$ определяемое как 
\[
	g \cdot a = \varphi(g)(a),  \forall g \in G, \forall a \in A
\]
удовлетворяет свойствам группового действия $G$ на $A$. Поэтому действия группы $G$ на множестве $A$ и гомоморфизмы из $G$ в симметрическую группу $S_A$ находятся во взаимно однозначном соответствии (то есть по сути это одно и тоже понятие, выраженное разными словами).

Если быть точнее, то данное выше определение следовало бы назвать определением \textbf{левого действия}, так как групповой элемент появляется слева от элемента множества. Подобным же образом можно определить правое действие.

В качестве множества $A$, на которое действует $G$, можно взять саму группу $G$. Тогда действие, получаемое левыми умножениями на элемент $G$, будет называться \textbf{действием левыми сдвигами}, а соответствующие перестановочное представление --- \textbf{левым регулярным представлением}:  
  \[
  	g \mapsto
  	\left( 
  	\begin{array}{cccc}
	g_1 & g_2 & \ldots & g_{|G|} \\
	g g_1 & g g_2 & \ldots & g g_{|G|}
	\end{array} 
	\right) \in S_G 
  \]
  
Пусть далее $G$ --- это группа, действующая на множестве $A$.
\begin{definition}\label{def:stab}
     \textbf{Стабилизатором элемента $x \in A$} (обозначается $Stab_G(x)$ или $G_x$) называют множество элементов из $G$, которые оставляют $x$ на месте, то есть это множество 
     \[
     G_x = \left\{ g \in G \mid g \cdot x = x \right\}.
     \]
\end{definition}

Отношение $\sim$, заданное на $A$ следующим образом:
\[
	a \sim b \iff a = g \cdot b \text{ для какого-то } g \in G
\]
будет являться отношением эквивалентности на множестве $A$, поэтому будет также задавать разбиение этого множества.
\begin{definition}\label{def:orb}
      \textbf{Орбитой элемента $x \in A$} (обозначается $Orb_G(x)$ или $Gx$) называют класс эквивалентности для $x$ относительно $\sim$, определённого выше. Другими словами, орбита элемента $x$ --- это множество элементов из $A$, в которые может перейти $x$ под действием элементов из $G$, то есть 
      \[
      	Gx = \left\{ g \cdot x \mid g \in G \right\}.
      \]
\end{definition}

\begin{definition}\label{def:tr_action}
     Действие называют \textbf{транзитивным}, если у него всего одна орбита, то есть каждый элемент множества может перейти в любой другой под действием какого-то группового элемента.
\end{definition}
