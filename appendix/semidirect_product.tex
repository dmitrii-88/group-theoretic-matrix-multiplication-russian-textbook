\section{Полупрямое произведение}\label{semidirect_product}

Сначала будет нужно напомнить некоторые определения
\begin{definition}
  \textbf{Прямым произведением} $G_1 \times G_2 \times \ldots \times G_n$ групп $G_1, G_2, \ldots, G_n$ с операциями $\star_1, \star_2, \ldots, \star_n$ соответственно называется множество кортежей $(g_1, g_2, \ldots, g_n)$, где $g_i \in G_i$, с операцией, определяемой покомпонентно
  \[
   	(g_1, g_2, \ldots, g_n) (h_1, h_2, \ldots, h_n) = (g_1 \star_1 h_1, g_2 \star_2 h_2, \ldots, g_n \star_n h_n)
  \]
\end{definition}


Полупрямое произведение является обобщением понятия прямого произведения. Чтобы объяснить его происхождение начнём издалека. Пусть у нас есть группа $G$ содержащая подгруппы $H$ и $K$ такие что:
\begin{enumerate}[(a)]
  \item $H \unlhd G$ (но $K$ не обязательно должна быть нормальной в $G$) \label{it:a}
  \item $H \cap K = 1$ \label{it:b}
\end{enumerate}
По следствию 3.15 из \cite[94]{Dummit99} $HK=\{hk \mid h \in H, k \in K\}$ будет подгруппой $G$ и любой элемент $HK$ можно будет записать единственным способом как $hk$ для какого-то $h \in H$ и $k \in K$, то есть существует взаимно однозначное соответствие между $HK$ и множеством упорядоченных пар $(h,k)$, задаваемое как $hk \mapsto (h,k)$. Для двух элементов из $HK$ $h_1k_1$ и $h_2k_2$ их произведение в $G$ запишется следующим образом
\begin{equation}
	(h_1k_1)(h_2k_2)=h_1k_1h_2(k_1^{-1}k_1)k_2=h_1(k_1h_2k_1^{-1})k_1k_2=h_3k_3
\label{eq:1}
\end{equation}
где $h_3=h_1(k_1h_2k_1^{-1})$ и $k_3=k_1k_2$. Обратите внимание, что так как $H \unlhd G$, то $k_1h_2k_1^{-1} \in H$, поэтому $h_3 \in H$ и $k_3 \in K$.

Потому как $H$ нормальна в $G$, то группа $K$ будет действовать на $H$ сопряжениями
\[
	k \cdot h=khk^{-1} \mbox{ для } h \in H, k \in K
\] 
(символ $ \cdot $ используется для обозначения действия) поэтому \eqref{eq:1} можно записать как 
\[
	(h_1k_1)(h_2k_2)=(h_1(k_1 \cdot h_2))(k_1k_2)
\]
Действие сопряжениями $K$ на $H$ даёт гомоморфизм $\varphi$ из $K$ в группу автоморфизмов $Aut(H)$, поэтому умножение в $HK$ будет зависеть только от умножения в $H$, умножения в $K$ и гомоморфизма $\varphi$, то есть определяется исключительно в терминах $H$ и $K$.

Эти вычисления были сделаны при допущении, что уже существует группа $G$, содержащая подгруппы $H$ и $K$, такие что $H \unlhd G$ и $H \cap K = 1$. Основная идея полупрямого произведения в том, чтобы перевернуть эту конструкцию с ног на голову, а именно пусть у нас есть две абстрактные группы $H$ и $K$, и мы попытаемся задать группу, содержащую их (точнее их изоморфные копии), таким образом, что (\ref{it:a}) и (\ref{it:b}) выполняются. Для этого нам лишь нужно задать действие $K$ на $H$, то есть определить гомоморфизм $\varphi:K \rightarrow Aut(H)$.

\begin{theorem}\label{th:10} \cite[176]{Dummit99}
  Пусть $H$ и $K$ группы, а $\varphi$ - гомоморфизм из $K$ в $Aut(H)$. Обозначим через $ \cdot $ (левое) действие $K$ на $H$, определяемое с помощью $\varphi$. Пусть $G$ будет множеством упорядоченных пар $(h,k)$, где $ h \in H, k \in K$, и определим умножение в $G$ следующим способом:
  \[
  	(h_1, k_1)(h_2, k_2)=(h_1 (k_1 \cdot h_2), k_1 k_2)
  \]
  \begin{enumerate}[(1)]
    \item Это умножение превращает $G$ в группу порядка $|G|=|H| |K|$
    \item Множества $\{(h,1) \mid h \in H\}$ и $\{(1,k) \mid k \in K\}$ являются подгруппами $G$, и отображения $h \mapsto (h,1)\ h \in H$ и  $k \mapsto (1,k)\ k \in K$ являются изоморфизмами этих подгрупп с группами $H$ и $K$ соответственно:
   \[
   	H \cong \{(h,1) \mid h \in H\} \mbox{ и } K \cong \{(1,k) \mid k \in K\}
   \] \label{it:2}
   
   Отождествляя $H$ и $K$ с их изоморфными копиями в $G$, описанными в \ref{it:2}, мы получаем
   \item $H \unlhd G$
   \item $H \cap K = 1$
   \item для всех $h \in H$ и $k \in K$ $khk^{-1}=k \cdot h= \varphi(k)(h)$
  \end{enumerate}
\end{theorem}

\begin{definition}
  Пусть $H$ и $K$ --- группы, и пусть $\varphi$ --- гомоморфизм из $K$ в $Aut(H)$. Группа, описанная в теореме \ref{th:10}, называется \textbf{полупрямым произведением} $H$ и $K$ относительно $\varphi$ и обозначается $H \rtimes_{\varphi} K$ (если нет опасности запутаться, будем писать $H \rtimes K$)
\end{definition}

Используя отождествления $H$ с подмножеством $\{(h,1) \mid h \in H\}$ и $K$ с $\{(1,k) \mid k \in K\}$, можно записывать элементы $G = H \rtimes K$ не только как $(h, k)$, но и в виде простого произведения $hk$, при этом ещё будут существовать такие $h' \in H$ и $k' \in K$, что $(h, k) = hk = k' h'$. 

Обозначение для полупрямого произведения выбрано таким образом, чтобы напоминать нам, что копия $H$ в $H \rtimes K$ является нормальным <<фактором>>, и что конструкция полупрямого произведения несимметрична относительно $H$ и $K$ (в отличие от прямого произведения). Нужно также сказать, когда полупрямое произведение превращается в прямое.

\begin{prop}\cite[177]{Dummit99}
  Пусть $H$ и $K$ группы и пусть $\varphi:K \rightarrow Aut(H)$ гомоморфизм. Тогда следующие утверждения эквивалентны:
  \begin{enumerate}[(1)]
    \item тождественное отображение (множеств) из $H \rtimes K$ в $H \times K$ является групповым гомоморфизмом (следовательно изоморфизмом)
    \item $\varphi$ является тривиальным гомоморфизмом из $K$ в $Aut(H)$, то есть 
    \[
    	\forall k \in K \ \varphi(k)=1 \in Aut(H)
    \]
    \item $K \unlhd H \rtimes K$
  \end{enumerate}
\end{prop}

В тексте часто упоминается одна из разновидностей полупрямого произведения, а именно конструкция называемая сплетением.
\begin{definition}\label{def:wreath_product}
  Пусть $K$ и $L$ --- группы, $n \in \mathbb{Z}^+$, $\rho:K \rightarrow S_n$ --- гомоморфизм, $H=L \times \ldots \times L$ --- прямое произведение $n$ копий $L$. Пусть $\psi: S_n \rightarrow Aut(H)$ --- инъективный гомоморфизм, который позволяет элементам из $S_n$ переставлять факторы $H$. \textbf{Сплетением групп $L$ и $K$} (обозначается $L \wr K$) называется полупрямое произведение $H \rtimes K$ относительно гомоморфизма $\psi \circ \rho: K \to Aut(H)$. Сплетение зависит от выбора гомоморфизма $\rho$, если он явно не задан, то предполагаем, что $\rho$ --- это левое регулярное представление $K$.
\end{definition}











