\begin{tabular}{lp{11cm}}
  \textit{Обозначение} & \textit{Смысл} \\
  \hline
  $\ln x$ & натуральный лографм: $\log_e x$  \\
  $\left\lceil \frac{g}{f} \right\rceil$ & функция потолок, то есть округление $\frac{g}{f}$ до ближайшего целого в большую сторону (стр. \pageref{lem:bi:6.5})\\
  $\left\langle n,m,p \right\rangle_F$ & тензор прямоугольного матричного умножения для $n \times m$ и $m \times p$-матриц с элементами из поля $F$. Если $F = \mathbb{C}$, то будем писать просто $\left\langle n,m,p \right\rangle$ (стр. \pageref{matrix_tensor})\\
  $R(t)$ & ранг тензора $t$ (стр. \pageref{def:tensor_rank})\\
  $\omega$ & экспонента матричного умножения над полем $\mathbb{C}$ (стр. \pageref{def:omega})\\
  $Cyc_k$ & циклическая группа порядка $k$ со сложением в качестве групповой операции\\
  $[k]$ & множество $\left\{ 1,2, \dotsc, k \right\}$, где $k$ натуральное число \\
  $A \setminus B$ & множество ${\{a \in A \mid a \notin B\}}$\\
  $A - B$ & множество $\{a - b \mid a \in A, b \in B\}$, где $A$ и $B$ являются подмножествами абелевой группы\\
  $S_M$ & группа перестановок элементов множества $M$\\
  $S_n$ & группа перестановок элементов множества $[n]$\\
  $R[G]$ & групповая алгебра над $G$ с коэффициентами из $R$, где $G$ --- группа, а $R$ --- кольцо (стр. \pageref{group_algebra})\\
  $g \cdot a$ & элемент группы $G$ \textit{действует слева} на элемент множества $A$ (стр. \pageref{action})\\
  $a^g$ & элемент группы $G$ \textit{действует справа} на элемент множества $A$ (стр. \pageref{action})\\
  $H \rtimes G, \; G \ltimes H$ & полупрямое произведение (стр. \pageref{semidirect_product})\\
  $O(g(x))$ & $O$-большое от функции $g(x)$ (стр. \pageref{o_notation})\\
  $o(1)$ & функция, стремящаяся к 0 при стремлении к бесконечности какой-то переменной из контекста, например, таковой будет $1/x$ при $x \to \infty$ (стр. \pageref{little-oh})\\
  $\binom{N}{\mu}$ & мультиномиальный коэффициент, то есть $\frac{N!}{\mu_1! \dotsm \mu_n!}$ (стр. \pageref{lem:05:1.1})\\
  $2^S$ & множество всех подмножеств $S$ (стр. \pageref{def:05:6.5})\\
  $fin(S)$ & множества всех \textit{конечных} подмножеств $S$ (стр. \pageref{sec:general_model})\\
  $Q(S)$ & множество правых частных $S$, то есть $\{ s_1 s_2^{-1} \mid s_1, s_2 \in S\}$ (стр. \pageref{sec:realizing})\\
\end{tabular}
